\documentclass[article]{beamer}% para se tiver muitas sections
%\documentclass[11pt,compress,xcolor=dvipsnames]{beamer}
%---------------------------------------------------------------------
% Color and themes
%---------------------------------------------------------------------
\definecolor{ipb}{rgb}{0.36, 0.54,0.66}
%\definecolor{royalazure}{rgb}{0.07, 0.04, 0.56}
\definecolor{royalazure}{RGB}{8,71,155}
\setbeamercolor{footline}{bg=ipb}
\setbeamercolor{frametitle}{bg=ipb,fg=white}
\setbeamercolor{title}{bg=ipb}
\setbeamerfont{frametitle}{size=\large}
\setbeamertemplate{bibliography item}[text]
\setbeamertemplate{caption}[numbered]
\setbeamertemplate{blocks}[rounded][shadow]
\setbeamercolor{palette primary}{use=structure,fg=white,bg=structure.fg}
\setbeamercolor{palette secondary}{use=structure,fg=white,bg=structure.fg!75!black}
\setbeamercolor{palette tertiary}{use=structure,fg=white,bg=structure.fg!50!black}
\setbeamercolor{palette quaternary}{fg=white,bg=structure.fg!50!black}
\setbeamercolor*{sidebar}{use=structure,bg=structure.fg}
\setbeamercolor{titlelike}{parent=palette primary}
\setbeamercolor{block title}{bg=ipb,fg=white}
\setbeamercolor*{block title example}{use={normal text,example text},bg=white,fg=ipb}
\setbeamercolor{fine separation line}{}
\setbeamercolor{item projected}{fg=black}
\setbeamercolor{palette sidebar primary}{use=normal text,fg=normal text.fg}
\setbeamercolor{palette sidebar quaternary}{use=structure,fg=structure.fg}
\setbeamercolor{palette sidebar secondary}{use=structure,fg=structure.fg}
\setbeamercolor{palette sidebar tertiary}{use=normal text,fg=normal text.fg}
\setbeamercolor{palette sidebar quaternary}{fg=ipb}
%\setbeamercolor{section in sidebar}{fg=brown}
%\setbeamercolor{section in sidebar shaded}{fg=grey}
\setbeamercolor{sidebar}{bg=ipb}
\setbeamercolor{sidebar}{parent=palette primary}
\setbeamercolor{structure}{fg=ipb}
%\setbeamercolor{subsection in sidebar}{fg=brown}
%\setbeamercolor{subsection in sidebar shaded}{fg=grey}
\setbeamercolor{section in head/foot}{fg=white,bg=royalazure}
%\setbeamercolor{subsection in head/foot}{fg=white,bg=royalazure}
\usetheme{Warsaw}
%---------------------------------------------------------------------
% Footline
%---------------------------------------------------------------------
\setbeamertemplate{footline}
 {\leavevmode
\hbox{
   \begin{beamercolorbox}[wd=0.49\paperwidth,ht=2.25ex,dp=1ex,leftskip=0.3cm]{author in head/foot}%
     \usebeamerfont{author in head/foot}\textcolor{white}{\insertshortauthor}
   \end{beamercolorbox}%
   \begin{beamercolorbox}[wd=0.34\paperwidth,ht=2.25ex,dp=1ex,center]{title in head/foot}
     \usebeamerfont{title in head/foot}\textcolor{white}{\insertshorttitle}
   \end{beamercolorbox}%
   \begin{beamercolorbox}[wd=0.17\paperwidth,ht=2.25ex,dp=1ex,leftskip=0.3cm,rightskip=0.3cm]{title in head/foot}%
   \hfill\usebeamerfont{page number in head/foot}
   \insertframenumber{} / \textcolor{white}{\inserttotalframenumber}
   \end{beamercolorbox}}
}
%---------------------------------------------------------------------
% Packages
%---------------------------------------------------------------------
\usefonttheme[]{serif}
\usepackage{amsmath, latexsym, color, graphicx, amssymb, bm, here}
\usepackage{epsf, epsfig, pifont,tikz,subfigure}
\usepackage{graphics, calrsfs}
\usepackage{times}
\usepackage{fancybox,calc}
\usepackage{palatino,mathpazo}
\usepackage{amsfonts}
\usepackage{wrapfig}
\usepackage{multicol}
\usepackage{sidecap}
\usepackage{academicons}
%\usepackage{pdfauthor}
%\usepackage{pdfcreator}
\usepackage{hyperref}
\usepackage{listings}
\usepackage[portuguese]{babel}
\usepackage[]{hyperref}
%---------------------------------------------------------------------
% Definir
%---------------------------------------------------------------------
\def\inst#1{\unskip$^{#1}$}
\def\orcidID#1{\unskip$^{[#1]}$}
\def\fnmsep{\unskip$^,$}
\def\email#1{{\tt#1}}

%---------------------------------------------------------------------
% Dados
%---------------------------------------------------------------------
\title{Apresentação IPB através em \LaTeX{} \emph{Beamer}}
%opçao para 1 autor
%\author{Primeiro Autor~\orcidID{a12345}}
%opçao para 2 autor
\author{Primeiro Autor~\orcidID{a12345} \and Segundo Autor~\orcidID{a12345}}
%opcao para 3 autor
%\author{Primeiro Autor~\orcidID{a12345}\\
%        Segundo Autor~\orcidID{a12345}\\
 %       Terceiro Autor~\orcidID{a12345}
  %      }
%\institute{Instituto Politécnico de Bragança- Escola Superior de Tecnologia e Gestão\\
            %\vspace{0.3cm}
            %Licenciatura em Curso}
\institute{Instituto Politécnico de Bragança - Escola Superior de Tecnologia e Gestão\\
            \vspace{0.3cm}
            Mestrado em Curso
            }
\date{\vfill\scriptsize{\today}\\\vspace{0.4cm}\includegraphics[scale=0.3]{Imagens/logo.png}}
%---------------------------------------------------------------------
% Index
%---------------------------------------------------------------------
\AtBeginSection[]
{
  \begin{frame}{Conteúdo}
    \tableofcontents[currentsection]
  \end{frame}
}
\definecolor{azuel}{rgb}{0.07, 0.04, 0.56}
	\definecolor{backcolour}{rgb}{0.95,0.95,0.92}
	\definecolor{codegreen}{rgb}{0,0.6,0}
	\definecolor{mygreen}{RGB}{28,172,0}
	\definecolor{mylilas}{RGB}{170,55,241}
	\definecolor{codegray}{rgb}{0.5,0.5,0.5}
	\definecolor{codepurple}{rgb}{0.58,0,0.82}
	\lstdefinestyle{list2}{language=Matlab,% lst
		basicstyle=\color{black},
		backgroundcolor=\color{white},
		breaklines=true,
		breakatwhitespace=false,
		keepspaces=true,
		morekeywords={matlab2tikz},
		showspaces=false, 
		showtabs=false,
		tabsize=2,
		rulecolor=\color{black},
		frame=single,
		keywordstyle={\scriptsize\color{blue}},
		morekeywords=[2]{1}, 
		keywordstyle=[2]{\scriptsize\color{black}},
		identifierstyle={\scriptsize\color{black}},%
		stringstyle={\scriptsize\color{mylilas}},
		commentstyle={\scriptsize\color{mygreen}},
		showstringspaces=false,
		numbers=left,%
		numberstyle={\scriptsize\color{black}},
		numbersep=10pt, 
		emph=[1]{for,end,break},emphstyle=[1]\scriptsize\color{blue}, 
	}
	%\renewcommand{\lstlistingname}{Alg.}

\begin{document}


\maketitle


\begin{frame}
    \frametitle{Conteúdo}
    \tableofcontents
\end{frame}

\section{Introdução}


\begin{frame}
    \frametitle{Sobre \emph{Beamer}}
\begin{itemize}
    \item Em $LaTex$, o $Beamer$ serve para fazer apresentações;
    \item  Diferente de programas $WYSWYG$;
    \item  Uma apresentação $Beamer$ é como qualquer outro documento LaTeX, contém:
    \begin{itemize}
        \item Preâmbulo e um corpo;
        \item O preâmbulo pode-se dizer que é o ``índice'', tipo do documento e pacotes;
        \item O corpo contém $sections$ e $subsections$;
        \item Os dispositivos deverão ser estruturados utilizando ambientes de $item$ e $enumerate$, ou texto simples (curto).
\end{itemize} 
\end{itemize}
\end{frame}
%%%%%%%%%%%%%%%%%%%%%%%%%%%%%%Frame 4
\section{Estrutura do Documento}
\subsection{Exemplos}


%%%%%%%%%%%%%%%%%%%%%%%%%%%%%%Frame 5

\begin{frame}
\frametitle{Simples Exemplo}
\scriptsize{
\textbackslash documentclass\{beamer\}\\ \vspace{0.1cm}

\textbackslash title\{\textbackslash Apresentação IPB através em \LaTeX\{\} \emph{Beamer}\} \\
\textbackslash author\{Primeiro Autor\}\\
\textbackslash institute\{Licentura em Curso \textbackslash\textbackslash Instituto Politécnico de Bragança\}\\
\textbackslash date\{\textbackslash scriptsize\{\textbackslash today\}\}
 \\ \vspace{0.3cm}

\textbackslash begin\{document\} \\ \vspace{0.1cm}

\hspace{0.5cm}\textbackslash maketitle\\ \vspace{0.1cm}

\hspace{0.5cm}\textbackslash section\{Nome da $Section$\}\\
\hspace{0.5cm}\textbackslash begin\{frame\}\\
\hspace{1cm}\textbackslash frametitle\{Nome da $Frame$\}\\
\hspace{1cm}\textbackslash Here is one slide.\\
\hspace{0.5cm}\textbackslash end\{frame\}\\ \vspace{0.1cm}

\textbackslash end\{document\}\\
}

\end{frame}


\subsection{Corpo}

\begin{frame}
\frametitle{Título da \emph{frame}}
Usar os seguintes comandos no preâmbulo:
\begin{block}{Comando para o título da $frame$}
\scriptsize{\tt{
\textbackslash title\{Título\} \\
\textbackslash subtitle\{Sub Título\} \\
\textbackslash author\{Nome\}\\
\textbackslash institute\{Nome do Curso\}\\
\textbackslash date\{Data\} \\
}}
\end{block}
e o comando \textbackslash \tt{maketitle}, 
\begin{block}{Gerar a $frame$ no corpo do documento}
\scriptsize{\tt{
\textbackslash begin\{document\} \\ 
\textbackslash maketitle\\ 
\vdots
\textbackslash end\{document\}\\
}}
\end{block}
\end{frame}

\begin{frame}
\frametitle{Criar \emph{frame}}
Para criar $frames$ utiliza-se os seguintes comandos:
\begin{block}{}
\scriptsize{\tt{
\textbackslash begin\{frame\} \\
\textbackslash frametitle\{...\} \\
\textbackslash framesubtitle\{...\} \\
...\\
\textbackslash end\{frame\}
}}
\end{block}
Ou, 
\begin{block}{}
\scriptsize{\tt{
\textbackslash frame\{ \\
\textbackslash frametitle\{...\}\\
\textbackslash framesubtitle\{...\}\\
...\\
\}
}}
\end{block}
\end{frame}

\begin{frame}
\frametitle{\emph{section} e \emph{subsections}}
\vspace{1cm}
\scriptsize{\tt{
...\\
\textbackslash end\{frame\}\\
\vspace{0.5cm}
\alert{
\textbackslash section\{section name\}\\
\textbackslash subsection\{subsection name\}\\
\textbackslash subsubsection\{subsubsection name\}}\\
\vspace{0.5cm}
\textbackslash begin\{frame\}\\
...\\
}
}
\end{frame}

\begin{frame}
\frametitle{Conteúdo} 

\begin{block}{Manual}
\scriptsize{\tt{
...\\
\textbackslash end\{frame\}\\
\vspace{0.1cm}
\textbackslash begin\{frame\}\\
\textbackslash frametitle\{Lista de Conteúdo\}\\
\textbackslash tableofcontents[currentsection]\\
\textbackslash end\{frame\}\\
\vspace{0.2cm}
\textbackslash begin\{frame\}\\
...\\
}}
\end{block}

\begin{block}{Automático}
\scriptsize{\tt{
...\\
\textbackslash AtBeginSection[]\{\\
\textbackslash begin\{frame\}\{Outline\}\\
\textbackslash tableofcontents[currentsection]\\
\textbackslash end\{frame\}\}\\
...\\
}}
\end{block}
\end{frame}

\section{Estratura da \emph{frame}}
\subsection{Criar colunas}


%%%%%%%%%%%%%%%%%%%%%%%%%%%%%%Frame 15
\begin{frame}
\frametitle{Duas $colums$}
\begin{block}{Exemplo de duas $colums$}\scriptsize{
\textbackslash begin\{columns\}\\
\textbackslash column\{.4\textbackslash textwidth\}\\
Left column\\
\textbackslash column\{.4\textbackslash textwidth\}\\
Right column\\
\textbackslash end\{columns\}\\
}
\end{block}
\vspace{1cm}
%\onslide<2>{
\begin{columns}
\column{.4\textwidth}
Left column\\
\column{.4\textwidth}
Right column\\
\end{columns}
%}
\end{frame}

\subsection{Criar \emph{Block's}}
\begin{frame}
\frametitle{Block}

\begin{block}{Beamer Introduction}
Beamer is a { \LaTeX} class.
\end{block}

\end{frame}

\subsection{Criar listas}
\begin{frame}[fragile]
\frametitle{\emph{itemize}}

\begin{columns}
\column{.5\textwidth}
\begin{block}
\scriptsize{
\begin{verbatim}
\begin{itemize}
\item The first one.
\item The second one.
\begin{itemize}
\item The larger one.
\item The smaller one.
\end{itemize}
\item The third one.
\end{itemize}
\end{verbatim}
}
\end{block}
\column{.5\textwidth}

\begin{itemize}
\item The first one.
\item The second one.
\begin{itemize}
\item The larger one.
\item The smaller one.
\end{itemize}
\item The third one.
\end{itemize}
\end{columns}
\vspace{1.5cm}
\href{https://tug.ctan.org/macros/latex/contrib/beamer/doc/beameruserguide.pdf}{Clique aqui para mais informações.}
\end{frame}

\begin{frame}[fragile]
\frametitle{\emph{enumerate}}

\begin{columns}
\column{.5\textwidth}
\begin{block}
\scriptsize{
\begin{verbatim}
\begin{enumerate}
\item The first one.
\item The second one.
\begin{enumerate}
\item The large one.
\item The small one.
\end{enumerate}
\item The third one.
\end{enumerate}
\end{verbatim}
}
\end{block}
\column{.5\textwidth}

\begin{enumerate}
\item The first one.
\item The second one.
\begin{enumerate}
\item The large one.
\item The small one.
\end{enumerate}
\item The third one.
\end{enumerate}
\end{columns}
\vspace{1.5cm}
\href{https://tug.ctan.org/macros/latex/contrib/beamer/doc/beameruserguide.pdf}{Clique aqui para mais informações.}

\end{frame}
%\section{Referências}
%\begin{frame}[allowframebreaks]
 %   \bibliography{bibliography}
    %\bibliographystyle{ieeetr}
  %  \nocite{*} % used here because no citation happens in slides
    % if there are too many try use:
   %  \tiny\bibliographystyle{ieeetr}
%\end{frame}

\end{document}