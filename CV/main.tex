%%%%
% MTecknology's Resume
%%%%
% Author: Michael Lustfield
% License: CC-BY-4
% - https://creativecommons.org/licenses/by/4.0/legalcode.txt
%%%%

\documentclass[letterpaper,10pt]{article}
%%%%%%%%%%%%%%%%%%%%%%%
%% BEGIN_FILE: mteck.sty
%% NOTE: Everything between here and END_FILE can
%% be relocated to "mteck.sty" and then included with:
%\usepackage{mteck}

% Dependencies
% NOTE: Some packages (lastpage, hyperref) require second build!
\usepackage[empty]{fullpage}
\usepackage{titlesec}
\usepackage{enumitem}
\usepackage{fancyhdr}
\usepackage{fontawesome5}
\usepackage{multicol}
\usepackage{bookmark}
\usepackage{lastpage,amsmath,mathtools}

% Sans-Serif
%\usepackage[sfdefault]{FiraSans}
%\usepackage[sfdefault]{roboto}
%\usepackage[sfdefault]{noto-sans}
%\usepackage[default]{sourcesanspro}

% Serif
\usepackage{CormorantGaramond}
\usepackage{charter}

% Colors
% Use with \color{Name}
% Can wrap entire heading with {\color{accent} [...] }
\usepackage{xcolor}
\usepackage{hyperref}
\hypersetup{colorlinks,urlcolor=myblue}
\definecolor{myblue}{HTML}{3972b3}       
\definecolor{accentTitle}{HTML}{3972b3}
\definecolor{accentText}{HTML}{3972b3}
\definecolor{accentLine}{HTML}{000000}

% Misc. Options
\pagestyle{fancy}
\fancyhf{}
\fancyfoot{}
\renewcommand{\headrulewidth}{0pt}
\renewcommand{\footrulewidth}{0pt}
\urlstyle{same}

% Adjust Margins
\addtolength{\oddsidemargin}{-0.7in}
\addtolength{\evensidemargin}{-0.5in}
\addtolength{\textwidth}{1.19in}
\addtolength{\topmargin}{-0.7in}
\addtolength{\textheight}{1.4in}

\setlength{\multicolsep}{-3.0pt}
\setlength{\columnsep}{-1pt}
\setlength{\tabcolsep}{0pt}
\setlength{\footskip}{3.7pt}
\raggedbottom
\raggedright

% ATS Readability
\input{glyphtounicode}
\pdfgentounicode=1

%-----------------%
% Custom Commands %
%-----------------%
\newcommand{\sbt}{\,\begin{picture}(-1,1)(-1,-3)\circle*{3}\end{picture}\ }
% Centered title along with subtitle between HR
% Usage: \documentTitle{Name}{Details}
\newcommand{\documentTitle}[2]{
  \begin{center}
    {\Huge\scshape\color{accentTitle} #1}
    \vspace{10pt}
    {\color{accentLine} \hrule}
    \vspace{2pt}
    %{\footnotesize\color{accentTitle} #2}
    \footnotesize{#2}
    \vspace{2pt}
    {\color{accentLine} \hrule}
  \end{center}
}

% Create a footer with correct padding
% Usage: \documentFooter{\thepage of X}
\newcommand{\documentFooter}[1]{
  \setlength{\footskip}{10.25pt}
  \fancyfoot[C]{\footnotesize #1}
}

% Simple wrapper to set up page numbering
% Usage: \numberedPages
% WARNING: Must run pdflatex twice!
\newcommand{\numberedPages}{
  \documentFooter{\thepage/\pageref{LastPage}}
}

% Section heading with horizontal rule
% Usage: \section{Title}
\titleformat{\section}{
  \vspace{-5pt}
  \color{accentText}
  \scshape\raggedright\large\bfseries
}{}{0em}{}[\color{accentLine}\titlerule]

% Section heading with separator and content on same line
% Usage: \tinysection{Title}
\newcommand{\tinysection}[1]{
  \phantomsection
  \addcontentsline{toc}{section}{#1}
  {\scshape\large{\bfseries\color{accentText}#1} {\color{accentLine} |}}
}

% Indented line with arguments left/right aligned in document
% Usage: \heading{Left}{Right}
\newcommand{\heading}[2]{
  \hspace{0pt}#1\hfill#2\\
}

% Adds \textbf to \heading
\newcommand{\headingBf}[2]{
  \heading{\textbf{#1}}{\textbf{#2}}
}

% Adds \textit to \heading
\newcommand{\headingIt}[2]{
  \heading{\textit{#1}}{\textit{#2}}
}

% Template for itemized lists
% Usage: \begin{resume_list} [items] \end{resume_list}
\newenvironment{resume_list}{
  \vspace{-7pt}
  \begin{itemize}[itemsep=-2px, parsep=1pt, leftmargin=30pt, label={$\circ$}]
}{
  \end{itemize}
  %\vspace{-2pt}
}

% Formats an item to use as an itemized title
% Usage: \itemTitle{Title}
\newcommand{\itemTitle}[1]{
  \item[] \underline{#1}\vspace{5pt}
}

% Bullets used in itemized lists
\renewcommand\labelitemi{--}

%% END_FILE: mteck.sty
%%%%%%%%%%%%%%%%%%%%%%


%===================%
% John Doe's Resume %
%===================%

%\numberedPages % NOTE: lastpage requires a second build
%\documentFooter{\thepage of 2} % Does similar without using lastpage
\begin{document}

  %---------%
  % Heading %
  %---------%

  \documentTitle{Soham Chatterjee}{
    % Web Version
    %\raisebox{-0.05\height} \faPhone\ [redacted - web copy] ~
    %\raisebox{-0.15\height} \faEnvelope\ [redacted - web copy] ~
    %%
    \href{tel:+919433548242}{
      \raisebox{-0.05\height} \faPhone\ +91 94335 48242} ~
    \href{mailto:sohamc@cmi.ac.in}{
      \raisebox{-0.15\height} \faEnvelope\ sohamc@cmi.ac.in} ~
%    \href{https://linkedin.com/in/soham-chatterjee-513b77202/}{
%      \raisebox{-0.15\height} \faLinkedin\ soham-chatterjee-513b77202} ~
    \href{https://github.com/sohamch08}{
      \raisebox{-0.15\height} \faGithub\ sohamch08} ~
   \href{https://sohamch08.github.io}{
  	\raisebox{-0.15\height} \faGlobe\ sohamch08.github.io}
  }

  %---------%
  % Summary %
  %---------%
%
%  \tinysection{Summary}
%  Simplified version of a monstrosity that I built back in college using current best practices.

  %--------%
  % Skills %
  %--------%

  \section{Education}

%  \begin{multicols}{2}
%    \begin{itemize}[itemsep=-2px, parsep=1pt, leftmargin=75pt]
%      \item[\textbf{Automation}] SaltStack, Ansible, Chef, Puppet
%      \item[\textbf{Cloud}] Salt-Cloud, Linode, GCP, AWS
%      \item[\textbf{Languages}] Python, Bash, PHP, Perl, VB/C\#.Net
%      \item[\textbf{OS}] Debian, Ubuntu, CentOS, BSD, AIX
%      \item[\textbf{Policies}] CIS, SOC2, PCI-DSS, GDPR, ITIL
%      \item[\textbf{Testing}] Pytest, Docker, CircleCI, Jenkins, Inspec
%    \end{itemize}
%  \end{multicols}

\begin{resume_list}
	\item \headingBf{Chennai Mathematical Institute}{2021 -- 2024}
	\headingIt{B.Sc. - Mathematics and Computer Science}{Chennai, India}
	\item \headingBf{Baranagar Narendranath Vidyamandir}{2018 -- 2020}
	\headingIt{Higher Secondary ($12^{th}$ Standard) Education}{Kolkata, India}
	\item \headingBf{Baranagar Ramakrishna Mission Ashrama High School}{2008 -- 2018}
	\headingIt{Secondary ($12^{th}$ Standard) Education}{Kolkata, India}
\end{resume_list}

%------------%
% Experience %
%------------%
\section{Academic Achievements}
\begin{resume_list}
	\item \headingBf{GS Exam, I-PhD, Computer Science, 2024}{TIFR Mumbai, India}
	\headingIt{Nation wide entrance exam in Computer Science for Tata Insititute of Fundamental Reseach. Only 2 people got selected.}{}
	
	\item \headingBf{JEST, I-PhD, Theoretical Computer Science, 2024 - Rank 5}{IMSC, India}
	\headingIt{Nation wide entrance exam in Computer Science for Insitute of Mathematical Sciences}{}
	\item \headingBf{NEST, B.Sc., 2021}{NISER, India}
	\headingIt{Nation wide bachelors entrance exam for National Institute of Science Education and Research }{}
	\item \headingBf{WBJEE, B.Tech, 2020 - Rank 1893}{WBJEEB}
	\headingIt{Joint Entrance exam for B.Tech for West Bengal state}{}
	\item \headingBf{$\boldsymbol{12^{th}}$ Statistics Olympiad, 2020 - Rank 28}{AIMSCS}
	\headingIt{Organised by C R Rao Advanced Institute of Mathematics, Statistics and Computer Science}{}
	
\end{resume_list}
\section{Internships}
\begin{resume_list}
	\item \headingBf{Polyhedral Combinatorics and Derandomization of Isolation Lemma}{}
	\headingIt{Supervisor: \href{https://www.cse.iitb.ac.in/~rgurjar/}{Rohit Gurjar}, IIT Mumbai}{May - Jul, 2024}
\begin{itemize}
	\item I read the papers\begin{itemize}[label=$\sbt$]
		\item  'Bipartite Perfect Matching is in \textsc{Quasi-NC}' by Fenner, Gurjar and Thierauf
		\item 'Linear Matroid Intersection Is in \textsc{Quasi-NC}' by Gurjar and Thierauf
		\item `Fractional Linear Matroid Matching is in \textsc{Quasi-NC}' by Gurjar, Oki and Raj
	\end{itemize}Learned how the idea of giving nonzero circulations to cycles and bounding number of integral vectors (corresponding those cycles) twice the size of smallest vector helps construct an isolating weights for bipartite perfect matching polytope to fractional matroid matching polytopes
	\item Additionally I read about isolating a path connecting the source vertex and sink vertex in a black-box layered graph from the paper `Derandomizing Isolation in Space-Bounded Settings' by Melkebeek and Prakriya.
\end{itemize}

\item \headingBf{Quantum Property Testing of Junta Functions and Partially Symmetric Functions.}{}
\headingIt{Supervisor: \href{https://sites.google.com/site/homepagearijitghosh/}{Arijit Ghosh}, Indian Statistical Institute, Kolkata}{Dec, 2024 -- Going on}
\begin{itemize}
	\item I learned about Fourier Analysis of Quantum Boolean Functions and Quantum algorithms for Testing and Learning Stabilizer States from Quantum boolean functions' by Montanaro and Osborne 
	\item Also  learned about Classical Junta Testing from Eric Blais' paper Testing Juntas Nearly Optimally and then read about Quantum Junta Testing Algorithm from `Testing and Learning Quantum Juntas Nearly Optimally' by Chen, Nadimpalli and Yuen
	\item And I learned about Partially Symmetric Boolean Functions and it's classical algorthm of testing partially symmetric functions from the paper `Partially Symmetric Functions are Efficiently Isomorphism-Testable' by Blais, Weinstein and Yoshida and we were trying to come up with a Quantum Algorithm for Testing Partially Symmetric Boolean Functions.
\end{itemize}


\item \headingBf{Factorization of Arithmetic Circuits in Algebraic Complexity Theory}{}
\headingIt{Supervisor: \href{https://www.cse.iitk.ac.in/users/nitin/}{Nitin Saxena}, IIT Kanpur}{May - Jul, 2022}
\begin{itemize}
	\item I read `Discovering the roots: Uniform closure results for algebraic classes under factoring' by Dutta, Saxena and Sinhababu  where I learned factorizing multivariate arithmetic circuits and \textsc{VP} closure under factorization. 
	\item  Also read the Kaltofen's proof of \textsc{VP} closed under factorization.
	\item Also leanred how Polynomial Identity Testing and Multivariate Factorizations are equivalent from `Equivalence of Polynomial Identity Testing and Deterministic Multivariate Polynomial Factorization' by  Kopparty, Saraf and Shpilka
	\item I also read how \textsc{VBP} is closed under factorization from  Sinhababu and Tierauf's paper `Factorization of Polynomials given by Arithmetic Branching Programs'
	\item Learned about the difficulties about proving factor closure for \textsc{VF} from the above mentioned two papers. I read about factorization of formulas with individual degree bounded form the paper `Factors of low individual degree polynomials' by Rafael Oliveira and we were trying to remove the condition for formulas
\end{itemize}


\item \headingBf{Computational Number Theroy and Algebra for Algebraic Comlexity Theory.}{}
\headingIt{Supervisor: \href{https://www.cse.iitk.ac.in/users/nitin/}{Nitin Saxena}, IIT Kanpur}{Dec - Jan, 2022}
\begin{itemize}
	\item I learned about Computational Number Theory and Algebra from Nitin Saxena's Course and read the book `Modern Computer Algebra' by Von Zur Gathen and Jurgen Gerhard
	\item Also I learned about Arithmetic Circuits from \href{https://www.nowpublishers.com/article/Details/TCS-039}{Amir Shpilka's Survey} and 
	\href{https://github.com/dasarpmar/lowerbounds-survey}{Ramprasad Saptharishi's Survey} on Arithmetic Circuits.
\end{itemize}
\end{resume_list}
\section{Course Projects}
\begin{resume_list}
	\item \textbf{Presentation on Iterated Mod Problem:\hfill \href{https://sohamch08.github.io/assets/parallel-presentation-iterated-mod.pdf}{Slides}}
	
	Presented the paper ``\href{https://www.sciencedirect.com/science/article/pii/0890540189900084}{Iterated Mod Problem}" by Karloff and Ruzzo in Parallel Algorithms and Complexity course at CMI
	\item \textbf{Report on Algebraic Geometric Codes:\hfill \href{https://sohamch08.github.io/assets/act-report.pdf}{Report}}
	
	Followed the survey ``\href{https://dl.acm.org/doi/abs/10.5555/334156.334207}{Algebraic-geometry codes}" by Blake, Heegardm H\o{}holdt, Wei and Gil Cohen's \href{https://www.gilcohen.org/2022-ag-codes}{Course}
	\item \textbf{Qiskit Implementation of Quantum Circuit of Modular Exponentiation:\hfill \href{https://github.com/bluecheese123/-Best_Project-}{Link}}
	
	Implemented the paper ``\href{https://arxiv.org/pdf/quant-ph/9511018.pdf}{Quantum Networks for Elementary Arithmetic Operations}" by Vedral, Barenco and Ekert
	\item \textbf{Qiskit Implementation of Kushlevitz and Mansour Algorithm:\hfill \href{https://github.com/sohamch08/Qiskit-Quantum-Algo/blob/master/Kushlevitz and Mansour Algorithm.ipynb}{Link}}
	
	Implemented the paper ``\href{https://dl.acm.org/doi/pdf/10.1145/103418.103466}{Learning Decision Trees Using the Fourier Spectrum}" bu Kushilevitz and Mansour
	\item \textbf{Qiskit Implementation of Some Quantum Algorithms:\hfill \href{https://github.com/sohamch08/Qiskit-Quantum-Algo}{Link}}
	
	Implemented Iterative Phase Estimation and Grover Search for $2\times 2$ sudoku
\end{resume_list}

\section{Relevant Course Work}


\begin{multicols}{2}
	\headingBf{Math Courses:}{}
	
\begin{resume_list}
	\itemTitle{Algebra}
	\item Linear Algebra (Algebra 1)
	\item Group Theory (Algebra 2)
	\item Ring and  Field Theory (Algebra 3)
	\item Commutative Algebra
	\itemTitle{Analysis}
	\item Real Analysis (Analysis 1)
	\item Analysis in Euclidean Space (Analysis 2)
	\item Analysis in Metric Space (Analysis 3)
	\vspace{3pt}
	\itemTitle{Other Math Courses}
	\item Complex Analysis
	\item Calculus
	\item Probability Theory
	\item Topology
\end{resume_list}
\columnbreak

\headingBf{Computer Science Courses:}{}
\begin{resume_list}
	\item Discrete Mathematics
	\item  Design and Analysis of Algorithms
	\item  Theory of Computation
	\item Complexity Theory
	\item  Expander Graphs and Application
	\item Parallel Algorithms and Complexity
	\item  Algorithmic Coding Theory (Two Parts)
	\item  Quantum Algorithmic Thinking
	\item  Quantum Information Theory
\end{resume_list}
\end{multicols}

\section{Workshop, Conferences Attended}
\begin{resume_list}
	\item \headingBf{\href{https://www.cmi.ac.in/activities/kohli-centre/quantum-semester-2024/}{Quantum Semester Online}}{Chennai Mathematical Institute}
	\headingIt{Chennai, India}{2024, Jan-May}{}
	\item \headingBf{\href{https://www.cmi.ac.in/~mkummini/sagedays122/index.html}{Sage Days 122}}{Chennai Mathematical Institute}
	\headingIt{Chennai, India}{2024, Jan-May}{}
	\item \headingBf{\href{https://sohamch08.github.io/assets/certificates_merged_SohamChatterjee.pdf}{p-adic Number Theory Lecture Series: Ram Murty}}{Chennai Mathematical Institute}
	\headingIt{Chennai, India}{2024, Jan-May}{}
\end{resume_list}


  %----------------------------%
  % Extracurricular Activities %
  %----------------------------%

  \section{Computer Skills}

  
  \begin{resume_list}
    \item \textbf{Programming Languages:} C (Basic), Python (Basic), Qiskit (Intermediate), Haskell (Basic), Java (Intermediate), Unix/Linux
    Shell Scripting, HTML, CSS
    \item \textbf{Technical Skills:} LaTeX (Advanced), Markdown, Git, Basic works in terminal, VIM, Obsidian
  \end{resume_list}

\end{document}