\documentclass[11pt]{article}




\usepackage[all]{xy}
\usepackage{graphics}
\usepackage{enumitem}
\usepackage{epsfig}
\usepackage{amsmath,amsthm}
\usepackage{amscd}
\usepackage{tikz-cd}
\usepackage{verbatim}
\usepackage{pdfpages}
%\usepackage{showkeys}
\usepackage{amsfonts,latexsym,amssymb, fullpage, mathtools}
\usepackage{parskip}
%\usepackage{MnSymbol}
\usepackage{hyperref}
\hypersetup{
	colorlinks=true,
	linkcolor=blue,
	filecolor=magenta,      
	urlcolor=blue!70!red,
	pdftitle={Assignment}, %%%%%%%%%%%%%%%%   WRITE ASSIGNMENT PDF NAME  %%%%%%%%%%%%%%%%%%%%
}
\usepackage{mdwlist}
%\usepackage{tgbonum}
\usepackage[utf8]{inputenc}
\usepackage{amsmath}





%%%%%%%%%%%%%%%%%%%%%%%%%%%%%%%%%%%%%%%%%%%%%%%%%%%%%%%%



\newcommand{\cA}{{\mathcal{A}}}   \newcommand{\cB}{{\mathcal{B}}}
\newcommand{\cC}{{\mathcal{C}}}   \newcommand{\cD}{{\mathcal{D}}}
\newcommand{\cE}{{\mathcal{E}}}   \newcommand{\cF}{{\mathcal{F}}}
\newcommand{\cG}{{\mathcal{G}}}   \newcommand{\cH}{{\mathcal{H}}}
\newcommand{\cI}{{\mathcal{I}}}   \newcommand{\cJ}{{\mathcal{J}}}
\newcommand{\cK}{{\mathcal{K}}}   \newcommand{\cL}{{\mathcal{L}}}
\newcommand{\cM}{{\mathcal{M}}}   \newcommand{\cN}{{\mathcal{N}}}
\newcommand{\cO}{{\mathcal{O}}}   \newcommand{\cP}{{\mathcal{P}}}
\newcommand{\cQ}{{\mathcal{Q}}}   \newcommand{\cR}{{\mathcal{R}}}
\newcommand{\cS}{{\mathcal{S}}}   \newcommand{\cT}{{\mathcal{T}}}
\newcommand{\cU}{{\mathcal{U}}}   \newcommand{\cV}{{\mathcal{V}}}
\newcommand{\cW}{{\mathcal{W}}}   \newcommand{\cX}{{\mathcal{X}}}
\newcommand{\cY}{{\mathcal{Y}}}   \newcommand{\cZ}{{\mathcal{Z}}}

\newcommand{\hcP}{\hat{\mathcal{P}}}
\newcommand{\hcQ}{\hat{\mathcal{Q}}}
\newcommand{\hcR}{\hat{\mathcal{R}}}
\newcommand{\hcL}{\hat{\mathcal{L}}}
\newcommand{\hcM}{\hat{\mathcal{M}}} \newcommand{\hphi}{\hat{\phi}}
\newcommand{\bbk}{\mathbb{k}}   \newcommand{\bfv}{\mathbf{v}}
\newcommand{\bfnu}{\mathbf{nu}}  \newcommand{\hXX}{\hat{\mathbb{X}}}
\newcommand{\bJ}{\mathbf{J}}

\newcommand{\hD}{{\hat{D}}}   \newcommand{\hE}{{\hat{E}}}
\newcommand{\hF}{{\hat{F}}}   \newcommand{\hH}{{\hat{H}}}
\newcommand{\hY}{{\hat{Y}}}   \newcommand{\hP}{{\hat{P}}}
\newcommand{\hT}{{\hat{T}}}   \newcommand{\hQ}{{\hat{Q}}}
\newcommand{\hq}{{\hat{q}}}
\newcommand{\hr}{{\hat{r}}}
\newcommand{\hu}{{\hat{u}}}
\newcommand{\hv}{{\hat{v}}}
\newcommand{\hf}{{\hat{f}}}
\newcommand{\hg}{{\hat{g}}}
\newcommand{\hw}{{\hat{w}}}
\newcommand{\hS}{{\hat{S}}}
\newcommand{\hV}{{\hat{V}}}
\newcommand{\hG}{{\hat{G}}}
\newcommand{\hmu}{{\hat{\mu}}}
\newcommand {\y}{\V{y}}
\newcommand {\V}[1]{\mbox{\boldmath$#1$}}
\newcommand{\iExp}{{\mathrm{iExp\,}}}

\newcommand{\htheta}{{\hat{\theta}}}



\newcommand{\htu}{{\hat{\tilde{u}}}}


\newcommand{\hTR}{{\widehat{TR}}}
\newcommand{\tsigma}{{\tilde{\sigma}}}
\newcommand{\tphi}{{\tilde{\phi}}}
\newcommand{\tpsi}{{\tilde{\psi}}}
\newcommand{\tzeta}{{\tilde{\zeta}}}
\newcommand{\tdelta}{{\tilde{\delta}}}
\newcommand{\tgamma}{{\tilde{\gamma}}}
\newcommand{\tGamma}{{\tilde{\Gamma}}}
\newcommand{\tlog}{{\widetilde{\log}}}


\newcommand{\txi}{{\tilde{\xi}}}
\newcommand{\tomega}{{\tilde{\omega}}}
\newcommand{\tH}{{\tilde{H}}}
\newcommand{\tI}{{\tilde{I}}}

\newcommand{\tX}{{\tilde{X}}}
\newcommand{\tV}{{\tilde{V}}}
\newcommand{\tz}{{\tilde{z}}}
\newcommand{\ty}{{\tilde{y}}}
\newcommand{\tx}{{\tilde{x}}}
\newcommand{\te}{{\tilde{e}}}
\newcommand{\tf}{{\tilde{f}}}
\newcommand{\tg}{{\tilde{g}}}
\newcommand{\tu}{{\tilde{u}}}
\newcommand{\tm}{{\tilde{m}}}
\newcommand{\tn}{{\tilde{n}}}
\newcommand{\tilt}{{\tilde{t}}}
\newcommand{\tT}{{\tilde{T}}}
\newcommand{\tL}{{\tilde{L}}}
\newcommand{\tQ}{{\tilde{Q}}}
\newcommand{\tB}{{\tilde{B}}}
\newcommand{\tC}{{\tilde{C}}}
\newcommand{\tD}{{\tilde{D}}}
\newcommand{\tU}{{\tilde{U}}}
\newcommand{\utL}{{\underline{\tilde{L}}}}
\newcommand{\tF}{{\tilde{F}}}\newcommand{\tilh}{{\tilde{h}}}
\newcommand{\tk}{{\tilde{k}}}
\newcommand{\tv}{{\tilde{v}}}
\newcommand{\tw}{{\tilde{w}}}
\newcommand{\bx}{\mathbf x}
\newcommand{\bz}{\mathbf z}
\newcommand{\bu}{\mathbf u}
\newcommand{\bv}{\mathbf v}
\newcommand{\bt}{\mathbf t}
\newcommand{\bi}{\mathbf i}
\newcommand{\bj}{\mathbf j}
\newcommand{\bL}{\mathbf L}
\newcommand{\bN}{\mathbf N}
\newcommand{\bM}{\mathbf M}
\newcommand{\bB}{\mathbf B}
\newcommand{\bA}{\mathbf A}
\newcommand{\tbz}{{\tilde{\mathbf z}}}
\newcommand{\hbx}{\hat{\mathbf x}}
\newcommand{\tcO}{{\tilde{\mathcal{O}}}}
\newcommand{\tcC}{{\tilde{\mathcal{C}}}}
\newcommand{\ocC}{{\overline{\mathcal{C}}}}
\newcommand{\tcR}{{\tilde{\mathcal{R}}}}
\newcommand{\tcA}{{\tilde{\mathcal{A}}}}








\newcommand{\uL}{\underline L}
\newcommand{\uM}{\underline M}
\newcommand{\uE}{\underline E}

\newcommand{\chA}{\check A}
\newcommand{\chE}{\check E}
\newcommand{\chL}{\check L}
\newcommand{\chV}{\check V}
\newcommand{\chv}{\check v}
\newcommand{\chw}{\check w}
\newcommand{\chW}{\check W}
\newcommand{\chM}{\check M}
\newcommand{\chQ}{\check Q}
\newcommand{\chsigma}{\check\sigma}



\newcommand{\uchL}{\underline{\check L}}






\newcommand{\BA}{\mathbb{A}}  \newcommand{\BB}{\mathbb{B}}
\newcommand{\CC}{\mathbb{C}}  \newcommand{\EE}{\mathbb{E}}
\newcommand{\FF}{\mathbb{F}}  \newcommand{\HH}{\mathbb{H}}
\newcommand{\JJ}{\mathbb{J}}  \newcommand{\LL}{\mathbb{L}}
\newcommand{\NN}{\mathbb{N}}  \newcommand{\PP}{\mathbb{P}}
\newcommand{\QQ}{\mathbb{Q}}  \newcommand{\RR}{\mathbb{R}}
\newcommand{\TT}{\mathbb{T}}  \newcommand{\VV}{\mathbb{V}}
\newcommand{\XX}{\mathbb{X}}  \newcommand{\WW}{\mathbb{W}}
\newcommand{\ZZ}{\mathbb{Z}}

\newcommand{\FM}{\mathfrak{M}}
\newcommand{\fm}{\mathfrak{m}}


\newcommand{\isom}{\cong}
\newcommand{\Ext}{\operatorname{Ext}}
\newcommand{\Grass}{\operatorname{Grass}}
\newcommand{\coker}{\operatorname{coker}}
\newcommand{\Hilb}{\operatorname{Hilb}}
\newcommand{\Hom}{\operatorname{Hom}}
\newcommand{\Quot}{\operatorname{Quot}}
\newcommand{\Pic}{\operatorname{Pic}}
\newcommand{\NS}{\operatorname{NS}}
\newcommand{\Sym}{\operatorname{Sym}}
\newcommand{\id}{\operatorname{I}}
\newcommand{\im}{\operatorname{im}}
\newcommand{\surj}{\twoheadrightarrow}
\newcommand{\inj}{\hookrightarrow}
\newcommand{\gr}{\operatorname{gr}}
\newcommand{\rk}{\operatorname{rk}}
\newcommand{\reg}{\operatorname{reg}}
\newcommand{\wt}{\widetilde}
\newcommand{\del}{{\partial}}
\newcommand{\delb}{{\overline\partial}}

\newcommand{\oX}{{\overline X}}
\newcommand{\oD}{{\overline D}}
\newcommand{\ox}{{\overline x}}
\newcommand{\ow}{{\overline w}}
\newcommand{\oz}{{\overline z}}
\newcommand{\oh}{{\overline{h}}}
\newcommand{\oalpha}{{\overline \alpha}}
\newcommand{\ndiv}{\hspace{-4pt}\not|\hspace{2pt}}




\newcommand{\Res}{\operatorname{Res}}
\newcommand{\ch}{\operatorname{ch}}
\newcommand{\tr}{\operatorname{tr}}
\newcommand{\pardeg}{\operatorname{par-deg}}
\newcommand{\ad}{{ad\,}}
\newcommand{\diag}{\operatorname{diag}}
\newcommand{\codim}{\operatorname{codim}}

\hyphenation{pa-ra-bo-lic}
\newcommand{\bbQ}{\mathbb{Q}}
\newcommand{\bbR}{\mathbb{R}}
\newcommand{\bbP}{\mathbb{P}}
\newcommand{\bbC}{\mathbb{C}}
\newcommand{\bbT}{\mathbb{T}}
\newcommand{\bbU}{\mathbb{U}}
\newcommand{\bbZ}{\mathbb{Z}}
\newcommand{\bbN}{\mathbb{N}}
\newcommand{\bbF}{\mathbb{F}}





\newtheorem{proposition}{Proposition}[section]
\newtheorem{theorem}[proposition]{Theorem}
\newtheorem{lemma}[proposition]{Lemma}
\newtheorem{conjecture}[proposition]{Conjecture}
\newtheorem{corollary}[proposition]{Corollary}



%\theoremstyle{definition}
%\newtheorem{definition}[proposition]{Definition}
%\newtheorem{remark}[proposition]{Remark}
%\newtheorem{notation}[proposition]{Notation}
%\newtheorem{example}[proposition]{Example}
%\newtheorem{ex}{Exercise}[section]

\usepackage[most,many,breakable]{tcolorbox}



\definecolor{mytheorembg}{HTML}{F2F2F9}
\definecolor{mytheoremfr}{HTML}{00007B}


\tcbuselibrary{theorems,skins,hooks}
\newtcbtheorem{problem}{Problem}
{%
	enhanced,
	breakable,
	colback = mytheorembg,
	frame hidden,
	boxrule = 0sp,
	borderline west = {2pt}{0pt}{mytheoremfr},
	sharp corners,
	detach title,
	before upper = \tcbtitle\par\smallskip,
	coltitle = mytheoremfr,
	fonttitle = \bfseries\sffamily,
	description font = \mdseries,
	separator sign none,
	segmentation style={solid, mytheoremfr},
}
{p}

\newcommand{\Qed}{\begin{flushright}\qed\end{flushright}}
\newcommand{\solve}[1]{\setlength{\parindent}{0cm}\textbf{\textit{Solution: }}\setlength{\parindent}{1cm}#1 \Qed}

\newcommand{\parinn}{\setlength{\parindent}{1cm}}
\newcommand{\parinf}{\setlength{\parindent}{0cm}}

\begin{document}

\title{\textbf{Fulton Chapter 3: Local Properties of Plane Curves}\\ Intersection Numbers}
\date{}
\author{}
\maketitle
\textsf{\noindent \large\textbf{Aritra Kundu} \hfill \textbf{Problem Set - 6}\\
	\textbf{Email}: \href{aritra@cmi.ac.in}{aritra@cmi.ac.in} \hfill \textbf{Topic}: Algebraic Geometry\\
	\noindent\rule{\textwidth}{2.8pt}}
\begin{problem}{3.18}{}
	If $P$ is a simple point on $F $, then $I(P,F\cap G)=\text{ord}^F_P(G)$. Give a proof of this using properties (1)-(7).
\end{problem}

\solve{\textbf{\underline{Case 1}: }$F$ is irreducible. As $P$ is a simple point of $F$ so $O_P(F)$ is a D.V.R.and the uniformizing parameter  of $M_P(F)$ is a line passes through $P$ but not the tangent at $P$. Let the line is $L$. Let $O_P^F(G)=n$. so,$$g=G+(F)=L^n u \quad[u\text{ is an unit in }O_P(F)]$$So, by property 7 $I(P,F\cap G)=I(P,F\cap g)$. By property 6,$$I(P,F\cap g)=I(P,F\cap L^n u)=n I(P,F\cap L)+I(P,F\cap u)$$By property 2, $$I(P,F\cap u)=0[\text{as }u\text{ is an unit in }O_P(F)\implies u(P)\neq 0 ]$$Now tangent of $L$ at $P$ is $L$. So, $F$ and $L$ do not share their tangents at $P$. By property 5, $I(P,F\cap L)=1$[as $P$ is a simple point of $F\implies m_P(F)=1$]. So, $$I(P,F\cap G)=I(P,F\cap g)=n=O_P^F(G)$$
	\parinf
	
\textbf{\underline{Case 2}: }$F$ is reducible. Let $F=\Pi_{i=1}^n F_i^{a_i}$. $P$ is a simple point of $F\implies m_P(F)=\sum_{i=1}^n a_i m_P(F_i)=1\implies$ for some $i,m_P(F_i)=1;a_i=1;m_P(F_j)=0\forall i\neq j\implies F_i$ is the only irreducible component passes through $P$. So,$$ O_P^F(G)=O_P^{F_i}(G)=I(P,F_i\cap G)\quad[\text{by case 1}]$$Now $$I(P,F\cap G)=\sum_{j=1}^n a_j I(P,F_j\cap G)=I(P,F_i\cap G)\quad[\text{as }\forall\ i\neq j F_j\text{ does not pass through }P\text{ and }a_i=1]$$
}

\begin{problem}{3.20}{}
	If $P$ is a simple point on $F$, then $I(P, F \cap(G+H)) \geq \min (I(P, F \cap G), I(P, F \cap H))$. Give an example to show that this may be false if $P$ is not simple on $F$.
\end{problem}

\solve{Let $I(P,F\cap G)=m;I(P,F\cap H)=n$\\
by the previous problem we know that $m=O_P^F(G);n=O_P^F(H)$\\
let $L$ be a line which passes through P but not the tangent of $F$ at $P$\\
so, $g=L^mu_1;h=L^nu_2$\\
WLOG $m\geq n$\\
so, $g+h=L^n(L^{m-n}u_1+u_2)$\\
so, $O_P^F(G+H)\geq n$\\
let $P=(0,0);F=x^2+y^2;H=x-y;G=x+y$\\
clearly $I(P,F\cap G)=I(P,F\cap H)=\infty$\\
but $G+H=2x$ and $ I(P,2x\cap x^2-y^2)=I(P,x\cap x^2-y^2)=2I(P,x\cap y)=2$\\
so the proposition will be failed if $P$ is not a simple point of $F$.\\\\}


\begin{problem}{3.21}{}
	Let $F$ be an affine plane curve. Let $L$ be a line that is not a component of $F$. Suppose $L=\{(a+t b, c+t d) \mid t \in k\}$. Define $G(T)=F(a+T b, c+T d)$. Factor $G(T)=$ $\epsilon \prod\left(T-\lambda_i\right)^{e_i}, \lambda_i$ distinct. Show that there is a natural one-to-one correspondence between the $\lambda_i$ and the points $P_i \in L \cap F$. Show that under this correspondence, $I\left(P_i, L \cap F\right)=e_i$. In particular, $\sum I(P, L \cap F) \leq \operatorname{deg}(F)$
\end{problem}


\solve{Let $P\in L\cap F$. Therefore, $P=(a+kb,c+kd)$ for some $k\in K$. $$F(a+kb,c+kd)=0\implies G(k)=0\implies k=\lambda_i$$ for some $i$. So, $$P=(a+\lambda_ib,c+\lambda_id)$$ and for all $\lambda_i,$ $(a+\lambda_ib,c+\lambda_id)\in L\cap F$. Ao, there is an one one correspondence between $\lambda_i$ and $P_i$ and $P_i=(a+\lambda_ib,c+\lambda_id)$. Now $L$ is not a component of $F\implies I(P_i,L\cap F)=m_{P_i}(L)m_{P_i}(F)=m_P(F)$ [as the tangent at $P_i$ of $L_i$ is $L_i$]. So, $$m_{P_i}(F(X,Y))=m_P(F(X+a+\lambda_ib,Y+c+\lambda_id))\quad [\text{where }P=(0,0)]$$Now either of $b,d$ is non zero [as $L$ is a line]. Let $b\neq 0$. Let $Y=dX/b$. $$F(X+a+\lambda_ib,dX/b+c+\lambda_id)=F(a+b(\lambda_i+X/b),c+d(\lambda_i+X/b))=G(\lambda_i+X/b)$$Now the lowest degree of $X$ in $G(X/b+\lambda_i)=m_{P_i}(F)$ [as in the least degree homogeneous term if we put $Y=dX/b$ then the degree will be same]. Now $$G(X/b+\lambda_i)=(X/b)^{e_i}\prod_{i\neq j}(X/b+\lambda_i-\lambda_j)^{e_j}$$So, the least degree is $e_i\implies m_{P_i}(F)=e_i$ [as $\lambda_i\neq \lambda_j\forall i\neq j$]. So $\sum\limits_{i}I(P_i,F\cap L_i)\leq \deg (F)$ [as $\deg (G)\leq \deg( F)]$}

\begin{problem}{3.23}{}
	A point $P$ on a curve $F$ is called a hypercusp if $m_P(F)>1, F$ has only one tangent line $L$ at $P$, and $I(P, L \cap F)=m_P(F)+1$. Generalize the results of the preceding problem to this case.
\end{problem}

\solve{Suppose $P=(0,0),L=Y$.$P$ is a hypercusp if and only if $\frac{\partial F}{{\partial^n X}}(P) \neq 0$ where $n=m_P(F)+1$. Let $F=YG+H(X)$ clearly $H(0)=0$[as $F(0,0)=0$]. Now $F=Y^{n-1}+F_1$ where $m_P(F_1)\geq n$[as $Y$ is the only tangent at $P$]. 
So,$H(x)=X^k(H_1(X)$ where $H_1(0)\neq 0$ and $k\geq n$. $\frac{\partial F}{{\partial^n X}}(P) \neq 0\iff$ the coefficient of $X^n$ is non zero.
Now $I(P,F\cap Y)=n\iff I(P,Y\cap H(X))=n\iff I(P,Y\cap X^k)=n\iff k=n\iff$the coefficient of $X^n$ is non zero. [as $H_1(0)\neq 0$]
\parinf

\textbf{\underline{2nd Part}:}

I will show that $F$ has only one irreducible component passing through $P$. Let assume $P=(0,0)$. Let $F=\Pi_{i=1}^n F_i^{a_i}$ where $F_i$'s are irreducible .
\parinn 


WLOG assume that  $F_1,F_2,\dots,F_k$ passes through $P$.
Let $L$ be the tangent of $F$ at $P$
So, $L$ be the only tangent of $F_i$ at $P$ [as if there is a tangent other than $L$ then it will be a tangent of $F$ as well because the least degree form of $F$ is the product of least degree form of $F_i$]. 
So, $I(P,F\cap L)=\sum_{i=1}^k a_i I(P,F_i\cap L)$\\
$ I(P,F_i\cap L)>m_P(F_i)m_P(L)\implies  I(P,F_i\cap L)\geq b_i+1$[where $b_i=m_P(F_i)$]\\
$ I(P,F\cap L)=\sum_{i=1}^k a_i b_i$\\
$ I(P,F\cap L)=m_P(F)+1=\sum_{i=1}^k a_i b_i+1\geq\sum_{i=1}^k a_i(b_i+1)$\\
so, $\sum_{i=1}^k a_i \leq 1$\\
but as $F$ passes through $P\implies$ at least one $a_i>0\implies \sum_{i=1}^k a_i\geq 1$\\
so, $\sum_{i=1}^k a_i=1\implies a_j=1;a_i=0\forall i\neq j$\\
so, $F$ has only one irreducible component passing through $P$\\\\}

\begin{problem}{3.24}{}
	The object of this problem is to find a property of the local ring ${O}_P(F)$ that determines whether or not $P$ is an ordinary multiple point on $F$.
	
	Let $F$ be an irreducible plane curve, $P=(0,0), m=m_P(F)>1$. Let $\mathfrak{m}=\mathfrak{m}_P(F)$. For $G \in k[X, Y]$, denote its residue in $\Gamma(F)$ by $g$; and for $g \in \mathfrak{m}$, denote its residue in $\mathfrak{m} / \mathfrak{m}^2$ by $\bar{g}$. (a) Show that the map from $\{$ forms of degree 1 in $k[X, Y]\}$ to $\mathfrak{m} / \mathfrak{m}^2$ taking $a X+$ $b Y$ to $\overline{a x+b y}$ is an isomorphism of vector spaces (see Problem 3.13). (b) Suppose $P$ is an ordinary multiple point, with tangents $L_1, \ldots, L_m$. Show that $I\left(P, F \cap L_i\right)>m$ and $\bar{l}_i \neq \lambda \overline{l_j}$ for all $i \neq j$, all $\lambda \in k$. (c) Suppose there are $G_1, \ldots, G_m \in k[X, Y]$ such that $I\left(P, F \cap G_i\right)>m$ and $\bar{g}_i \neq \lambda \bar{g}_j$ for all $i \neq j$, and all $\lambda \in k$. Show that $P$ is an ordinary multiple point on $F$. (Hint:: Write $G_i=L_i+$ higher terms. $\bar{l}_i=\bar{g}_i \neq 0$, and $L_i$ is the tangent to $G_i$, so $L_i$ is tangent to $F$ by Property (5) of intersection numbers. Thus $F$ has $m$ tangents at $P$.) (d) Show that $P$ is an ordinary multiple point on $F$ if and only if there are $g_1, \ldots, g_m \in \mathfrak{m}$ such that $\bar{g}_i \neq \lambda \bar{g}_j$ for all $i \neq j, \lambda \in k$, and $\operatorname{dim} {O}_P(F) /\left(g_i\right)>m$
\end{problem}


\solve{Clearly $M_P(F)=M=(x,y)$ where $x=X+(F),y=Y+(F)$ and both are non zero [as $m_P(F)>1]$\\
(a)\\
let $f:$forms of degree 1 in $k[X,Y]$=V $\to M/M^2$\\
s.t. $f(aX+bY)=\overline{ax+by}$\\
clearly $V$ is a vector space of dimension 2 with bases ${X,Y}$ and $F$ is a homomorphism of two k- vector space.\\
let $aX+bY\in ker(f)$\\
so, $ax+by\in M^2\implies aX+bY+G\in (F)$ where $m_p(G)>1$\\
$\implies m_p(F)=1$ which is not possible.\\
so, $a=b=0\implies f$ is injective .\\
by problem 3.13 $M/M^2$ is a vector space of dimension 2\\
so, $f$ is an isomorphism of vector space [by rank nullity theorem].\\
(b)\\
$I(P,F\cap L_i)>m_P(F)m_P(L_i)=m$[as $F$ and $L_i$ shares tangent at $P$]\\
now if $\overline{l_i}=\lambda \overline{l_j}\implies l_i-\lambda l_j\in M^2\implies L_i-L_j+G\in (F)$\\
but $M_P(F)>1\implies L_i=\lambda L_j$\\
which is not possible as $L_i$ are distinct tangents at $P$\\
(c)\\
$\overline{g_i}\neq 0\implies m_P(G_i)\leq 1$\\
if $m_P(G_i)=0\implies I(P,F\cap G)=0$ which is not possible.\\
so, $m_P(G_i)=1\implies G_i=L_i+H_i,m_P(H_i)>1$\\
so, $G_i$ has only i tangent $L_i$ at $P$ and $I(P,F\cap G_i)>m_P(F)m_P(G_i)\implies F and G_i$ share tangent at $P$\\
as $\overline{g_i}\neq \lambda \overline{g_j}\implies L_i, L_j$ are distinct [as $\overline{g_i}=\overline{l_i}$]\\
so, $F$ has m distinct tangents at $P$ and $m_P(F)=m\implies P$ is an ordinary point.\\
(d)\\
$dim_kO_P(F)/(g_i)=dim_kO_P(A^2)/(F,G)=I(P,F\cap G)$\\
so, by (c) if $g_1,g_2..g_m\in M$s.t. $\overline{g_i}\neq \overline{g_j}\forall i\neq j,\lambda \in k k$ and $dim_kO_P(F)/(g_i)=I(P,F\cap G_i)>m\implies P$ is an ordinary point.\\
if $P$ is an ordinary point take $G_i=L_i$where $L_i$'s are distinct tangent at $P$ $\implies\overline{l_i}\neq \lambda\overline{l_j}\forall i\neq j,\lambda\in k$ and $dim_kO_P(F)/(l_i)>m$\\
}












































\end{document}