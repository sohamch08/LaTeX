\documentclass[a4paper, 11pt]{article}
\usepackage{comment} % enables the use of multi-line comments (\ifx \fi) 
\usepackage{fullpage} % changes the margin
\usepackage[a4paper, total={7in, 10in}]{geometry}
\usepackage[fleqn]{amsmath,mathtools}
\usepackage{amssymb,amsthm}  % assumes amsmath package installed
\usepackage{float}
\usepackage{xcolor}
\usepackage{mdframed}
\usepackage[shortlabels]{enumitem}
\usepackage{indentfirst}
\usepackage{hyperref}
\hypersetup{
	colorlinks=true,
	linkcolor=blue,
	filecolor=magenta,      
	urlcolor=blue!70!red,
	pdftitle={Probability Assignment 3}, %%%%%%%%%%%%%%%%   WRITE ASSIGNMENT PDF NAME  %%%%%%%%%%%%%%%%%%%%
}
\usepackage[most,many,breakable]{tcolorbox}



\definecolor{mytheorembg}{HTML}{F2F2F9}
\definecolor{mytheoremfr}{HTML}{00007B}


\tcbuselibrary{theorems,skins,hooks}
\newtcbtheorem{problem}{Problem}
{%
	enhanced,
	breakable,
	colback = mytheorembg,
	frame hidden,
	boxrule = 0sp,
	borderline west = {2pt}{0pt}{mytheoremfr},
	sharp corners,
	detach title,
	before upper = \tcbtitle\par\smallskip,
	coltitle = mytheoremfr,
	fonttitle = \bfseries\sffamily,
	description font = \mdseries,
	separator sign none,
	segmentation style={solid, mytheoremfr},
}
{p}

% To give references for any problem use like this
% suppose the problem number is p3 then 2 options either 
% \hyperref[p:p3]{<text you want to use to hyperlink> \ref{p:p3}}
%                  or directly 
%                   \ref{p:p3}



%---------------------------------------
% BlackBoard Math Fonts :-
%---------------------------------------

%Captital Letters
\newcommand{\bbA}{\mathbb{A}}	\newcommand{\bbB}{\mathbb{B}}
\newcommand{\bbC}{\mathbb{C}}	\newcommand{\bbD}{\mathbb{D}}
\newcommand{\bbE}{\mathbb{E}}	\newcommand{\bbF}{\mathbb{F}}
\newcommand{\bbG}{\mathbb{G}}	\newcommand{\bbH}{\mathbb{H}}
\newcommand{\bbI}{\mathbb{I}}	\newcommand{\bbJ}{\mathbb{J}}
\newcommand{\bbK}{\mathbb{K}}	\newcommand{\bbL}{\mathbb{L}}
\newcommand{\bbM}{\mathbb{M}}	\newcommand{\bbN}{\mathbb{N}}
\newcommand{\bbO}{\mathbb{O}}	\newcommand{\bbP}{\mathbb{P}}
\newcommand{\bbQ}{\mathbb{Q}}	\newcommand{\bbR}{\mathbb{R}}
\newcommand{\bbS}{\mathbb{S}}	\newcommand{\bbT}{\mathbb{T}}
\newcommand{\bbU}{\mathbb{U}}	\newcommand{\bbV}{\mathbb{V}}
\newcommand{\bbW}{\mathbb{W}}	\newcommand{\bbX}{\mathbb{X}}
\newcommand{\bbY}{\mathbb{Y}}	\newcommand{\bbZ}{\mathbb{Z}}

%---------------------------------------
% MathCal Fonts :-
%---------------------------------------

%Captital Letters
\newcommand{\mcA}{\mathcal{A}}	\newcommand{\mcB}{\mathcal{B}}
\newcommand{\mcC}{\mathcal{C}}	\newcommand{\mcD}{\mathcal{D}}
\newcommand{\mcE}{\mathcal{E}}	\newcommand{\mcF}{\mathcal{F}}
\newcommand{\mcG}{\mathcal{G}}	\newcommand{\mcH}{\mathcal{H}}
\newcommand{\mcI}{\mathcal{I}}	\newcommand{\mcJ}{\mathcal{J}}
\newcommand{\mcK}{\mathcal{K}}	\newcommand{\mcL}{\mathcal{L}}
\newcommand{\mcM}{\mathcal{M}}	\newcommand{\mcN}{\mathcal{N}}
\newcommand{\mcO}{\mathcal{O}}	\newcommand{\mcP}{\mathcal{P}}
\newcommand{\mcQ}{\mathcal{Q}}	\newcommand{\mcR}{\mathcal{R}}
\newcommand{\mcS}{\mathcal{S}}	\newcommand{\mcT}{\mathcal{T}}
\newcommand{\mcU}{\mathcal{U}}	\newcommand{\mcV}{\mathcal{V}}
\newcommand{\mcW}{\mathcal{W}}	\newcommand{\mcX}{\mathcal{X}}
\newcommand{\mcY}{\mathcal{Y}}	\newcommand{\mcZ}{\mathcal{Z}}



%---------------------------------------
% Bold Math Fonts :-
%---------------------------------------

%Captital Letters
\newcommand{\bmA}{\boldsymbol{A}}	\newcommand{\bmB}{\boldsymbol{B}}
\newcommand{\bmC}{\boldsymbol{C}}	\newcommand{\bmD}{\boldsymbol{D}}
\newcommand{\bmE}{\boldsymbol{E}}	\newcommand{\bmF}{\boldsymbol{F}}
\newcommand{\bmG}{\boldsymbol{G}}	\newcommand{\bmH}{\boldsymbol{H}}
\newcommand{\bmI}{\boldsymbol{I}}	\newcommand{\bmJ}{\boldsymbol{J}}
\newcommand{\bmK}{\boldsymbol{K}}	\newcommand{\bmL}{\boldsymbol{L}}
\newcommand{\bmM}{\boldsymbol{M}}	\newcommand{\bmN}{\boldsymbol{N}}
\newcommand{\bmO}{\boldsymbol{O}}	\newcommand{\bmP}{\boldsymbol{P}}
\newcommand{\bmQ}{\boldsymbol{Q}}	\newcommand{\bmR}{\boldsymbol{R}}
\newcommand{\bmS}{\boldsymbol{S}}	\newcommand{\bmT}{\boldsymbol{T}}
\newcommand{\bmU}{\boldsymbol{U}}	\newcommand{\bmV}{\boldsymbol{V}}
\newcommand{\bmW}{\boldsymbol{W}}	\newcommand{\bmX}{\boldsymbol{X}}
\newcommand{\bmY}{\boldsymbol{Y}}	\newcommand{\bmZ}{\boldsymbol{Z}}
%Small Letters
\newcommand{\bma}{\boldsymbol{a}}	\newcommand{\bmb}{\boldsymbol{b}}
\newcommand{\bmc}{\boldsymbol{c}}	\newcommand{\bmd}{\boldsymbol{d}}
\newcommand{\bme}{\boldsymbol{e}}	\newcommand{\bmf}{\boldsymbol{f}}
\newcommand{\bmg}{\boldsymbol{g}}	\newcommand{\bmh}{\boldsymbol{h}}
\newcommand{\bmi}{\boldsymbol{i}}	\newcommand{\bmj}{\boldsymbol{j}}
\newcommand{\bmk}{\boldsymbol{k}}	\newcommand{\bml}{\boldsymbol{l}}
\newcommand{\bmm}{\boldsymbol{m}}	\newcommand{\bmn}{\boldsymbol{n}}
\newcommand{\bmo}{\boldsymbol{o}}	\newcommand{\bmp}{\boldsymbol{p}}
\newcommand{\bmq}{\boldsymbol{q}}	\newcommand{\bmr}{\boldsymbol{r}}
\newcommand{\bms}{\boldsymbol{s}}	\newcommand{\bmt}{\boldsymbol{t}}
\newcommand{\bmu}{\boldsymbol{u}}	\newcommand{\bmv}{\boldsymbol{v}}
\newcommand{\bmw}{\boldsymbol{w}}	\newcommand{\bmx}{\boldsymbol{x}}
\newcommand{\bmy}{\boldsymbol{y}}	\newcommand{\bmz}{\boldsymbol{z}}

%---------------------------------------
% Scr Math Fonts :-
%---------------------------------------

\newcommand{\sA}{{\mathscr{A}}}   \newcommand{\sB}{{\mathscr{B}}}
\newcommand{\sC}{{\mathscr{C}}}   \newcommand{\sD}{{\mathscr{D}}}
\newcommand{\sE}{{\mathscr{E}}}   \newcommand{\sF}{{\mathscr{F}}}
\newcommand{\sG}{{\mathscr{G}}}   \newcommand{\sH}{{\mathscr{H}}}
\newcommand{\sI}{{\mathscr{I}}}   \newcommand{\sJ}{{\mathscr{J}}}
\newcommand{\sK}{{\mathscr{K}}}   \newcommand{\sL}{{\mathscr{L}}}
\newcommand{\sM}{{\mathscr{M}}}   \newcommand{\sN}{{\mathscr{N}}}
\newcommand{\sO}{{\mathscr{O}}}   \newcommand{\sP}{{\mathscr{P}}}
\newcommand{\sQ}{{\mathscr{Q}}}   \newcommand{\sR}{{\mathscr{R}}}
\newcommand{\sS}{{\mathscr{S}}}   \newcommand{\sT}{{\mathscr{T}}}
\newcommand{\sU}{{\mathscr{U}}}   \newcommand{\sV}{{\mathscr{V}}}
\newcommand{\sW}{{\mathscr{W}}}   \newcommand{\sX}{{\mathscr{X}}}
\newcommand{\sY}{{\mathscr{Y}}}   \newcommand{\sZ}{{\mathscr{Z}}}


%---------------------------------------
% Math Fraktur Font
%---------------------------------------

%Captital Letters
\newcommand{\mfA}{\mathfrak{A}}	\newcommand{\mfB}{\mathfrak{B}}
\newcommand{\mfC}{\mathfrak{C}}	\newcommand{\mfD}{\mathfrak{D}}
\newcommand{\mfE}{\mathfrak{E}}	\newcommand{\mfF}{\mathfrak{F}}
\newcommand{\mfG}{\mathfrak{G}}	\newcommand{\mfH}{\mathfrak{H}}
\newcommand{\mfI}{\mathfrak{I}}	\newcommand{\mfJ}{\mathfrak{J}}
\newcommand{\mfK}{\mathfrak{K}}	\newcommand{\mfL}{\mathfrak{L}}
\newcommand{\mfM}{\mathfrak{M}}	\newcommand{\mfN}{\mathfrak{N}}
\newcommand{\mfO}{\mathfrak{O}}	\newcommand{\mfP}{\mathfrak{P}}
\newcommand{\mfQ}{\mathfrak{Q}}	\newcommand{\mfR}{\mathfrak{R}}
\newcommand{\mfS}{\mathfrak{S}}	\newcommand{\mfT}{\mathfrak{T}}
\newcommand{\mfU}{\mathfrak{U}}	\newcommand{\mfV}{\mathfrak{V}}
\newcommand{\mfW}{\mathfrak{W}}	\newcommand{\mfX}{\mathfrak{X}}
\newcommand{\mfY}{\mathfrak{Y}}	\newcommand{\mfZ}{\mathfrak{Z}}
%Small Letters
\newcommand{\mfa}{\mathfrak{a}}	\newcommand{\mfb}{\mathfrak{b}}
\newcommand{\mfc}{\mathfrak{c}}	\newcommand{\mfd}{\mathfrak{d}}
\newcommand{\mfe}{\mathfrak{e}}	\newcommand{\mff}{\mathfrak{f}}
\newcommand{\mfg}{\mathfrak{g}}	\newcommand{\mfh}{\mathfrak{h}}
\newcommand{\mfi}{\mathfrak{i}}	\newcommand{\mfj}{\mathfrak{j}}
\newcommand{\mfk}{\mathfrak{k}}	\newcommand{\mfl}{\mathfrak{l}}
\newcommand{\mfm}{\mathfrak{m}}	\newcommand{\mfn}{\mathfrak{n}}
\newcommand{\mfo}{\mathfrak{o}}	\newcommand{\mfp}{\mathfrak{p}}
\newcommand{\mfq}{\mathfrak{q}}	\newcommand{\mfr}{\mathfrak{r}}
\newcommand{\mfs}{\mathfrak{s}}	\newcommand{\mft}{\mathfrak{t}}
\newcommand{\mfu}{\mathfrak{u}}	\newcommand{\mfv}{\mathfrak{v}}
\newcommand{\mfw}{\mathfrak{w}}	\newcommand{\mfx}{\mathfrak{x}}
\newcommand{\mfy}{\mathfrak{y}}	\newcommand{\mfz}{\mathfrak{z}}

%---------------------------------------
% Bar
%---------------------------------------

%Captital Letters
\newcommand{\ovA}{\overline{A}}	\newcommand{\ovB}{\overline{B}}
\newcommand{\ovC}{\overline{C}}	\newcommand{\ovD}{\overline{D}}
\newcommand{\ovE}{\overline{E}}	\newcommand{\ovF}{\overline{F}}
\newcommand{\ovG}{\overline{G}}	\newcommand{\ovH}{\overline{H}}
\newcommand{\ovI}{\overline{I}}	\newcommand{\ovJ}{\overline{J}}
\newcommand{\ovK}{\overline{K}}	\newcommand{\ovL}{\overline{L}}
\newcommand{\ovM}{\overline{M}}	\newcommand{\ovN}{\overline{N}}
\newcommand{\ovO}{\overline{O}}	\newcommand{\ovP}{\overline{P}}
\newcommand{\ovQ}{\overline{Q}}	\newcommand{\ovR}{\overline{R}}
\newcommand{\ovS}{\overline{S}}	\newcommand{\ovT}{\overline{T}}
\newcommand{\ovU}{\overline{U}}	\newcommand{\ovV}{\overline{V}}
\newcommand{\ovW}{\overline{W}}	\newcommand{\ovX}{\overline{X}}
\newcommand{\ovY}{\overline{Y}}	\newcommand{\ovZ}{\overline{Z}}
%Small Letters
\newcommand{\ova}{\overline{a}}	\newcommand{\ovb}{\overline{b}}
\newcommand{\ovc}{\overline{c}}	\newcommand{\ovd}{\overline{d}}
\newcommand{\ove}{\overline{e}}	\newcommand{\ovf}{\overline{f}}
\newcommand{\ovg}{\overline{g}}	\newcommand{\ovh}{\overline{h}}
\newcommand{\ovi}{\overline{i}}	\newcommand{\ovj}{\overline{j}}
\newcommand{\ovk}{\overline{k}}	\newcommand{\ovl}{\overline{l}}
\newcommand{\ovm}{\overline{m}}	\newcommand{\ovn}{\overline{n}}
\newcommand{\ovo}{\overline{o}}	\newcommand{\ovp}{\overline{p}}
\newcommand{\ovq}{\overline{q}}	\newcommand{\ovr}{\overline{r}}
\newcommand{\ovs}{\overline{s}}	\newcommand{\ovt}{\overline{t}}
\newcommand{\ovu}{\overline{u}}	\newcommand{\ovv}{\overline{v}}
\newcommand{\ovw}{\overline{w}}	\newcommand{\ovx}{\overline{x}}
\newcommand{\ovy}{\overline{y}}	\newcommand{\ovz}{\overline{z}}

%---------------------------------------
% Tilde
%---------------------------------------

%Captital Letters
\newcommand{\tdA}{\tilde{A}}	\newcommand{\tdB}{\tilde{B}}
\newcommand{\tdC}{\tilde{C}}	\newcommand{\tdD}{\tilde{D}}
\newcommand{\tdE}{\tilde{E}}	\newcommand{\tdF}{\tilde{F}}
\newcommand{\tdG}{\tilde{G}}	\newcommand{\tdH}{\tilde{H}}
\newcommand{\tdI}{\tilde{I}}	\newcommand{\tdJ}{\tilde{J}}
\newcommand{\tdK}{\tilde{K}}	\newcommand{\tdL}{\tilde{L}}
\newcommand{\tdM}{\tilde{M}}	\newcommand{\tdN}{\tilde{N}}
\newcommand{\tdO}{\tilde{O}}	\newcommand{\tdP}{\tilde{P}}
\newcommand{\tdQ}{\tilde{Q}}	\newcommand{\tdR}{\tilde{R}}
\newcommand{\tdS}{\tilde{S}}	\newcommand{\tdT}{\tilde{T}}
\newcommand{\tdU}{\tilde{U}}	\newcommand{\tdV}{\tilde{V}}
\newcommand{\tdW}{\tilde{W}}	\newcommand{\tdX}{\tilde{X}}
\newcommand{\tdY}{\tilde{Y}}	\newcommand{\tdZ}{\tilde{Z}}
%Small Letters
\newcommand{\tda}{\tilde{a}}	\newcommand{\tdb}{\tilde{b}}
\newcommand{\tdc}{\tilde{c}}	\newcommand{\tdd}{\tilde{d}}
\newcommand{\tde}{\tilde{e}}	\newcommand{\tdf}{\tilde{f}}
\newcommand{\tdg}{\tilde{g}}	\newcommand{\tdh}{\tilde{h}}
\newcommand{\tdi}{\tilde{i}}	\newcommand{\tdj}{\tilde{j}}
\newcommand{\tdk}{\tilde{k}}	\newcommand{\tdl}{\tilde{l}}
\newcommand{\tdm}{\tilde{m}}	\newcommand{\tdn}{\tilde{n}}
\newcommand{\tdo}{\tilde{o}}	\newcommand{\tdp}{\tilde{p}}
\newcommand{\tdq}{\tilde{q}}	\newcommand{\tdr}{\tilde{r}}
\newcommand{\tds}{\tilde{s}}	\newcommand{\tdt}{\tilde{t}}
\newcommand{\tdu}{\tilde{u}}	\newcommand{\tdv}{\tilde{v}}
\newcommand{\tdw}{\tilde{w}}	\newcommand{\tdx}{\tilde{x}}
\newcommand{\tdy}{\tilde{y}}	\newcommand{\tdz}{\tilde{z}}

%---------------------------------------
% Vec
%---------------------------------------

%Captital Letters
\newcommand{\vcA}{\vec{A}}	\newcommand{\vcB}{\vec{B}}
\newcommand{\vcC}{\vec{C}}	\newcommand{\vcD}{\vec{D}}
\newcommand{\vcE}{\vec{E}}	\newcommand{\vcF}{\vec{F}}
\newcommand{\vcG}{\vec{G}}	\newcommand{\vcH}{\vec{H}}
\newcommand{\vcI}{\vec{I}}	\newcommand{\vcJ}{\vec{J}}
\newcommand{\vcK}{\vec{K}}	\newcommand{\vcL}{\vec{L}}
\newcommand{\vcM}{\vec{M}}	\newcommand{\vcN}{\vec{N}}
\newcommand{\vcO}{\vec{O}}	\newcommand{\vcP}{\vec{P}}
\newcommand{\vcQ}{\vec{Q}}	\newcommand{\vcR}{\vec{R}}
\newcommand{\vcS}{\vec{S}}	\newcommand{\vcT}{\vec{T}}
\newcommand{\vcU}{\vec{U}}	\newcommand{\vcV}{\vec{V}}
\newcommand{\vcW}{\vec{W}}	\newcommand{\vcX}{\vec{X}}
\newcommand{\vcY}{\vec{Y}}	\newcommand{\vcZ}{\vec{Z}}
%Small Letters
\newcommand{\vca}{\vec{a}}	\newcommand{\vcb}{\vec{b}}
\newcommand{\vcc}{\vec{c}}	\newcommand{\vcd}{\vec{d}}
\newcommand{\vce}{\vec{e}}	\newcommand{\vcf}{\vec{f}}
\newcommand{\vcg}{\vec{g}}	\newcommand{\vch}{\vec{h}}
\newcommand{\vci}{\vec{i}}	\newcommand{\vcj}{\vec{j}}
\newcommand{\vck}{\vec{k}}	\newcommand{\vcl}{\vec{l}}
\newcommand{\vcm}{\vec{m}}	\newcommand{\vcn}{\vec{n}}
\newcommand{\vco}{\vec{o}}	\newcommand{\vcp}{\vec{p}}
\newcommand{\vcq}{\vec{q}}	\newcommand{\vcr}{\vec{r}}
\newcommand{\vcs}{\vec{s}}	\newcommand{\vct}{\vec{t}}
\newcommand{\vcu}{\vec{u}}	\newcommand{\vcv}{\vec{v}}
\newcommand{\vcw}{\vec{w}}	\newcommand{\vcx}{\vec{x}}
\newcommand{\vcy}{\vec{y}}	\newcommand{\vcz}{\vec{z}}

%---------------------------------------
% Greek Letters:-
%---------------------------------------
\newcommand{\eps}{\epsilon}
\newcommand{\veps}{\varepsilon}
\newcommand{\lm}{\lambda}
\newcommand{\Lm}{\Lambda}
\newcommand{\gm}{\gamma}
\newcommand{\Gm}{\Gamma}
\newcommand{\vph}{\varphi}
\newcommand{\ph}{\phi}
\newcommand{\om}{\omega}
\newcommand{\Om}{\Omega}


%%%%%%%%%%%%%%%%%%%%%%%%%%%%%%%%%%%%%%%% MACROS %%%%%%%%%%%%%%%%%%%%%%%%%%%%%%%%%%%%%%%%

%%%%%%%%%%%%%%% Link With an Icon %%%%%%%%%%%%%%% 
\newcommand{\link}[1]{
    \href{#1}{\faIcon{link}}
}

%%%%%%%%%%%%%%% Name Template %%%%%%%%%%%%%%% 
\newcommand{\name}[2]{
    % Name
    \Huge % Font size
    \raggedright \textbf{#1} \par

    \vspace*{0.3cm}
    
    % Profession
    \Large % Font size
    \raggedright #2 \par
    \normalsize \normalfont
}

%%%%%%%%%%%%%%% Contact Details %%%%%%%%%%%%%%%
\newcommand{\info}[2]{
    \faIcon{#2} \hspace{0.2em} #1
}

%%%%%%%%%%%%%%% Email %%%%%%%%%%%%%%%
\newcommand{\email}[1]{
    \info{#1}{envelope}
}

%%%%%%%%%%%%%%% Phone Number %%%%%%%%%%%%%%%
\newcommand{\phone}[1]{
    \info{#1}{mobile-alt}
}

%%%%%%%%%%%%%%% Address %%%%%%%%%%%%%%%
\newcommand{\address}[1]{
    \info{#1}{map-marker-alt}
}

%%%%%%%%%%%%%%% GitHub %%%%%%%%%%%%%%%
\newcommand{\github}[2]{
    \info{\href{#1}{\underline{#2}}}{github}
}

%%%%%%%%%%%%%%% LinkedIn %%%%%%%%%%%%%%%
\newcommand{\linkedin}[2]{
    \info{\href{#1}{\underline{#2}}}{linkedin}
}

%%%%%%%%%%%%%%% ResearchGate %%%%%%%%%%%%%%%
\newcommand{\researchgate}[2]{
    \info{\href{#1}{\underline{#2}}}{researchgate}
}

%%%\newcommand*{\Researchgate}[1]{\sociallink{\researchgatesocialsymbol}{http://www.#1}{#1}}

%%%%%%%%%%%%%%% Website %%%%%%%%%%%%%%%
\newcommand{\website}[1]{
    \info{#1}{link}
}

%%%%%%%%%%%%%%% Draw Skill Bars %%%%%%%%%%%%%%% 
\newcommand{\drawskillbars}[1]{
    \begin{tikzpicture}
        % Draw 5 gray bars
        \foreach \i in {0, 1, 2, 3, 4}{
            \fill[lightgray] (\i * 0.7 + 0.2 *\i,0) rectangle (0.7 + \i * 0.7 + \i * 0.2,0.1);
        }
        
        % Draw number of black bars depending on the skill level
        \foreach \i in {#1}{
            \fill[blue!40] (\i * 0.7 + 0.2 *\i,0) rectangle (0.7 + \i * 0.7 + \i * 0.2,0.1);
            %\fill[title] (\i * 0.7 + 0.2 *\i,0) rectangle (0.7 + \i * 0.7 + \i * 0.2,0.1);
        }
    \end{tikzpicture} \par
}
    
%%%%%%%%%%%%%%% Skills %%%%%%%%%%%%%%%
\newcommand{\skill}[3]{
    % Name of the skill
    \large
    \noindent \hangafter=0
    \adjustbox{valign=t}{\begin{minipage}{0.72\textwidth}
        \large \noindent \hangafter=0
        % Name of the skill
        \textmd{#1} 
        \normalsize \par 
        \vspace{1em}
         % Description
        \noindent \small \color{subtitle} \parbox{1\linewidth}{\textsl{#3}} \par
        \normalsize \par
        \end{minipage}}
    \adjustbox{valign=t}{\begin{minipage}{0.2\textwidth}
        % Skill bars
        \large \hangafter=0
        %\noindent 
        \drawskillbars{#2}
        \end{minipage}}
    \normalsize \par 
    % Skill bars
    %%\drawskillbars{#2}
    %%\vspace{0.5em}
    
    \vspace{1.0em}
    \normalsize \color{black} \par
}

%%%%%%%%%%%%%%% Software %%%%%%%%%%%%%%%
\newcommand{\soft}[2]{
    \adjustbox{valign=t}{\begin{minipage}{0.40\textwidth}
        \large \noindent \hangafter=0
        % Name of the skill
        \textmd{#1} 
        \normalsize \par 
        \vspace{1em}
        \end{minipage}}
    \adjustbox{valign=t}{\begin{minipage}{0.5\textwidth}
        % Skill bars
        \large \noindent \hangafter=0
        \drawskillbars{#2}
        \end{minipage}}
    \normalsize \par 
    \vspace{1em}
}

%%%%%%%%%%%%%%% Personal details %%%%%%%%%%%%%%%
\newcommand{\details}[2]{
    % Name of the language
    \large
    \noindent \hangafter=0 \color{black}
    \adjustbox{valign=t}{\parbox{0.27\linewidth}{#1}}  \adjustbox{valign=t}{\parbox{0.55\linewidth}{#2}} \par
    \vspace{.3em}
    \normalsize \color{black} \par
 }

%%%%%%%%%%%%%%% Language %%%%%%%%%%%%%%%
\newcommand{\lan}[2]{
    % Name of the language
    \large
    \noindent \hangafter=0 \color{black}
    \parbox{0.3\linewidth}{\textmd{#1}}   \color{subtitle} \parbox{0.4\linewidth}{\textsl{#2}} \par
    %\large English \color{subtitle} \textit{Advanced} 
    %\normalsize \par 
    % Knowledge level
    %\noindent \small \color{subtitle} \parbox{1\linewidth}{\textsl{#2}} \par
    \vspace{1.0em}
    \normalsize \color{black} \par
 }

%%%%%%%%%%%%%%% Education %%%%%%%%%%%%%%%
\newcommand{\education}[4]{
    % Name of the studies
    \noindent \large \parbox{.65\linewidth}{\textbf{#1}}
    % Duration in a Box
    \hfill \small
    \tcbox[enhanced,nobeforeafter,box align=base,colback=title,colframe=title,size=fbox,arc=0mm, valign=bottom]{{\textbf{#2}}} \par
    \vspace{0.3em}
    % School Name 
    \normalsize
    \noindent \color{subtitle} \parbox{.9\linewidth}{\textsl{#3}} \par
    % Description
    \normalsize \color{black}
    \vspace*{0.3em}
    \small #4 
    \normalsize \par
    \vspace*{0.5em}
}

%%%%%%%%%%%%%%% Work Experience %%%%%%%%%%%%%%%
\newcommand{\work}[4]{
    % Name of the Job
    \noindent \large \parbox{.65\linewidth}{\textbf{#1}}
    % Duration in a Box 
    \hfill \small
    \tcbox[enhanced,box align=base,nobeforeafter,colback=title,colframe=title,size=fbox,arc=0mm]{\textbf{#2}} \par
    \vspace{0.3em}
    % Name of the Employer
    \noindent \large \color{subtitle} \parbox{.9\linewidth}{\textsl{#3}} \par
    % Description of the job
    \vspace*{0.3em} \color{black}
    \small #4 
    \normalsize \par
}

%%%%%%%%%%%%%%% Teaching %%%%%%%%%%%%%%%
\newcommand{\teaching}[3]{
    % What, Topic and Who/Where/when
    \noindent \adjustbox{valign=t}{\parbox{.99\linewidth}{\text{#1} \text{#2} \textbf{#3}}} 
    \vspace{0.5em}
    \vspace*{1em} 
}

%%%%%%%%%%%%%%% Publications %%%%%%%%%%%%%%%
\newcommand{\publ}[4]{
    % Authors, Title and journal
    \noindent \parbox{.99\linewidth}{\textsl{#3}. \textbf{#1} \textsl{#2} \link{#4}}
    \vspace{0.5em}
    \vspace*{1em} \color{black}
    }

%%%%%%%%%%%%%%% Talks %%%%%%%%%%%%%%%
\newcommand{\talk}[3]{
    % Authors, Title and journal
    \noindent \parbox{.99\linewidth}{\textsl{#3}. \textbf{#1} \textsl{#2}}
    \vspace{0.5em}
    \vspace*{1em} \color{black}
}

%%%%%%%%%%%%%%% Events %%%%%%%%%%%%%%%
\newcommand{\event}[3]{
    \noindent \parbox{.99\linewidth}{\textbf{#1} \textsl{#3} \textsl{#2}}
    \vspace{0.5em}
    \vspace*{1em} \color{black}
}

\setlength{\parindent}{0pt}

%%%%%%%%%%%%%%%%%%%%%%%%%%%%%%%%%%%%%%%%%%%%%%%%%%%%%%%%%%%%%%%%%%%%%%%%%%%%%%%%%%%%%%%%%%%%%%%%%%%%%%%%%%%%%%%%%%%%%%%%%%%%%%%%%%%%%%%%

\begin{document}

%%%%%%%%%%%%%%%%%%%%%%%%%%%%%%%%%%%%%%%%%%%%%%%%%%%%%%%%%%%%%%%%%%%%%%%%%%%%%%%%%%%%%%%%%%%%%%%%%%%%%%%%%%%%%%%%%%%%%%%%%%%%%%%%%%%%%%%%

\textsf{\noindent \large\textbf{Soham Chatterjee} \hfill \textbf{Assignment - 3}\\
	Email: \href{sohamc@cmi.ac.in}{sohamc@cmi.ac.in} \hfill Roll: BMC202175\\
	\normalsize Course:
	%
	Probability
	%
	Theory \hfill Date: April 17, 2022 \\
	\noindent\rule{7in}{2.8pt}}

%%%%%%%%%%%%%%%%%%%%%%%%%%%%%%%%%%%%%%%%%%%%%%%%%%%%%%%%%%%%%%%%%%%%%%%%%%%%%%%%%%%%%%%%%%%%%%%%%%%%%%%%%%%%%%%%%%%%%%%%%%%%%%%%%%%%%%%%
% Problem 1
%%%%%%%%%%%%%%%%%%%%%%%%%%%%%%%%%%%%%%%%%%%%%%%%%%%%%%%%%%%%%%%%%%%%%%%%%%%%%%%%%%%%%%%%%%%%%%%%%%%%%%%%%%%%%%%%%%%%%%%%%%%%%%%%%%%%%%%%

\begin{problem}{%problem statemnet
}{p1% problem reference text
}
%Problem
Suppose that $X, Y$ are independent Poisson random variables with parameters $\lambda, \mu$.
\begin{enumerate}[label=(\alph*)]
	\item Show that $X+Y$ is also a Poisson variable with parameter $\lambda+\mu$.
	\item Show that the conditional distribution of $Y$ given $X+Y=n$ is a binomial random variable.
\end{enumerate}
\end{problem}

\solve{
	%Solution
	\begin{enumerate}[label=(\alph*)]
		\item Given that $X,Y$ are independent Poisson Random Variables with parameters $\lm.\mu$ respectively. Hence $$P(X=n)=e^{-\lm}\frac{\lm^n}{n!}\text{ and }P(Y=m)=e^{-\mu}\frac{\mu^m}{m!}$$.Now\begin{align*}
			      P(X+Y=k) & = \sum_{i=0}^k P(X=i,Y=k-i)                                                 \\
			               & = \sum_{i=0}^k P(X=i)P(Y=k-i)                                               \\
			               & = \sum_{i=0}^k e^{-\lm}\frac{\lm^i}{i!} \, e^{-\mu}\frac{\mu^{k-i}}{(k-i)!} \\
			               & = e^{-(\lm+\mu)}\sum_{i=0}^k\frac{\lm^i\mu^{k-i}}{i!(k-i)!}                 \\
			               & = e^{-(\lm+\mu)}\frac1{k!}\sum_{i=0}^k\frac{k!}{i!(k-i)!}\lm^i\mu^{k-i}     \\
			               & = e^{-(\lm+\mu)} \frac{(\lm+\mu)^k}{k!}
		      \end{align*}
		      Hence $X+Y$ is also a Poisson Random Variable.
		\item $X, Y$ are independent Poisson random variables with parameters $\lambda, \mu$. Hence by the part (a) $X+Y$ is also a Poisson Random Variable. To find the conditional distribution of $Y$ given $X+Y=n$ let $Z=X+Y$ then the parameter of $Z$ is $\lm+\mu$. Let $f\st_{{} Y\mid Z}(k)=P(Y=k\mid X+Y=n)$. Hence \begin{align*}
			      f\st_{Y\mid Z}(k\mid n) & =P(Y=k\mid X+Y=n)                                                                                              \\
			                              & =\frac{P(Y=k,X+Y=n)}{P(X+Y=n)}                                                                                 \\
			                              & =\frac{P(X=n-k)P(Y=k)}{P(X+Y=n)}                                                                               \\
			                              & =\frac{e^{-\lm}\dfrac{\lm^{n-k}}{(n-k)!}\, e^{-\mu}\dfrac{\mu^{k}}{k!}}{e^{-(\lm+\mu)}\dfrac{(\lm+\mu)^n}{n!}} \\
			                              & =\frac{n!}{k!(n-k)!}\, \frac{\lm^{n-k}}{(\lm+\mu)^{n-k}}\, \frac{\mu^k}{(\lm+\mu)^k}                           \\
			                              & =\binom{n}{k} \lt(\frac{\mu}{\lm+\mu}\rt)^k\lt(1-\frac{\mu}{\lm+\mu}\rt)^{n-k}
		      \end{align*}
		      Hence $f\st_{Y\mid Z}$ is a Binomial Random Variable $Bin\lt(n,\frac{\mu}{\lm+\mu}\rt)$. Therefore the conditional distribution of $Y$ given $X+Y=n$ is a binomial random variable.
	\end{enumerate}
}


%%%%%%%%%%%%%%%%%%%%%%%%%%%%%%%%%%%%%%%%%%%%%%%%%%%%%%%%%%%%%%%%%%%%%%%%%
% Problem 2
%%%%%%%%%%%%%%%%%%%%%%%%%%%%%%%%%%%%%%%%%%%%%%%%%%%%%%%%%%%%%%%%%%%%%%%%%

\begin{problem}{%problem statemnet
}{p2% problem reference text
}
%Problem		
\begin{enumerate}[label=(\alph*)]
	\item Show that $Z=X+Y$ is a binomial random variable if $X, Y$ are independent binomial variables $Bin(n, p), Bin(m, p)$.
	\item The hypergeometric random with parameters $N, n, k$ is the random variable with mass function $f(x)={\binom{k}{x}\binom{N-k}{n-x}/\binom{N}{n}},\ x \in \mathbb{Z}_{\geq 0}$. Show that if $X, Y$ are independent binomial variables $B(n, p)$ and if $Z=X+Y$, then $X \mid Z$ is hypergeometric.
\end{enumerate}
\end{problem}

\solve{
	%Solution
	\begin{enumerate}[label=(\alph*)]
		\item $X,Y$ are independent Binomial Random Variables $Bin(n,p),Bin(m,p)$ respectively. Hence $$f\st+X(x)=P(X=x)=\binom{n}{x}p^x(1-p)^{n-x} \text{ and }f\st_Y(y)=P(Y=y)\binom{n}{y}p^y(1-p)^{n-y}$$Now let $Z=X+Y$. Then \begin{align*}
			      f\st_Z(z) & =P(Z=z)                                                                     \\
			                & =\sum_{k=0}^zP(X=k,Y=z-k)                                                   \\
			                & =\sum_{k=0}^zP(X=k)P(Y=z-k)                                                 \\
			                & =\sum_{k=0}^z\binom{n}{k}p^k(1-p)^{n-k}\binom{m}{z-k}p^{z-k}(1-p)^{m-(z-k)} \\
			                & =p^{z}(1-p)^{n+m-z}\sum_{k=0}^z\binom{n}{k}\binom{m}{z-k}                   \\
			                & =\binom{n+m}{z}p^z(1-p)^{n+m-z}
		      \end{align*}
		      Hence $Z$ is a Binomial Random Variable $Bin(n+m,p)$
		\item $X,Y$ are independent Binomial Random Variables $B(n,p)$ and $Z=X+Y$. Hence$$f\st_X(x)=P(X=x)=\binom{n}{x}p^x(1-p)^{n-x}\text{ and }f\st_Y(y)=P(Y=y)=\binom{n}{y}p^y(1-p)^{n-y}$$ $$f\st_Z(z)=P(Z=z)=\binom{2n}{z}p^z(1-p)^{2n-z}$$Now \begin{align*}
			      f\st_{X\mid Z}(x\mid z) & =P(X=x\mid Z=z)                \\
			                              & =\frac{P(X=x,Z=z)}{P(Z=z)}     \\
			                              & =\frac{P(X=x,Y=z-x)}{P(Z=z)}   \\
			                              & =\frac{P(X=x)P(Y=z-x)}{P(Z=z)}
		      \end{align*}
		      \begin{align*}
			      \qquad\qquad\quad & =\frac{\displaystyle{\binom{n}{x}p^x(1-p)^{n-x}\binom{n}{z-x}p^{z-x}(1-p)^{n-(z-x)}}}{\displaystyle{\binom{2n}{z}p^z(1-p)^{2n-z}}}                                \\
			                        & =\frac{\displaystyle{\binom{n}{x}\binom{n}{z-x}}}{\displaystyle{\binom{2n}{z}}}=\frac{\displaystyle{\binom{n}{x}\binom{2n-n}{z-x}}}{\displaystyle{\binom{2n}{z}}}
		      \end{align*}Hence $f\st_{X\mid Z}=f(x)$ where $f$ is the mass function of hypergeometric random variable with parameters $2n,z,n$. Hence $X\mid Z$ is hypergeometric
	\end{enumerate}
}


%%%%%%%%%%%%%%%%%%%%%%%%%%%%%%%%%%%%%%%%%%%%%%%%%%%%%%%%%%%%%%%%%%%%%%%%%
% Problem 3
%%%%%%%%%%%%%%%%%%%%%%%%%%%%%%%%%%%%%%%%%%%%%%%%%%%%%%%%%%%%%%%%%%%%%%%%%

\begin{problem}{%problem statemnet
}{p3% problem reference text
}
%Problem		
Let $X=\left(X_{1}, \ldots, X_{n}\right)$ be a vector of (discrete) random variables. Assume that $\operatorname{cov}\left(X_{i} X_{j}\right)$ exists for all $i, j$. The \textit{covariance matrix} $V(X)$ of $X$ is defined as the $n \times n$ matrix $\left(v_{i, j}\right)$ where $v_{i, j}=\operatorname{cov}\left(X_{i}, X_{j}\right)$. Show that $\operatorname{det}(V(X))=0$ if and only if there exist $a_{0}, a_{1}, \ldots, a_{n} \in \mathbb{R}$ such that $P\left(\sum\limits_{1 \leq j \leq n} a_{j} X_{j}=a_{0}\right)=1$.
\end{problem}

\solve{
	%Solution
	For this we will first prove two lemmas
	\vspace*{5mm}
	\parinf

	\textbf{\textit{Lemma 1: }}If $X,Y$  are two random variables and $a\in\bbR$ then $aCov(X,Y)=Cov(aX,Y)=Cov(X,aY).$

	\textbf{\textit{Proof: }}$Cov(X,Y)=E((X-E(X))(Y-E(Y))$. Hence\begin{multline*}
		aCov(X,Y)=aE((X-E(X))(Y-E(Y))=E(a(X-E(X))(Y-E(Y))\\
		=E((aX-E(aX))(Y-E(Y))=Cov(aX,Y)
	\end{multline*}Similarly\begin{multline*}
		aCov(X,Y)=aE((X-E(X))(Y-E(Y))=E((X-E(X))a(Y-E(Y))\\
		=E((X-E(X))(aY-E(aY))=Cov(X,aY)
	\end{multline*}Therefore $aCov(X,Y)=Cov(aX,Y)=Cov(X,aY)$

	\textbf{\textit{Lemma 2: }} Let $X$ is a discrete random variable. If $Var(X)=0$ then  $P(X=E(X))=1$ and if there exists a real number $a$ such that $P(x=a)=1$ then $a=E(X)$ and $Var(X)=0$

	\textbf{\textit{Proof: }}\parinn If $E(X^2)=0$ then $\sum\limits_{x}x^2P(X=x)=0$ for all $x\in S$ where $S$ is the set of all values the random variable takes. Since for all $x\in S$, $x^2P(X=x)\geq 0$ we can say $P(X=x)=0$ for all $x\in S$, $x\neq 0$. Hence $$P(X=0)=1-\sum_{x\neq 0}x^2P(x)=1-0=1$$. Hence $P(X=E(X)=0)=1$. Now let's assume $Var(X)=0$. Then $$Var(X)=0\implies E((X-E(X))^2)=0\implies P(X=E(X))=1$$

	Now let there exists a real number $a$ such that $P(X=a)=1$. Then $P(X=x)=0$ where $\forall \ x\neq a$. Hence $$E(X)=\sum_{x}xP(X=x)=aP(X=a)=a$$Hence $a=E(X)$ proved. Now $$Var(X)=E(X^2)-E(X)^2=E(X^2)-a^2=\sum_{x}x^2P(X=x)-a^2=a^2P(X=a)-a^2=a^2-a^2=0$$Hence $Var(X)=0$ proved\parinf


	\textbf{\textit{Lemma 3: }}Covariance Matrix is positive semidefinite

	\textbf{\textit{Proof: }}We have $v_{ij}=Cov(X_i,X_j)=E\lt[(X_i-E(X_i))(X_j-E(X_j))\rt]$. Let $Y=\begin{bmatrix}y_1 & y_2 & \cdots & y_n\end{bmatrix}^T$ be a real vector where each $y_i\in \bbR$. Now\begin{align*}Y^TV(X)Y & = \begin{bmatrix}y_1 & y_2 & \cdots & y_n\end{bmatrix}\begin{bmatrix}v_{11} & \cdots & v_{1n}\\	\vdots & \ddots & \vdots\\	v_{n1} & \cdots & v_{nn}\end{bmatrix}\begin{bmatrix}y_1\\ y_2 \\ \vdots \\y_n\end{bmatrix} \\
                       & =\sum_{1\leq i,j\leq n} y_iy_iCov(X_i,X_j)                                                                                                                                                                                                      \\
                       & =\sum_{1\leq i,j\leq n}Cov(y_iX_i,y_jX_j)\qquad [\text{Using }Lemma\ 1]                                                                                                                                                                         \\
                       & =\sum_{1\leq i,j\leq n}Cov(Z_i,Z_j)\qquad [\text{Let }Z_i=y_iX_i]                                                                                                                                                                               \\
                       & =\sum_{1\leq i,j\leq n}E\lt((Z_i-E(Z_i))(Z_j-E(Z_j))\rt)                                                                                                                                                                                        \\
                       & =E\lt(\sum_{1\leq i,j\leq n}(Z_i-E(Z_i))(Z_j-E(Z_j))\rt)                                                                                                                                                                                        \\
                       & = E\lt(\lt(\sum_{i=1}^n(Z_i-E(Z_i))\rt)^2\rt)                                                                                                                                                                                                   \\
                       & =E\lt[\lt[\lt(\sum_{i=1}^nZ_i\rt)-\lt(\sum_{i=1}^nE(Z_i)\rt)\rt]^2\rt]                                                                                                                                                                          \\
                       & =E\lt[\lt[\lt(\sum_{i=1}^nZ_i\rt)-E\lt(\sum_{i=1}^nZ_i\rt)\rt]^2\rt]
	\end{align*}Now for a random discrete random variable, $U$. Now $E(U^2)=\sum\limits_{u} u^2P(U=u)$ where $u^2\geq 0, P(U=u)\geq 0$ hence $E(U^2)\geq 0$ Therefore $Y^TV(X)Y\geq 0$ and hence covariance matrix is positive semidefinite.

	\textbf{\textit{Lemma 4: }}$\det(V(X))=0\iff\exs\ u\in\bbR^n\setminus\{0\}$ s.t $u^TV(X)u=0$

	\textbf{\textit{Proof: }}$\det(V(X))=0\iff $$\exs$ a vector $Y$ such that $V(X)Y=0$. Let $Y=\begin{bmatrix}
		y_1 & y_2 & \cdots & y_n
	\end{bmatrix}^T$ and $V(X)Y=0\iff Y^TV(X)Y=0$. Now $$Y^TV(X)Y=0\iff E\lt[\lt[\lt(\sum_{i=1}^nZ_i\rt)-E\lt(\sum_{i=1}^nZ_i\rt)\rt]^2\rt]  =0\iff Var\lt(\sum_{i=1}^nZ_i\rt)=0$$\parinn

	Now for the backward direction. There exists a vector $Y\in \bbR^n\setminus\{0\}$ such that $Y^TV(X)Y=0$. By \textit{Lemma 3} $V(X)$ is positive semidefinite. hence all of eigen values of $V(X)$ are nonnegative. If all eigen values are positive then $V(X)$ is positive definite. Then $\forall\ Y\in\bbR^n\setminus\{0\}, $ $Y^TV(X)Y>0$. This contradicts our assumption that there exist a vector $u\in\bbR^n\setminus\{0\}$ such that $u^TV(X)u=0$. Hence contradiction. Therefore at least one eigen value is 0. Hence $\det(V(X))=0$

	\parinn
	\vspace{5mm}

	Now coming back to the solution of for the covariance matrix $V(X)$,  $\det(V(X))=0\iff\exs\ Y\in\bbR^n\setminus\{0\}$ s.t $Y^TV(X)Y=0$ where $Y=\begin{bmatrix}
			y_1 & y_2 & \cdots & y_n		\end{bmatrix}^T$  Let $Z=\begin{bmatrix}
			y_1X_1 & y_2X_2 & \cdots & y_nX_n
		\end{bmatrix}^T$.                     Hence \begin{multline*}
		\det(V(X))=0\iff Var\lt(\sum\limits_{i=1}^nZ_i\rt)=0\xLeftrightarrow{\text{By \textit{Lemma 2}}} P\lt(\sum\limits_{i=1}^n a_iX_i = E\lt(\sum\limits_{i=1}^n a_iX_i\rt)\rt)=1\\\iff\exs\ a_0,a_1,a_2,\dots,a_n\in\bbR\text{ such that }P\lt(\sum_{i=1}^na_iX_i=a_0\rt)=1
	\end{multline*}
}
\end{document}
