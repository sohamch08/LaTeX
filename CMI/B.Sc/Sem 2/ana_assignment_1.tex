\documentclass[a4paper, 11pt]{article}
\usepackage{comment} % enables the use of multi-line comments (\ifx \fi) 
\usepackage{fullpage} % changes the margin
\usepackage[a4paper, total={7in, 10in}]{geometry}
\usepackage[fleqn]{amsmath,mathtools}
\usepackage{amssymb,amsthm}  % assumes amsmath package installed
\usepackage{float}
\usepackage{xcolor}
\usepackage{mdframed}
\usepackage[shortlabels]{enumitem}
\usepackage{indentfirst}
\usepackage{hyperref}
\hypersetup{
	colorlinks=true,
	linkcolor=blue,
	filecolor=magenta,      
	urlcolor=blue!70!red,
	pdftitle={Overleaf Example},
}
\usepackage[most,many,breakable]{tcolorbox}

%---------------------------------------
% BlackBoard Math Fonts :-
%---------------------------------------

%Captital Letters
\newcommand{\bbA}{\mathbb{A}}	\newcommand{\bbB}{\mathbb{B}}
\newcommand{\bbC}{\mathbb{C}}	\newcommand{\bbD}{\mathbb{D}}
\newcommand{\bbE}{\mathbb{E}}	\newcommand{\bbF}{\mathbb{F}}
\newcommand{\bbG}{\mathbb{G}}	\newcommand{\bbH}{\mathbb{H}}
\newcommand{\bbI}{\mathbb{I}}	\newcommand{\bbJ}{\mathbb{J}}
\newcommand{\bbK}{\mathbb{K}}	\newcommand{\bbL}{\mathbb{L}}
\newcommand{\bbM}{\mathbb{M}}	\newcommand{\bbN}{\mathbb{N}}
\newcommand{\bbO}{\mathbb{O}}	\newcommand{\bbP}{\mathbb{P}}
\newcommand{\bbQ}{\mathbb{Q}}	\newcommand{\bbR}{\mathbb{R}}
\newcommand{\bbS}{\mathbb{S}}	\newcommand{\bbT}{\mathbb{T}}
\newcommand{\bbU}{\mathbb{U}}	\newcommand{\bbV}{\mathbb{V}}
\newcommand{\bbW}{\mathbb{W}}	\newcommand{\bbX}{\mathbb{X}}
\newcommand{\bbY}{\mathbb{Y}}	\newcommand{\bbZ}{\mathbb{Z}}

%---------------------------------------
% MathCal Fonts :-
%---------------------------------------

%Captital Letters
\newcommand{\mcA}{\mathcal{A}}	\newcommand{\mcB}{\mathcal{B}}
\newcommand{\mcC}{\mathcal{C}}	\newcommand{\mcD}{\mathcal{D}}
\newcommand{\mcE}{\mathcal{E}}	\newcommand{\mcF}{\mathcal{F}}
\newcommand{\mcG}{\mathcal{G}}	\newcommand{\mcH}{\mathcal{H}}
\newcommand{\mcI}{\mathcal{I}}	\newcommand{\mcJ}{\mathcal{J}}
\newcommand{\mcK}{\mathcal{K}}	\newcommand{\mcL}{\mathcal{L}}
\newcommand{\mcM}{\mathcal{M}}	\newcommand{\mcN}{\mathcal{N}}
\newcommand{\mcO}{\mathcal{O}}	\newcommand{\mcP}{\mathcal{P}}
\newcommand{\mcQ}{\mathcal{Q}}	\newcommand{\mcR}{\mathcal{R}}
\newcommand{\mcS}{\mathcal{S}}	\newcommand{\mcT}{\mathcal{T}}
\newcommand{\mcU}{\mathcal{U}}	\newcommand{\mcV}{\mathcal{V}}
\newcommand{\mcW}{\mathcal{W}}	\newcommand{\mcX}{\mathcal{X}}
\newcommand{\mcY}{\mathcal{Y}}	\newcommand{\mcZ}{\mathcal{Z}}



%---------------------------------------
% Bold Math Fonts :-
%---------------------------------------

%Captital Letters
\newcommand{\bmA}{\boldsymbol{A}}	\newcommand{\bmB}{\boldsymbol{B}}
\newcommand{\bmC}{\boldsymbol{C}}	\newcommand{\bmD}{\boldsymbol{D}}
\newcommand{\bmE}{\boldsymbol{E}}	\newcommand{\bmF}{\boldsymbol{F}}
\newcommand{\bmG}{\boldsymbol{G}}	\newcommand{\bmH}{\boldsymbol{H}}
\newcommand{\bmI}{\boldsymbol{I}}	\newcommand{\bmJ}{\boldsymbol{J}}
\newcommand{\bmK}{\boldsymbol{K}}	\newcommand{\bmL}{\boldsymbol{L}}
\newcommand{\bmM}{\boldsymbol{M}}	\newcommand{\bmN}{\boldsymbol{N}}
\newcommand{\bmO}{\boldsymbol{O}}	\newcommand{\bmP}{\boldsymbol{P}}
\newcommand{\bmQ}{\boldsymbol{Q}}	\newcommand{\bmR}{\boldsymbol{R}}
\newcommand{\bmS}{\boldsymbol{S}}	\newcommand{\bmT}{\boldsymbol{T}}
\newcommand{\bmU}{\boldsymbol{U}}	\newcommand{\bmV}{\boldsymbol{V}}
\newcommand{\bmW}{\boldsymbol{W}}	\newcommand{\bmX}{\boldsymbol{X}}
\newcommand{\bmY}{\boldsymbol{Y}}	\newcommand{\bmZ}{\boldsymbol{Z}}
%Small Letters
\newcommand{\bma}{\boldsymbol{a}}	\newcommand{\bmb}{\boldsymbol{b}}
\newcommand{\bmc}{\boldsymbol{c}}	\newcommand{\bmd}{\boldsymbol{d}}
\newcommand{\bme}{\boldsymbol{e}}	\newcommand{\bmf}{\boldsymbol{f}}
\newcommand{\bmg}{\boldsymbol{g}}	\newcommand{\bmh}{\boldsymbol{h}}
\newcommand{\bmi}{\boldsymbol{i}}	\newcommand{\bmj}{\boldsymbol{j}}
\newcommand{\bmk}{\boldsymbol{k}}	\newcommand{\bml}{\boldsymbol{l}}
\newcommand{\bmm}{\boldsymbol{m}}	\newcommand{\bmn}{\boldsymbol{n}}
\newcommand{\bmo}{\boldsymbol{o}}	\newcommand{\bmp}{\boldsymbol{p}}
\newcommand{\bmq}{\boldsymbol{q}}	\newcommand{\bmr}{\boldsymbol{r}}
\newcommand{\bms}{\boldsymbol{s}}	\newcommand{\bmt}{\boldsymbol{t}}
\newcommand{\bmu}{\boldsymbol{u}}	\newcommand{\bmv}{\boldsymbol{v}}
\newcommand{\bmw}{\boldsymbol{w}}	\newcommand{\bmx}{\boldsymbol{x}}
\newcommand{\bmy}{\boldsymbol{y}}	\newcommand{\bmz}{\boldsymbol{z}}

%---------------------------------------
% Scr Math Fonts :-
%---------------------------------------

\newcommand{\sA}{{\mathscr{A}}}   \newcommand{\sB}{{\mathscr{B}}}
\newcommand{\sC}{{\mathscr{C}}}   \newcommand{\sD}{{\mathscr{D}}}
\newcommand{\sE}{{\mathscr{E}}}   \newcommand{\sF}{{\mathscr{F}}}
\newcommand{\sG}{{\mathscr{G}}}   \newcommand{\sH}{{\mathscr{H}}}
\newcommand{\sI}{{\mathscr{I}}}   \newcommand{\sJ}{{\mathscr{J}}}
\newcommand{\sK}{{\mathscr{K}}}   \newcommand{\sL}{{\mathscr{L}}}
\newcommand{\sM}{{\mathscr{M}}}   \newcommand{\sN}{{\mathscr{N}}}
\newcommand{\sO}{{\mathscr{O}}}   \newcommand{\sP}{{\mathscr{P}}}
\newcommand{\sQ}{{\mathscr{Q}}}   \newcommand{\sR}{{\mathscr{R}}}
\newcommand{\sS}{{\mathscr{S}}}   \newcommand{\sT}{{\mathscr{T}}}
\newcommand{\sU}{{\mathscr{U}}}   \newcommand{\sV}{{\mathscr{V}}}
\newcommand{\sW}{{\mathscr{W}}}   \newcommand{\sX}{{\mathscr{X}}}
\newcommand{\sY}{{\mathscr{Y}}}   \newcommand{\sZ}{{\mathscr{Z}}}


%---------------------------------------
% Math Fraktur Font
%---------------------------------------

%Captital Letters
\newcommand{\mfA}{\mathfrak{A}}	\newcommand{\mfB}{\mathfrak{B}}
\newcommand{\mfC}{\mathfrak{C}}	\newcommand{\mfD}{\mathfrak{D}}
\newcommand{\mfE}{\mathfrak{E}}	\newcommand{\mfF}{\mathfrak{F}}
\newcommand{\mfG}{\mathfrak{G}}	\newcommand{\mfH}{\mathfrak{H}}
\newcommand{\mfI}{\mathfrak{I}}	\newcommand{\mfJ}{\mathfrak{J}}
\newcommand{\mfK}{\mathfrak{K}}	\newcommand{\mfL}{\mathfrak{L}}
\newcommand{\mfM}{\mathfrak{M}}	\newcommand{\mfN}{\mathfrak{N}}
\newcommand{\mfO}{\mathfrak{O}}	\newcommand{\mfP}{\mathfrak{P}}
\newcommand{\mfQ}{\mathfrak{Q}}	\newcommand{\mfR}{\mathfrak{R}}
\newcommand{\mfS}{\mathfrak{S}}	\newcommand{\mfT}{\mathfrak{T}}
\newcommand{\mfU}{\mathfrak{U}}	\newcommand{\mfV}{\mathfrak{V}}
\newcommand{\mfW}{\mathfrak{W}}	\newcommand{\mfX}{\mathfrak{X}}
\newcommand{\mfY}{\mathfrak{Y}}	\newcommand{\mfZ}{\mathfrak{Z}}
%Small Letters
\newcommand{\mfa}{\mathfrak{a}}	\newcommand{\mfb}{\mathfrak{b}}
\newcommand{\mfc}{\mathfrak{c}}	\newcommand{\mfd}{\mathfrak{d}}
\newcommand{\mfe}{\mathfrak{e}}	\newcommand{\mff}{\mathfrak{f}}
\newcommand{\mfg}{\mathfrak{g}}	\newcommand{\mfh}{\mathfrak{h}}
\newcommand{\mfi}{\mathfrak{i}}	\newcommand{\mfj}{\mathfrak{j}}
\newcommand{\mfk}{\mathfrak{k}}	\newcommand{\mfl}{\mathfrak{l}}
\newcommand{\mfm}{\mathfrak{m}}	\newcommand{\mfn}{\mathfrak{n}}
\newcommand{\mfo}{\mathfrak{o}}	\newcommand{\mfp}{\mathfrak{p}}
\newcommand{\mfq}{\mathfrak{q}}	\newcommand{\mfr}{\mathfrak{r}}
\newcommand{\mfs}{\mathfrak{s}}	\newcommand{\mft}{\mathfrak{t}}
\newcommand{\mfu}{\mathfrak{u}}	\newcommand{\mfv}{\mathfrak{v}}
\newcommand{\mfw}{\mathfrak{w}}	\newcommand{\mfx}{\mathfrak{x}}
\newcommand{\mfy}{\mathfrak{y}}	\newcommand{\mfz}{\mathfrak{z}}

%---------------------------------------
% Bar
%---------------------------------------

%Captital Letters
\newcommand{\ovA}{\overline{A}}	\newcommand{\ovB}{\overline{B}}
\newcommand{\ovC}{\overline{C}}	\newcommand{\ovD}{\overline{D}}
\newcommand{\ovE}{\overline{E}}	\newcommand{\ovF}{\overline{F}}
\newcommand{\ovG}{\overline{G}}	\newcommand{\ovH}{\overline{H}}
\newcommand{\ovI}{\overline{I}}	\newcommand{\ovJ}{\overline{J}}
\newcommand{\ovK}{\overline{K}}	\newcommand{\ovL}{\overline{L}}
\newcommand{\ovM}{\overline{M}}	\newcommand{\ovN}{\overline{N}}
\newcommand{\ovO}{\overline{O}}	\newcommand{\ovP}{\overline{P}}
\newcommand{\ovQ}{\overline{Q}}	\newcommand{\ovR}{\overline{R}}
\newcommand{\ovS}{\overline{S}}	\newcommand{\ovT}{\overline{T}}
\newcommand{\ovU}{\overline{U}}	\newcommand{\ovV}{\overline{V}}
\newcommand{\ovW}{\overline{W}}	\newcommand{\ovX}{\overline{X}}
\newcommand{\ovY}{\overline{Y}}	\newcommand{\ovZ}{\overline{Z}}
%Small Letters
\newcommand{\ova}{\overline{a}}	\newcommand{\ovb}{\overline{b}}
\newcommand{\ovc}{\overline{c}}	\newcommand{\ovd}{\overline{d}}
\newcommand{\ove}{\overline{e}}	\newcommand{\ovf}{\overline{f}}
\newcommand{\ovg}{\overline{g}}	\newcommand{\ovh}{\overline{h}}
\newcommand{\ovi}{\overline{i}}	\newcommand{\ovj}{\overline{j}}
\newcommand{\ovk}{\overline{k}}	\newcommand{\ovl}{\overline{l}}
\newcommand{\ovm}{\overline{m}}	\newcommand{\ovn}{\overline{n}}
\newcommand{\ovo}{\overline{o}}	\newcommand{\ovp}{\overline{p}}
\newcommand{\ovq}{\overline{q}}	\newcommand{\ovr}{\overline{r}}
\newcommand{\ovs}{\overline{s}}	\newcommand{\ovt}{\overline{t}}
\newcommand{\ovu}{\overline{u}}	\newcommand{\ovv}{\overline{v}}
\newcommand{\ovw}{\overline{w}}	\newcommand{\ovx}{\overline{x}}
\newcommand{\ovy}{\overline{y}}	\newcommand{\ovz}{\overline{z}}

%---------------------------------------
% Tilde
%---------------------------------------

%Captital Letters
\newcommand{\tdA}{\tilde{A}}	\newcommand{\tdB}{\tilde{B}}
\newcommand{\tdC}{\tilde{C}}	\newcommand{\tdD}{\tilde{D}}
\newcommand{\tdE}{\tilde{E}}	\newcommand{\tdF}{\tilde{F}}
\newcommand{\tdG}{\tilde{G}}	\newcommand{\tdH}{\tilde{H}}
\newcommand{\tdI}{\tilde{I}}	\newcommand{\tdJ}{\tilde{J}}
\newcommand{\tdK}{\tilde{K}}	\newcommand{\tdL}{\tilde{L}}
\newcommand{\tdM}{\tilde{M}}	\newcommand{\tdN}{\tilde{N}}
\newcommand{\tdO}{\tilde{O}}	\newcommand{\tdP}{\tilde{P}}
\newcommand{\tdQ}{\tilde{Q}}	\newcommand{\tdR}{\tilde{R}}
\newcommand{\tdS}{\tilde{S}}	\newcommand{\tdT}{\tilde{T}}
\newcommand{\tdU}{\tilde{U}}	\newcommand{\tdV}{\tilde{V}}
\newcommand{\tdW}{\tilde{W}}	\newcommand{\tdX}{\tilde{X}}
\newcommand{\tdY}{\tilde{Y}}	\newcommand{\tdZ}{\tilde{Z}}
%Small Letters
\newcommand{\tda}{\tilde{a}}	\newcommand{\tdb}{\tilde{b}}
\newcommand{\tdc}{\tilde{c}}	\newcommand{\tdd}{\tilde{d}}
\newcommand{\tde}{\tilde{e}}	\newcommand{\tdf}{\tilde{f}}
\newcommand{\tdg}{\tilde{g}}	\newcommand{\tdh}{\tilde{h}}
\newcommand{\tdi}{\tilde{i}}	\newcommand{\tdj}{\tilde{j}}
\newcommand{\tdk}{\tilde{k}}	\newcommand{\tdl}{\tilde{l}}
\newcommand{\tdm}{\tilde{m}}	\newcommand{\tdn}{\tilde{n}}
\newcommand{\tdo}{\tilde{o}}	\newcommand{\tdp}{\tilde{p}}
\newcommand{\tdq}{\tilde{q}}	\newcommand{\tdr}{\tilde{r}}
\newcommand{\tds}{\tilde{s}}	\newcommand{\tdt}{\tilde{t}}
\newcommand{\tdu}{\tilde{u}}	\newcommand{\tdv}{\tilde{v}}
\newcommand{\tdw}{\tilde{w}}	\newcommand{\tdx}{\tilde{x}}
\newcommand{\tdy}{\tilde{y}}	\newcommand{\tdz}{\tilde{z}}

%---------------------------------------
% Vec
%---------------------------------------

%Captital Letters
\newcommand{\vcA}{\vec{A}}	\newcommand{\vcB}{\vec{B}}
\newcommand{\vcC}{\vec{C}}	\newcommand{\vcD}{\vec{D}}
\newcommand{\vcE}{\vec{E}}	\newcommand{\vcF}{\vec{F}}
\newcommand{\vcG}{\vec{G}}	\newcommand{\vcH}{\vec{H}}
\newcommand{\vcI}{\vec{I}}	\newcommand{\vcJ}{\vec{J}}
\newcommand{\vcK}{\vec{K}}	\newcommand{\vcL}{\vec{L}}
\newcommand{\vcM}{\vec{M}}	\newcommand{\vcN}{\vec{N}}
\newcommand{\vcO}{\vec{O}}	\newcommand{\vcP}{\vec{P}}
\newcommand{\vcQ}{\vec{Q}}	\newcommand{\vcR}{\vec{R}}
\newcommand{\vcS}{\vec{S}}	\newcommand{\vcT}{\vec{T}}
\newcommand{\vcU}{\vec{U}}	\newcommand{\vcV}{\vec{V}}
\newcommand{\vcW}{\vec{W}}	\newcommand{\vcX}{\vec{X}}
\newcommand{\vcY}{\vec{Y}}	\newcommand{\vcZ}{\vec{Z}}
%Small Letters
\newcommand{\vca}{\vec{a}}	\newcommand{\vcb}{\vec{b}}
\newcommand{\vcc}{\vec{c}}	\newcommand{\vcd}{\vec{d}}
\newcommand{\vce}{\vec{e}}	\newcommand{\vcf}{\vec{f}}
\newcommand{\vcg}{\vec{g}}	\newcommand{\vch}{\vec{h}}
\newcommand{\vci}{\vec{i}}	\newcommand{\vcj}{\vec{j}}
\newcommand{\vck}{\vec{k}}	\newcommand{\vcl}{\vec{l}}
\newcommand{\vcm}{\vec{m}}	\newcommand{\vcn}{\vec{n}}
\newcommand{\vco}{\vec{o}}	\newcommand{\vcp}{\vec{p}}
\newcommand{\vcq}{\vec{q}}	\newcommand{\vcr}{\vec{r}}
\newcommand{\vcs}{\vec{s}}	\newcommand{\vct}{\vec{t}}
\newcommand{\vcu}{\vec{u}}	\newcommand{\vcv}{\vec{v}}
\newcommand{\vcw}{\vec{w}}	\newcommand{\vcx}{\vec{x}}
\newcommand{\vcy}{\vec{y}}	\newcommand{\vcz}{\vec{z}}

%---------------------------------------
% Greek Letters:-
%---------------------------------------
\newcommand{\eps}{\epsilon}
\newcommand{\veps}{\varepsilon}
\newcommand{\lm}{\lambda}
\newcommand{\Lm}{\Lambda}
\newcommand{\gm}{\gamma}
\newcommand{\Gm}{\Gamma}
\newcommand{\vph}{\varphi}
\newcommand{\ph}{\phi}
\newcommand{\om}{\omega}
\newcommand{\Om}{\Omega}


\definecolor{mytheorembg}{HTML}{F2F2F9}
\definecolor{mytheoremfr}{HTML}{00007B}


\tcbuselibrary{theorems,skins,hooks}
\newtcbtheorem{problem}{Problem}
{%
	enhanced,
	breakable,
	colback = mytheorembg,
	frame hidden,
	boxrule = 0sp,
	borderline west = {2pt}{0pt}{mytheoremfr},
	sharp corners,
	detach title,
	before upper = \tcbtitle\par\smallskip,
	coltitle = mytheoremfr,
	fonttitle = \bfseries\sffamily,
	description font = \mdseries,
	separator sign none,
	segmentation style={solid, mytheoremfr},
}
{p}

\renewcommand{\thesubsection}{\thesection.\alph{subsection}}
\newcommand{\Z}{\mathbb{Z}}
\newcommand{\N}{\mathbb{N}}
\newcommand{\C}{\mathbb{C}}
\newcommand{\R}{\mathbb{R}}
\newcommand{\Qed}{\begin{flushright}\qed\end{flushright}}
\newcommand{\sol}[1]{\begin{solution}#1\end{solution}\Qed}
% The problem environment introduced.
%\newenvironment{problem}[2][Problem]
%{ \begin{mdframed}[backgroundcolor=gray!20] \textbf{#1 #2} \\}
%	{  \end{mdframed}}

% Define solution environment
\newenvironment{solution}
{\textbf{\textit{Solution:}}\setlength{\parindent}{1cm}}
{}

%\renewcommand{\qed}{\quad\qedsymbol}

\setlength{\parindent}{0pt}

%%%%%%%%%%%%%%%%%%%%%%%%%%%%%%%%%%%%%%%%%%%%%%%%%%%%%%%%%%%%%%%%%%%%%%%%%%%%%%%%%%%%%%%%%%%%%%%%%%%%%%%%%%%%%%%%%%%%%%%%%%%%%%%%%%%%%%%%

\begin{document}
	%Header-Make sure you update this information!!!!
	
	%%%%%%%%%%%%%%%%%%%%%%%%%%%%%%%%%%%%%%%%%%%%%%%%%%%%%%%%%%%%%%%%%%%%%%%%%%%%%%%%%%%%%%%%%%%%%%%%%%%%%%%%%%%%%%%%%%%%%%%%%%%%%%%%%%%%%%%%
	
	\textsf{\noindent \large\textbf{Soham Chatterjee} \hfill \textbf{Assignment - 1}\\
	Email: \href{sohamc@cmi.ac.in}{sohamc@cmi.ac.in} \hfill Roll: BMC202175\\
	\normalsize Course: Analysis 2 \hfill Date: \today \\
	\noindent\rule{7in}{2.8pt}}
	
	%%%%%%%%%%%%%%%%%%%%%%%%%%%%%%%%%%%%%%%%%%%%%%%%%%%%%%%%%%%%%%%%%%%%%%%%%%%%%%%%%%%%%%%%%%%%%%%%%%%%%%%%%%%%%%%%%%%%%%%%%%%%%%%%%%%%%%%%
	% Problem 1
	%%%%%%%%%%%%%%%%%%%%%%%%%%%%%%%%%%%%%%%%%%%%%%%%%%%%%%%%%%%%%%%%%%%%%%%%%%%%%%%%%%%%%%%%%%%%%%%%%%%%%%%%%%%%%%%%%%%%%%%%%%%%%%%%%%%%%%%%
	
	\begin{problem}{Rudin Chapt. 4 Problem 10}{p1}
            Complete the details of the following alternative proof of Theorem $4.19:$ If $f$ is not uniformly continuous, then for some $\varepsilon>0$ there are sequences $\left\{p_{n}\right\},\left\{q_{n}\right\}$ in $X$ such that $d_{X}\left(p_{n}, q_{n}\right) \rightarrow 0$ but $d_{Y}\left(f\left(p_{n}\right), f\left(q_{n}\right)\right)>\varepsilon$. Use Theorem $2.37$ to obtain a contradiction.
	\end{problem}
	
    \sol{ Theorem 2.37 says if $E$ is an infinite subset of a compact set $K$ then $E$ has a limit point. Now the set of all terms in the sequence $\{p_n\}$ is an infinite subset of the compact set. Hence the set has a limit point. Let the limit point is $p$. Hence there is a convergent subsequence $\{p_{n_k}\}$ of $\{p_n\}$ such that $p_{n_k}\to p$ as $n_k\to \infty$. Similarly $\{q_n\}$ has a convergent subsequence $\{q_{n_k}\}$ which converges to $q$. 

            Now given that $d(p_n,q_n)\to 0\implies d_X(p_{n_k},q_{n_k})\to 0$. Hence Both $\{p_{n_k}\}$ and $\{q_{n_k}\}$ converges to same limit. Hence $p=q$. Since $f$ is continuous $f(p_{n_k})\to f(p)$ and $f(q_{n_k})\to f(q)$. Since $p=q$ we have $d_Y\big(f(p_{n_k}),f(q_{n_k})\big)\to 0$. Hence $f(p)=f(q)$. Which contradicts the assumption that $d_Y(f(p_n),f(q_n))>\epsilon$

            Hence $f$ is uniformly continuous.
	}
	
	%\noindent\rule{7in}{2.8pt}
	
	%%%%%%%%%%%%%%%%%%%%%%%%%%%%%%%%%%%%%%%%%%%%%%%%%%%%%%%%%%%%%%%%%%%%%%%%%
	% Problem 2
	%%%%%%%%%%%%%%%%%%%%%%%%%%%%%%%%%%%%%%%%%%%%%%%%%%%%%%%%%%%%%%%%%%%%%%%%%%%%%%%%%%%%%%%%%%%%%%%%%%%%%%%%%%%%%%%%%%%%%%%%%%%%%%%%%%%%%%%%
	
    \begin{problem}{Rudin Chapt. 4 Problem 20}{p2}
            If $E$ is a nonempty subset of a metric space $X$, define the distance from $x \in X$ to $E$ by
            $$\rho_{E}(x)=\inf _{z \in E} d(x, z)$$\begin{enumerate}[label=(\alph*)]
                    \item Prove that $\rho{E}(x)=0$ if and only if $x \in \overline{E}$.
                            \item Prove that $\rho_{E}$ is a uniformly continuous function on $X$, by showing that$$\left|\rho_{E}(x)-\rho_{E}(y)\right| \leq d(x, y)$$for all $x \in X, y \in X$.

Hint: $\rho_{E}(x) \leq d(x, z) \leq d(x, y)+d(y, z)$, so that$$\rho_{E}(x) \leq d(x, y)+\rho_{E}(y) .$$
\end{enumerate}
			\end{problem}
	
            \sol{\begin{enumerate}[label=(\alph*)]
                    \item Let $x\in \overline{E}$. If its a limit point of $E$ then for every $\delta>0$ $\exists\ z\in E$ such that $z\in B_{\delta}(x)$. Hence $d(x,z)<\delta$. Hence $\inf\limits_{z\in E} d(x,z)=0\implies \rho_E(x)=0$. If $x$ is not a limit point then $x\in E$. Then $\inf\limits_{z\in E} d(x,z)=d(x,x)=0$
                    
                    Similarly if $\rho_E(x)=0$ hence for every $\delta>0$ $\exists\  z\in E$ such that $0\leq d(x,z)< \delta$. Hence for every $\delta>0$, $B_{\delta}(x)\cap E\neq \phi$. Hence $x $ is an limit point of $E$. Hence $x\in \overline{E}$. 
                    \item $\rho_E(x)\leq d(x,z)$ $\forall\ z\in E$. Since $d(x,z)\leq d(x,y)+d(y,z)$ we have $\rho_E(x)\leq d(x,y)+d(y,z)$ forall $z\in E$. Since forall $z\in E$ $\rho_E(x)\leq d(x,y)+d(y,z)$ we have $\rho_E(x)\leq d(y,z)+\rho_E(y)$. Therefore $$|\rho_E(x)-\rho_E(y)|\leq d(x,y)$$Hence $\rho_E$ is uniformly continuous function on $X$.
                    \end{enumerate}}
	
%	\noindent\rule{7in}{2.8pt}
	
	%%%%%%%%%%%%%%%%%%%%%%%%%%%%%%%%%%%%%%%%%%%%%%%%%%%%%%%%%%%%%%%%%%%%%%%%%
	% Problem 3
	%%%%%%%%%%%%%%%%%%%%%%%%%%%%%%%%%%%%%%%%%%%%%%%%%%%%%%%%%%%%%%%%%%%%%%%%%%%%%%%%%%%%%%%%%%%%%%%%%%%%%%%%%%%%%%%%%%%%%%%%%%%%%%%%%%%%%%%%
	
	\begin{problem}{Rudin Chapt. 4 Problem 21}{p3}
		Suppose $K$ and $F$ are disjoint sets in a metric space $X, K$ is compact, $F$ is closed. Prove that there exists $\delta>0$ such that $d(p, q)>\delta$ if $p \in K, q \in F$. Hint: $\rho_{F}$ is a continuous positive function on $K$.
		
		\hspace{1cm}Show that the conclusion may fail for two disjoint closed sets if neither is compact.
	\end{problem}
	
	\sol{
We have proved in the \hyperref[p:p2]{Problem \ref{p:p2}} that $\rho_F$ is a continuous function on the metric space $X$. Since $K$ is compact we have a continuous function on a compact set $K$. Therefore $\exists\ p\in K$ such that $\rho_F(p)=\inf\limits_{x\in K}\rho_F(x)$. Now if $\rho_F(p)=0$ then $p\in \overline{F}$. Since $F$ is a closed set $F=\overline{F}$ Hence $p\in F$ But $K$ and $F$ are disjoint. Therefore $\rho_F(p)\neq 0$. Let $2\delta=\rho_F(p)$. Hence $\forall\ p\in K,q\in F$, $d(p,q)>\delta$.

We can take the two sets as $K=\{\text{Set of all positive integers}\}$. and $F=\bigcup\limits_{n=1}^{\infty}\left[n+\frac{1}{n+1},n+1-\frac1{n+1}\right]$. Then there exists no positive number $\delta $ such that $d(p,q)>\delta$ $\forall\ p\in K,q\in F$
}
	
%	\noindent\rule{7in}{2.8pt}
	
	%%%%%%%%%%%%%%%%%%%%%%%%%%%%%%%%%%%%%%%%%%%%%%%%%%%%%%%%%%%%%%%%%%%%%%%%%
	% Problem 4
	%%%%%%%%%%%%%%%%%%%%%%%%%%%%%%%%%%%%%%%%%%%%%%%%%%%%%%%%%%%%%%%%%%%%%%%%%%%%%%%%%%%%%%%%%%%%%%%%%%%%%%%%%%%%%%%%%%%%%%%%%%%%%%%%%%%%%%%%
	
	\begin{problem}{Rudin Chapt. 4 Problem 22}{p4}
		Let $A$ and $B$ be disjoint nonempty closed sets in a metric space $X$, and define$$		f(p)=\frac{\rho_{A}(p)}{\rho_{A}(p)+\rho_{B}(p)} \quad(p \in X)$$Show that $f$ is a continuous function on $X$ whose range lies in $[0,1]$, that $f(p)=0$ precisely on $A$ and $f(p)=1$ precisely on $B$. This establishes a converse of Exercise 3: Every closed set $A \subset X$ is $Z(f)$ for some continuous real $f$ on $X$. Setting$$V=f^{-1}\left(\left[0, \frac{1}{2}\right)\right), \quad W=f^{-1}\left(\left(\frac{1}{2}, 1\right]\right),$$		show that $V$ and $W$ are open and disjoint, and that $A \subset V, B \subset W$. (Thus pairs of disjoint closed sets in a metric space can be covered by pairs of disjoint open sets. This property of metric spaces is called normality.)
	\end{problem}
	
	\sol{
	If $p\in A$. Then $\rho_A(p)=0$ Hence $$f(p)=\dfrac{\rho_{A}(p)}{\rho_{A}(p)+\rho_{B}(p)}=\frac{0}{0+\rho_{B}(p)}=0$$If $p\in B$ then $\rho_{B}(p)=0$. Hence $$f(p)=\dfrac{\rho_{A}(p)}{\rho_{A}(p)+\rho_{B}(p)}=\frac{\rho_{A}(p)}{\rho_{A}(p)+0}=1$$Hence $f(p)=0$ when $p\in A$ and $f(p)=1$ when $p\in B$. If $f(p)=0$ then $\inf\limits_{z\in A}d(p,z)=0$ or we can say $\forall\ \delta>0$ $\exists \ z\in A$ such that $d(p,z)<\delta$. Hence $p\in \overline{A}=A$ Hence $p\in A$ if $f(p)=0$. If $f(p)=1$ then $\rho_B(p)=0$. By similar argument in case of $\rho_A(p)=0$ we conclude $p\in B$. For any other $p\in X$ but $p\notin A\cup B$ $\rho_{A}(p)> 0$ and $\rho_B(p)> 0$. Now by \hyperref[p:p1]{Problem \ref{p:p1}}  $\rho_A(p)$ and $\rho_B(p)$ are continuous. Hence $\rho_A(p)+\rho_B(p)$ is also continuous. $\rho_A(p)+\rho_B(p)$ can not be 0 because if its 0 then both $\rho_A(p)=0$ and $\rho_B(p)=0$ which is not possible because that would mean $p\in A$ and $p\in B$ but $A,B$ are disjoint. Hence $\rho_A(p)+\rho_B(p)>0$ $\forall\ p\in X$. Therefore $\frac1{\rho_A(p)+\rho_B(p)}$ is also continuous. Hence $\rho_A(p)\times \frac1{\rho_A(p)+\rho_B(p)}=\frac{\rho_A(p)}{\rho_A(p)+\rho_B(p)}=f(p)$ is also continuous.
	
	$\left[\left.0,\frac12\right)\right.$ and $\left(\left.\frac12,1 \right]\right.$ are two open sets in $[0,1]$. As $f$ is continuous we have $f^{-1}(\text{open set})=$open set. Hence $V$ and $W$ are open sets and they are disjoint.
}
	
%	\noindent\rule{7in}{2.8pt}
	
	%%%%%%%%%%%%%%%%%%%%%%%%%%%%%%%%%%%%%%%%%%%%%%%%%%%%%%%%%%%%%%%%%%%%%%%%%
	% Problem 5
	%%%%%%%%%%%%%%%%%%%%%%%%%%%%%%%%%%%%%%%%%%%%%%%%%%%%%%%%%%%%%%%%%%%%%%%%%%%%%%%%%%%%%%%%%%%%%%%%%%%%%%%%%%%%%%%%%%%%%%%%%%%%%%%%%%%%%%%%
	
	\begin{problem}{}{p5}
		Show that a subset $K$ of $\bbR^n$ is compact if and only if every infinite subset of $K$ has a limit point in $K$. (Do this without appeal to the Heine-Borel theorem proved in class.)
	\end{problem}
	
	\sol{
		\subsubsection*{Proof of forward direction ($\boldsymbol{\Rightarrow}$)}
	Let $K$ is compact. Let $S$ is an infinite subset of $K$. If $S$ has no limit points in $K$ then $S$ is closed in $K$. Hence $S$ is a closed infinite subset of compact set $K$. Therefore $S$ is compact too. Since $S$ has no limit points $\forall\ x\in S$ $\exists \ r_x>0$ such that $B_{r_x}(x)$ contains at most one point of $S$. Now we create a collection of open balls with all such open balls $B_{r_x}(x)$ . Certainly this collection is an open cover of $S$. Since $S$ is compact there is a finite subcover of $S$. Let $B_{r_1}(x_1),B_{r_2}(x_2),\cdots,B_{r_m}(x_m)$. Each  $B_{r_i}(x_i)$ open ball cover at most one point of $S$. Hence $\bigcup\limits_{i=1}^mB_{r_i}(x_i)$ is a finite set. But $S$ is infinite set. Hence contradiction. $S$ contains a limit point.
	\subsubsection*{Proof of backward direction ($\boldsymbol{\Leftarrow}$)}
	Let every infinite subset of $K$ has a limit point in $K$. Let $x$ is a limit point of $K$. Hence $\forall\ \delta>0$ $\exists\ z\in K$ such that $z\in B_{\delta}(x)\cap K$. So if we choose a sequence of $\{\delta_n\}$ where $\delta_n=\frac1n$ then $\exists\ z_n$ such that $z_n\in B_{\delta_n}(x)\cap K$. If we take the set $S=\{z_n\mid z_n\in B_{\delta_n}\cap K\}$ then  $S\subseteq K$. If $S$ is a finite set then $x\in S$ and hence $x\in K$. If $S$ is an infinite set then $S$ has a limit point $y$ in $K$. Since the sequence $z_n\to x$ we have $y=x$. Hence $x\in K$. Therefore $K$ contains all its limit points. Hence $K$ is closed. 
	
	Suppose $K$ is not bounded then there exists $x_n\in K$ such that $\|x_n\|>n$. If we take the set $S=\{x_n\mid \|x_n\|>n\}$ then $S$ is an infinite subset of $K$. Hence $S$ should have a limit point in $K$. But the sequence $\{x_n\}$ is not convergent. Hence contradiction. $K$ is bounded. 
	
	Therefore $K$ is a closed and bounded set of $\bbR^n$. Hence $K$ can be bounded by $[-M,M]^n$ in $\bbR^n$. Now we know that $[-M,M]^n$ is a compact set. Hence $K$ is a closed subset of a compact set. Hence $K$ is also compact. Therefore if any infinite subset of $K$ has a limit point then $K$ is compact.
}
	
%	\noindent\rule{7in}{2.8pt}
	
	%%%%%%%%%%%%%%%%%%%%%%%%%%%%%%%%%%%%%%%%%%%%%%%%%%%%%%%%%%%%%%%%%%%%%%%%%
	
\end{document}
