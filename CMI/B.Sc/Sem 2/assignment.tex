\documentclass[a4paper, 11pt]{article}
\usepackage{comment} % enables the use of multi-line comments (\ifx \fi) 
\usepackage{fullpage} % changes the margin
\usepackage[a4paper, total={7in, 10in}]{geometry}
\usepackage[fleqn]{amsmath}
\usepackage{amssymb,amsthm}  % assumes amsmath package installed
\usepackage{float}
\usepackage{xcolor}
\usepackage{mdframed}
\usepackage[shortlabels]{enumitem}
\usepackage{indentfirst}
\usepackage{hyperref}
\hypersetup{
	colorlinks=true,
	linkcolor=blue,
	filecolor=magenta,      
	urlcolor=blue!70!red,
	pdftitle={Overleaf Example},
	pdfpagemode=FullScreen,
}
\renewcommand{\thesubsection}{\thesection.\alph{subsection}}
\newcommand{\Z}{\mathbb{Z}}
\newcommand{\N}{\mathbb{N}}
\newcommand{\C}{\mathbb{C}}
\newcommand{\R}{\mathbb{R}}

% The problem environment introduced.
\newenvironment{problem}[2][Problem]
    { \begin{mdframed}[backgroundcolor=gray!20] \textbf{#1 #2} \\}
    {  \end{mdframed}}

% Define solution environment
\newenvironment{solution}
    {\textit{Solution:}}
    {}

\renewcommand{\qed}{\quad\qedsymbol}

\setlength{\parindent}{0pt}

%%%%%%%%%%%%%%%%%%%%%%%%%%%%%%%%%%%%%%%%%%%%%%%%%%%%%%%%%%%%%%%%%%%%%%%%%%%%%%%%%%%%%%%%%%%%%%%%%%%%%%%%%%%%%%%%%%%%%%%%%%%%%%%%%%%%%%%%

\begin{document}
%Header-Make sure you update this information!!!!

%%%%%%%%%%%%%%%%%%%%%%%%%%%%%%%%%%%%%%%%%%%%%%%%%%%%%%%%%%%%%%%%%%%%%%%%%%%%%%%%%%%%%%%%%%%%%%%%%%%%%%%%%%%%%%%%%%%%%%%%%%%%%%%%%%%%%%%%

\noindent \large\textbf{Soham Chatterjee} \hfill \textbf{Assignment - 1}\\
Email: \href{sohamc@cmi.ac.in}{sohamc@cmi.ac.in} \hfill Roll: BMC202175\\
\normalsize Course: Analysis 2 \hfill Date: \today \\
\noindent\rule{7in}{2.8pt}

%%%%%%%%%%%%%%%%%%%%%%%%%%%%%%%%%%%%%%%%%%%%%%%%%%%%%%%%%%%%%%%%%%%%%%%%%%%%%%%%%%%%%%%%%%%%%%%%%%%%%%%%%%%%%%%%%%%%%%%%%%%%%%%%%%%%%%%%
% Problem 1
%%%%%%%%%%%%%%%%%%%%%%%%%%%%%%%%%%%%%%%%%%%%%%%%%%%%%%%%%%%%%%%%%%%%%%%%%%%%%%%%%%%%%%%%%%%%%%%%%%%%%%%%%%%%%%%%%%%%%%%%%%%%%%%%%%%%%%%%

\begin{problem}{1: Rudin Chapt. 4 Problem. 10}
    Complete
\end{problem}

\begin{solution}\\
    Let $G$ be a group of order 45.
    First note that $45 = 3^2 \times 5$.
    Let $n_p$ denote the number of Sylow $p$-subgroups of $G$.
    If $n_p = 1$ for any $p$, Then the corresponding Sylow $p$-subgroup must be normal.
    Reasoning is given in the solution for \textbf{Problem 4}.\\\\
    For $p = 5$, $n_5 \mid 3^2$ and $n_5 \equiv 1 \mod{5}$. 
    This is possible only when $n_5 = 1$.
    Then let the Sylow 5-subgroup be $S_5$.
    For $p = 3$, $n_3 \mid 5$ and $n_3 \equiv 1 \mod{3}$. 
    This is possible only when $n_3 = 1$.
    Then let the Sylow 3-subgroup be $S_3$.
    Note that $|S_5| = 5$ and $|S_3| = 9$.\\\\
    Clearly, $S_1 \cap S_2 = \{ e \}$.
    Then consider $S = \{ x \cdot y \mid x \in S_3, y \in S_5 \}$.
    If $x_1 y_1 = x_2 y_2$, then ${x_2}^{-1} x_1 = y_2 {y_1}^{-1}$.
    The LHS is in $S_3$ and the RHS is in $S_5$.
    So the element must be common to both i.e. it must be $e$.
    But this means $x_1 = x_2$ and $y_1 = y_2$.
    So every pair $x \cdot y$ is unique.
    So $|S| = |S_3| \times |S_5| = 9 \times 5 = 45 = |G|$.
    So $S = G$.
    So every $g \in G$ can be expressed as $g = x \cdot y$, where $x \in S_3$ and $y \in S_5$.\\\\
    Any group of prime order is cyclic and thus abelian.
    So $S_5$ is abelian.
    We show that any group of order $p^2$ is also abelian.
    Let $H$ be such a group.
    Then, by the class equation, $|Z(H)| \equiv 0 \mod{p}$.
    If $|Z(H)| = p^2$, we will be done.
    So let us assume $|Z(H)| = p$.
    Take any $a \notin Z(H)$.
    Then $\{ a \} \cup Z(H) \subseteq C(a)$.
    But this means $|C(a)| > p \Rightarrow |C(a)| = p^2$, a contradiction as $a \notin Z(H)$.
    So $|Z(H)|$ must be $p^2$.
    So any group of order $p^2$ must be abelian.
    So $S_3$ is also abelian.\\\\
    Now, for any $x \cdot y \in S$, consider $x y x^{-1} y^{-1}$.
    As $S_5$ is normal, $x y x^{-1} \in S_5$, $y^{-1} \in S_5$, implying $x y x^{-1} y^{-1} \in S_5$.
    Similarly, as $S_3$ is normal, $y x^{-1} y \in S_3$, $x \in S_3$, implying $x y x^{-1} y^{-1} \in S_3$.
    But this means $x y x^{-1} y^{-1} \in S_3 \cap S_5 \Rightarrow x y x^{-1} y^{-1} = e \Rightarrow x y = y x$.
    So $xy = yx$ for any $x \cdot y \in S$.
    Take any $g_1, g_2 \in G$.
    Then $g_1 = x_1 y_1$ and $g_2 = x_2 y_2$.
    Now, $g_1 g_2 = x_1 (y_1 x_2) y_2 = x_1 (x_2 y_1) y_2 = (x_1 x_2) (y_1 y_2) = (x_2 x_1) (y_2 y_1) = x_2 (x_1 y_2) y_1 = x_2 (y_2 x_1) y_1 = x_2 y_2 x_1 y_1 = g_2 g_1$.
    So $G$ must be abelian.
\end{solution}

\noindent\rule{7in}{2.8pt}

%%%%%%%%%%%%%%%%%%%%%%%%%%%%%%%%%%%%%%%%%%%%%%%%%%%%%%%%%%%%%%%%%%%%%%%%%
% Problem 2
%%%%%%%%%%%%%%%%%%%%%%%%%%%%%%%%%%%%%%%%%%%%%%%%%%%%%%%%%%%%%%%%%%%%%%%%%%%%%%%%%%%%%%%%%%%%%%%%%%%%%%%%%%%%%%%%%%%%%%%%%%%%%%%%%%%%%%%%

\begin{problem}{2}
    Let $p$ be a prime and $G$ be the matrix group $\text{GL}_2(\mathbb{Z}/\text{p}\mathbb{Z})$.
    Prove that there are $p+1$ Sylow $p$-subgroups in $G$.
    Then find the number of elements of order $p$.
    You may assume that $|G| = (p^2-1)(p^2 - p)$.
\end{problem}

\begin{solution}\\
    We are given that $|G| = (p^2 - 1) (p^2 - p) = p \cdot (p - 1)^2 \cdot (p + 1)$.
    Consider the matrix $A = \begin{bmatrix}
        1 & 1 \\
        0 & 1
    \end{bmatrix}$.
    Clearly, $\begin{bmatrix}
        1 & 1 \\
        0 & 1
    \end{bmatrix} \times \begin{bmatrix}
        1 & n \\
        0 & 1
    \end{bmatrix} = \begin{bmatrix}
        1 & n+1 \\
        0 & 1
    \end{bmatrix}$.
    So $A^n = \begin{bmatrix}
        1 & n \\
        0 & 1
    \end{bmatrix}$.
    Then $|A| = p$.
    Then $\langle A \rangle$ is a Sylow $p$-subgroup.
    Sylow $p$-subgroups are conjugate to each other, so $n_p = \frac{|G|}{|N(\langle A \rangle)|}$.
    Note that, for any $X = \begin{bmatrix}
        a & b \\
        c & d
    \end{bmatrix} \in \text{GL}_2(\mathbb{Z}/\text{p}\mathbb{Z})$, $X^{-1} = \frac{1}{ad - bc} \begin{bmatrix}
        d & -b \\
        -c & a
    \end{bmatrix}$.
    Then, $X A X^{-1} = \begin{bmatrix}
        a & b \\
        c & d
    \end{bmatrix} \times \begin{bmatrix}
        1 & 1 \\
        0 & 1
    \end{bmatrix} \times X^{-1} = \begin{bmatrix}
        a & a + b \\
        c & c + d
    \end{bmatrix} \times \frac{1}{ad - bc}\begin{bmatrix}
        d & -b \\
        -c & a
    \end{bmatrix} = \frac{1}{ad - bc} \begin{bmatrix}
        ad - ac - bc & a^2 \\
        -c^2 & ac + ad - bc
    \end{bmatrix}$.
    For $X A X^{-1} \in \langle A \rangle$, we need $c = 0$.
    Setting $c= 0$, we get $X A X^{-1} = \begin{bmatrix}
        1 & a d^{-1} \\
        0 & 1
    \end{bmatrix}$ which is clearly in $\langle A \rangle$.
    So the normaliser of $\langle A \rangle$ is the set of all invertible matrices of the form $\begin{bmatrix}
        a & b \\
        0 & d
    \end{bmatrix}$.
    The determinant of this matrix is then $ad \neq 0$.
    So we have $(p-1) \cdot (p-1)$ choices for $a$ and $d$.
    $b$ can take any value, so there are a total of $p$ choices for $b$.
    Then, $|N(\langle A \rangle)| = (p-1)^2 \cdot p$.
    Then $n_p = p+1$.\\\\
    In any subgroup of order $p$, there are $p-1$ elements of order $p$.
    Every Sylow $p$-subgroup will have trivial intersection.
    Reasoning is given in the solution for \textbf{Problem 3}.
    Any element of order $p$ will be in a Sylow $p$-subgroup.
    So the total number of elements of order $p$ will be $n_p \cdot (p-1) = (p+1) \cdot (p-1) = p^2 - 1$.
\end{solution}

\noindent\rule{7in}{2.8pt}

%%%%%%%%%%%%%%%%%%%%%%%%%%%%%%%%%%%%%%%%%%%%%%%%%%%%%%%%%%%%%%%%%%%%%%%%%
% Problem 3
%%%%%%%%%%%%%%%%%%%%%%%%%%%%%%%%%%%%%%%%%%%%%%%%%%%%%%%%%%%%%%%%%%%%%%%%%%%%%%%%%%%%%%%%%%%%%%%%%%%%%%%%%%%%%%%%%%%%%%%%%%%%%%%%%%%%%%%%

\begin{problem}{3}
    Prove that no group of order 105 could be simple.
\end{problem}

\begin{solution}\\
    Let $G$ be a group of order 105.
    First note that $105 = 3 \times 5 \times 7$.
    Let $n_p$ denote the number of Sylow $p$-subgroups of $G$.
    If $n_p = 1$ for any $p$, Then the corresponding Sylow $p$-subgroup must be normal.
    Reasoning is given in the solution for \textbf{Problem 4}.
    So we must have $n_p \neq 1$ for all primes $p$.\\\\
    Let $p = 5$.
    Then we have $n_5 \mid 3 \times 7$, $n_5 \equiv 1 \mod{5}$ and $n_5 \neq 1$.
    This is possible only when $n_5 = 21$.
    For any two Sylow 5-subgroups $S_1$, $S_2$, their intersection $S_1 \cap S_2 = \{ e \}$.
    This is because, for a group of prime order, any non-identity element generates the whole group.
    So nontrivial intersection would imply both the groups are same.
    So each of the 21 Sylow 5-subgroups will have 4 uniue elements.
    This accounts for a total of $4 \times 21 = 84$ elements of order 5.
    So the total number of elements of order 7 or 3 will be less than $105 - 84 - 1 = 15$ ($-1$ for identity).\\\\
    However, applying the same reasoning for $p = 7$, we get $n_7 = 15$.
    This means there are a total $15 * 6 = 90$ elements of order 7.
    90 is clearly greater than 15, so we get a contradiction.\\\\
    So $n_p$ for at least one prime will be 1.
    So $G$ will have at least one proper nontrivial normal subgroup.
    So $G$ cannot be simple.
\end{solution} 

\noindent\rule{7in}{2.8pt}

%%%%%%%%%%%%%%%%%%%%%%%%%%%%%%%%%%%%%%%%%%%%%%%%%%%%%%%%%%%%%%%%%%%%%%%%%
% Problem 4
%%%%%%%%%%%%%%%%%%%%%%%%%%%%%%%%%%%%%%%%%%%%%%%%%%%%%%%%%%%%%%%%%%%%%%%%%%%%%%%%%%%%%%%%%%%%%%%%%%%%%%%%%%%%%%%%%%%%%%%%%%%%%%%%%%%%%%%%

\begin{problem}{4}
    Let $p$ be an odd prime and let $G$ be a group of order $2p^n$ for some $n$.
    Show that $G$ is not simple.
\end{problem}

\begin{solution}\\
    Consider the Sylow $p$-subgroups.
    Let $n_p$ denote the number of such subgroups.
    Then $n_p \mid 2$, and $n_p \equiv 1 \mod{p}$.
    This implies $n_p = 1$.
    So there is only one Sylow $p$-subgroup, say $S$.
    For any $g \in G$, $g S g^{-1}$ is a subgroup.
    Now $|g S g^{-1}| = |S| = p^n$.
    But there is only one Sylow $p$-subgroup.
    So $g S g^{-1} = S$ for all $g \in G$.
    So $S$ is normal in $G$.
    So $G$ cannot be simple.
\end{solution}

\noindent\rule{7in}{2.8pt}

%%%%%%%%%%%%%%%%%%%%%%%%%%%%%%%%%%%%%%%%%%%%%%%%%%%%%%%%%%%%%%%%%%%%%%%%%
% Problem 5
%%%%%%%%%%%%%%%%%%%%%%%%%%%%%%%%%%%%%%%%%%%%%%%%%%%%%%%%%%%%%%%%%%%%%%%%%%%%%%%%%%%%%%%%%%%%%%%%%%%%%%%%%%%%%%%%%%%%%%%%%%%%%%%%%%%%%%%%

\begin{problem}{5}
    Let $G$ be a finite group such that for each $k$ dividing $|G|$, the number of elements satisfying $g^k = 1$ is at most $k$.
    Then prove that $G$ is cyclic.
\end{problem}

\begin{solution}\\
    First note that any subgroup of $G$ is normal in $G$.
    This is because, all $|H|$ elements of $|H| \leq G$ satisfy $g^{|H|} = 1$.
    Now, $g H g^{-1}$ for any $g \in G$ is a group of order $|H|$ .
    If it contains any element $h' \notin H$, then $h'$ also satisfies $h'^{|H|} = 1$.
    However, this means the number of elements satisfying $g^{|H|} = 1$ is $> |H|$, a contradiction.
    So all subgroups of $|G|$ are normal.\\\\
    Next note that the Sylow $p$-subgroups are cyclic.
    Assume any Sylow $p$-subgroup $S$ is not cyclic.
    Take any nontrivial element element $a \in S$.
    Then $S \backslash \langle a \rangle$ will be nonempty.
    Take any $b \in S \backslash \langle a \rangle$.
    Note that $| \langle a \rangle | = p^x$ and $| \langle b \rangle | = p^y$.
    By applying Cauchy's theorem, we will get unqiue elements of order $p$ in both these groups.
    However this is a contradiction, as we will get $2 \cdot (p - 1) + 1 = 2p - 1 > p$ elements satisfying $g^p = 1$.
    So the Sylow $p$-subgroups must be cyclic.\\\\
    Let $|G| = \prod_{i = 1}^{k} {p_i}^{r_i}$, where $p_i$ are unique primes, and $r_i > 0$.
    Let $a_i$ be the generator of the Sylow $p_i$-subgroup.
    We know that Sylow $p$-subgroups corresponding to different primes have trivial intersection.
    Also, normal subgroups which have trivial intersection commute with each other, by \textbf{Problem 1}.
    Then $(a_1 \cdot \ldots \cdot a_k)^j = a_1^j \cdot \ldots \cdot a_k^j$.
    Then the order of $a_1 \cdot \ldots \cdot a_k$ will be lcm$(p_1^{r_1}, p_2^{r_2}, \cdots, p_k^{r_k}) = \prod_{i = 1}^{k} {p_i}^{r_i} = |G|$.
    So $a_1 \cdot \ldots \cdot a_k$ has order $|G|$, and thus generates $G$.
    $G$ is thus cyclic.
\end{solution} 

\noindent\rule{7in}{2.8pt}

%%%%%%%%%%%%%%%%%%%%%%%%%%%%%%%%%%%%%%%%%%%%%%%%%%%%%%%%%%%%%%%%%%%%%%%%%

\end{document}