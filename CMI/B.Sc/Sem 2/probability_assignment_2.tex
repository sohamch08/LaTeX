\documentclass[a4paper, 11pt]{article}
\usepackage{comment} % enables the use of multi-line comments (\ifx \fi) 
\usepackage{fullpage} % changes the margin
\usepackage[a4paper, total={7in, 10in}]{geometry}
\usepackage[fleqn]{amsmath,mathtools}
\usepackage{amssymb,amsthm}  % assumes amsmath package installed
\usepackage{float}
\usepackage{xcolor}
\usepackage{mdframed}
\usepackage[shortlabels]{enumitem}
\usepackage{indentfirst}
\usepackage{hyperref}
\hypersetup{
	colorlinks=true,
	linkcolor=blue,
	filecolor=magenta,      
	urlcolor=blue!70!red,
	pdftitle={Assignment}, %%%%%%%%%%%%%%%%   WRITE ASSIGNMENT PDF NAME  %%%%%%%%%%%%%%%%%%%%
}
\usepackage[most,many,breakable]{tcolorbox}

%---------------------------------------
% BlackBoard Math Fonts :-
%---------------------------------------

%Captital Letters
\newcommand{\bbA}{\mathbb{A}}	\newcommand{\bbB}{\mathbb{B}}
\newcommand{\bbC}{\mathbb{C}}	\newcommand{\bbD}{\mathbb{D}}
\newcommand{\bbE}{\mathbb{E}}	\newcommand{\bbF}{\mathbb{F}}
\newcommand{\bbG}{\mathbb{G}}	\newcommand{\bbH}{\mathbb{H}}
\newcommand{\bbI}{\mathbb{I}}	\newcommand{\bbJ}{\mathbb{J}}
\newcommand{\bbK}{\mathbb{K}}	\newcommand{\bbL}{\mathbb{L}}
\newcommand{\bbM}{\mathbb{M}}	\newcommand{\bbN}{\mathbb{N}}
\newcommand{\bbO}{\mathbb{O}}	\newcommand{\bbP}{\mathbb{P}}
\newcommand{\bbQ}{\mathbb{Q}}	\newcommand{\bbR}{\mathbb{R}}
\newcommand{\bbS}{\mathbb{S}}	\newcommand{\bbT}{\mathbb{T}}
\newcommand{\bbU}{\mathbb{U}}	\newcommand{\bbV}{\mathbb{V}}
\newcommand{\bbW}{\mathbb{W}}	\newcommand{\bbX}{\mathbb{X}}
\newcommand{\bbY}{\mathbb{Y}}	\newcommand{\bbZ}{\mathbb{Z}}

%---------------------------------------
% MathCal Fonts :-
%---------------------------------------

%Captital Letters
\newcommand{\mcA}{\mathcal{A}}	\newcommand{\mcB}{\mathcal{B}}
\newcommand{\mcC}{\mathcal{C}}	\newcommand{\mcD}{\mathcal{D}}
\newcommand{\mcE}{\mathcal{E}}	\newcommand{\mcF}{\mathcal{F}}
\newcommand{\mcG}{\mathcal{G}}	\newcommand{\mcH}{\mathcal{H}}
\newcommand{\mcI}{\mathcal{I}}	\newcommand{\mcJ}{\mathcal{J}}
\newcommand{\mcK}{\mathcal{K}}	\newcommand{\mcL}{\mathcal{L}}
\newcommand{\mcM}{\mathcal{M}}	\newcommand{\mcN}{\mathcal{N}}
\newcommand{\mcO}{\mathcal{O}}	\newcommand{\mcP}{\mathcal{P}}
\newcommand{\mcQ}{\mathcal{Q}}	\newcommand{\mcR}{\mathcal{R}}
\newcommand{\mcS}{\mathcal{S}}	\newcommand{\mcT}{\mathcal{T}}
\newcommand{\mcU}{\mathcal{U}}	\newcommand{\mcV}{\mathcal{V}}
\newcommand{\mcW}{\mathcal{W}}	\newcommand{\mcX}{\mathcal{X}}
\newcommand{\mcY}{\mathcal{Y}}	\newcommand{\mcZ}{\mathcal{Z}}



%---------------------------------------
% Bold Math Fonts :-
%---------------------------------------

%Captital Letters
\newcommand{\bmA}{\boldsymbol{A}}	\newcommand{\bmB}{\boldsymbol{B}}
\newcommand{\bmC}{\boldsymbol{C}}	\newcommand{\bmD}{\boldsymbol{D}}
\newcommand{\bmE}{\boldsymbol{E}}	\newcommand{\bmF}{\boldsymbol{F}}
\newcommand{\bmG}{\boldsymbol{G}}	\newcommand{\bmH}{\boldsymbol{H}}
\newcommand{\bmI}{\boldsymbol{I}}	\newcommand{\bmJ}{\boldsymbol{J}}
\newcommand{\bmK}{\boldsymbol{K}}	\newcommand{\bmL}{\boldsymbol{L}}
\newcommand{\bmM}{\boldsymbol{M}}	\newcommand{\bmN}{\boldsymbol{N}}
\newcommand{\bmO}{\boldsymbol{O}}	\newcommand{\bmP}{\boldsymbol{P}}
\newcommand{\bmQ}{\boldsymbol{Q}}	\newcommand{\bmR}{\boldsymbol{R}}
\newcommand{\bmS}{\boldsymbol{S}}	\newcommand{\bmT}{\boldsymbol{T}}
\newcommand{\bmU}{\boldsymbol{U}}	\newcommand{\bmV}{\boldsymbol{V}}
\newcommand{\bmW}{\boldsymbol{W}}	\newcommand{\bmX}{\boldsymbol{X}}
\newcommand{\bmY}{\boldsymbol{Y}}	\newcommand{\bmZ}{\boldsymbol{Z}}
%Small Letters
\newcommand{\bma}{\boldsymbol{a}}	\newcommand{\bmb}{\boldsymbol{b}}
\newcommand{\bmc}{\boldsymbol{c}}	\newcommand{\bmd}{\boldsymbol{d}}
\newcommand{\bme}{\boldsymbol{e}}	\newcommand{\bmf}{\boldsymbol{f}}
\newcommand{\bmg}{\boldsymbol{g}}	\newcommand{\bmh}{\boldsymbol{h}}
\newcommand{\bmi}{\boldsymbol{i}}	\newcommand{\bmj}{\boldsymbol{j}}
\newcommand{\bmk}{\boldsymbol{k}}	\newcommand{\bml}{\boldsymbol{l}}
\newcommand{\bmm}{\boldsymbol{m}}	\newcommand{\bmn}{\boldsymbol{n}}
\newcommand{\bmo}{\boldsymbol{o}}	\newcommand{\bmp}{\boldsymbol{p}}
\newcommand{\bmq}{\boldsymbol{q}}	\newcommand{\bmr}{\boldsymbol{r}}
\newcommand{\bms}{\boldsymbol{s}}	\newcommand{\bmt}{\boldsymbol{t}}
\newcommand{\bmu}{\boldsymbol{u}}	\newcommand{\bmv}{\boldsymbol{v}}
\newcommand{\bmw}{\boldsymbol{w}}	\newcommand{\bmx}{\boldsymbol{x}}
\newcommand{\bmy}{\boldsymbol{y}}	\newcommand{\bmz}{\boldsymbol{z}}

%---------------------------------------
% Scr Math Fonts :-
%---------------------------------------

\newcommand{\sA}{{\mathscr{A}}}   \newcommand{\sB}{{\mathscr{B}}}
\newcommand{\sC}{{\mathscr{C}}}   \newcommand{\sD}{{\mathscr{D}}}
\newcommand{\sE}{{\mathscr{E}}}   \newcommand{\sF}{{\mathscr{F}}}
\newcommand{\sG}{{\mathscr{G}}}   \newcommand{\sH}{{\mathscr{H}}}
\newcommand{\sI}{{\mathscr{I}}}   \newcommand{\sJ}{{\mathscr{J}}}
\newcommand{\sK}{{\mathscr{K}}}   \newcommand{\sL}{{\mathscr{L}}}
\newcommand{\sM}{{\mathscr{M}}}   \newcommand{\sN}{{\mathscr{N}}}
\newcommand{\sO}{{\mathscr{O}}}   \newcommand{\sP}{{\mathscr{P}}}
\newcommand{\sQ}{{\mathscr{Q}}}   \newcommand{\sR}{{\mathscr{R}}}
\newcommand{\sS}{{\mathscr{S}}}   \newcommand{\sT}{{\mathscr{T}}}
\newcommand{\sU}{{\mathscr{U}}}   \newcommand{\sV}{{\mathscr{V}}}
\newcommand{\sW}{{\mathscr{W}}}   \newcommand{\sX}{{\mathscr{X}}}
\newcommand{\sY}{{\mathscr{Y}}}   \newcommand{\sZ}{{\mathscr{Z}}}


%---------------------------------------
% Math Fraktur Font
%---------------------------------------

%Captital Letters
\newcommand{\mfA}{\mathfrak{A}}	\newcommand{\mfB}{\mathfrak{B}}
\newcommand{\mfC}{\mathfrak{C}}	\newcommand{\mfD}{\mathfrak{D}}
\newcommand{\mfE}{\mathfrak{E}}	\newcommand{\mfF}{\mathfrak{F}}
\newcommand{\mfG}{\mathfrak{G}}	\newcommand{\mfH}{\mathfrak{H}}
\newcommand{\mfI}{\mathfrak{I}}	\newcommand{\mfJ}{\mathfrak{J}}
\newcommand{\mfK}{\mathfrak{K}}	\newcommand{\mfL}{\mathfrak{L}}
\newcommand{\mfM}{\mathfrak{M}}	\newcommand{\mfN}{\mathfrak{N}}
\newcommand{\mfO}{\mathfrak{O}}	\newcommand{\mfP}{\mathfrak{P}}
\newcommand{\mfQ}{\mathfrak{Q}}	\newcommand{\mfR}{\mathfrak{R}}
\newcommand{\mfS}{\mathfrak{S}}	\newcommand{\mfT}{\mathfrak{T}}
\newcommand{\mfU}{\mathfrak{U}}	\newcommand{\mfV}{\mathfrak{V}}
\newcommand{\mfW}{\mathfrak{W}}	\newcommand{\mfX}{\mathfrak{X}}
\newcommand{\mfY}{\mathfrak{Y}}	\newcommand{\mfZ}{\mathfrak{Z}}
%Small Letters
\newcommand{\mfa}{\mathfrak{a}}	\newcommand{\mfb}{\mathfrak{b}}
\newcommand{\mfc}{\mathfrak{c}}	\newcommand{\mfd}{\mathfrak{d}}
\newcommand{\mfe}{\mathfrak{e}}	\newcommand{\mff}{\mathfrak{f}}
\newcommand{\mfg}{\mathfrak{g}}	\newcommand{\mfh}{\mathfrak{h}}
\newcommand{\mfi}{\mathfrak{i}}	\newcommand{\mfj}{\mathfrak{j}}
\newcommand{\mfk}{\mathfrak{k}}	\newcommand{\mfl}{\mathfrak{l}}
\newcommand{\mfm}{\mathfrak{m}}	\newcommand{\mfn}{\mathfrak{n}}
\newcommand{\mfo}{\mathfrak{o}}	\newcommand{\mfp}{\mathfrak{p}}
\newcommand{\mfq}{\mathfrak{q}}	\newcommand{\mfr}{\mathfrak{r}}
\newcommand{\mfs}{\mathfrak{s}}	\newcommand{\mft}{\mathfrak{t}}
\newcommand{\mfu}{\mathfrak{u}}	\newcommand{\mfv}{\mathfrak{v}}
\newcommand{\mfw}{\mathfrak{w}}	\newcommand{\mfx}{\mathfrak{x}}
\newcommand{\mfy}{\mathfrak{y}}	\newcommand{\mfz}{\mathfrak{z}}

%---------------------------------------
% Bar
%---------------------------------------

%Captital Letters
\newcommand{\ovA}{\overline{A}}	\newcommand{\ovB}{\overline{B}}
\newcommand{\ovC}{\overline{C}}	\newcommand{\ovD}{\overline{D}}
\newcommand{\ovE}{\overline{E}}	\newcommand{\ovF}{\overline{F}}
\newcommand{\ovG}{\overline{G}}	\newcommand{\ovH}{\overline{H}}
\newcommand{\ovI}{\overline{I}}	\newcommand{\ovJ}{\overline{J}}
\newcommand{\ovK}{\overline{K}}	\newcommand{\ovL}{\overline{L}}
\newcommand{\ovM}{\overline{M}}	\newcommand{\ovN}{\overline{N}}
\newcommand{\ovO}{\overline{O}}	\newcommand{\ovP}{\overline{P}}
\newcommand{\ovQ}{\overline{Q}}	\newcommand{\ovR}{\overline{R}}
\newcommand{\ovS}{\overline{S}}	\newcommand{\ovT}{\overline{T}}
\newcommand{\ovU}{\overline{U}}	\newcommand{\ovV}{\overline{V}}
\newcommand{\ovW}{\overline{W}}	\newcommand{\ovX}{\overline{X}}
\newcommand{\ovY}{\overline{Y}}	\newcommand{\ovZ}{\overline{Z}}
%Small Letters
\newcommand{\ova}{\overline{a}}	\newcommand{\ovb}{\overline{b}}
\newcommand{\ovc}{\overline{c}}	\newcommand{\ovd}{\overline{d}}
\newcommand{\ove}{\overline{e}}	\newcommand{\ovf}{\overline{f}}
\newcommand{\ovg}{\overline{g}}	\newcommand{\ovh}{\overline{h}}
\newcommand{\ovi}{\overline{i}}	\newcommand{\ovj}{\overline{j}}
\newcommand{\ovk}{\overline{k}}	\newcommand{\ovl}{\overline{l}}
\newcommand{\ovm}{\overline{m}}	\newcommand{\ovn}{\overline{n}}
\newcommand{\ovo}{\overline{o}}	\newcommand{\ovp}{\overline{p}}
\newcommand{\ovq}{\overline{q}}	\newcommand{\ovr}{\overline{r}}
\newcommand{\ovs}{\overline{s}}	\newcommand{\ovt}{\overline{t}}
\newcommand{\ovu}{\overline{u}}	\newcommand{\ovv}{\overline{v}}
\newcommand{\ovw}{\overline{w}}	\newcommand{\ovx}{\overline{x}}
\newcommand{\ovy}{\overline{y}}	\newcommand{\ovz}{\overline{z}}

%---------------------------------------
% Tilde
%---------------------------------------

%Captital Letters
\newcommand{\tdA}{\tilde{A}}	\newcommand{\tdB}{\tilde{B}}
\newcommand{\tdC}{\tilde{C}}	\newcommand{\tdD}{\tilde{D}}
\newcommand{\tdE}{\tilde{E}}	\newcommand{\tdF}{\tilde{F}}
\newcommand{\tdG}{\tilde{G}}	\newcommand{\tdH}{\tilde{H}}
\newcommand{\tdI}{\tilde{I}}	\newcommand{\tdJ}{\tilde{J}}
\newcommand{\tdK}{\tilde{K}}	\newcommand{\tdL}{\tilde{L}}
\newcommand{\tdM}{\tilde{M}}	\newcommand{\tdN}{\tilde{N}}
\newcommand{\tdO}{\tilde{O}}	\newcommand{\tdP}{\tilde{P}}
\newcommand{\tdQ}{\tilde{Q}}	\newcommand{\tdR}{\tilde{R}}
\newcommand{\tdS}{\tilde{S}}	\newcommand{\tdT}{\tilde{T}}
\newcommand{\tdU}{\tilde{U}}	\newcommand{\tdV}{\tilde{V}}
\newcommand{\tdW}{\tilde{W}}	\newcommand{\tdX}{\tilde{X}}
\newcommand{\tdY}{\tilde{Y}}	\newcommand{\tdZ}{\tilde{Z}}
%Small Letters
\newcommand{\tda}{\tilde{a}}	\newcommand{\tdb}{\tilde{b}}
\newcommand{\tdc}{\tilde{c}}	\newcommand{\tdd}{\tilde{d}}
\newcommand{\tde}{\tilde{e}}	\newcommand{\tdf}{\tilde{f}}
\newcommand{\tdg}{\tilde{g}}	\newcommand{\tdh}{\tilde{h}}
\newcommand{\tdi}{\tilde{i}}	\newcommand{\tdj}{\tilde{j}}
\newcommand{\tdk}{\tilde{k}}	\newcommand{\tdl}{\tilde{l}}
\newcommand{\tdm}{\tilde{m}}	\newcommand{\tdn}{\tilde{n}}
\newcommand{\tdo}{\tilde{o}}	\newcommand{\tdp}{\tilde{p}}
\newcommand{\tdq}{\tilde{q}}	\newcommand{\tdr}{\tilde{r}}
\newcommand{\tds}{\tilde{s}}	\newcommand{\tdt}{\tilde{t}}
\newcommand{\tdu}{\tilde{u}}	\newcommand{\tdv}{\tilde{v}}
\newcommand{\tdw}{\tilde{w}}	\newcommand{\tdx}{\tilde{x}}
\newcommand{\tdy}{\tilde{y}}	\newcommand{\tdz}{\tilde{z}}

%---------------------------------------
% Vec
%---------------------------------------

%Captital Letters
\newcommand{\vcA}{\vec{A}}	\newcommand{\vcB}{\vec{B}}
\newcommand{\vcC}{\vec{C}}	\newcommand{\vcD}{\vec{D}}
\newcommand{\vcE}{\vec{E}}	\newcommand{\vcF}{\vec{F}}
\newcommand{\vcG}{\vec{G}}	\newcommand{\vcH}{\vec{H}}
\newcommand{\vcI}{\vec{I}}	\newcommand{\vcJ}{\vec{J}}
\newcommand{\vcK}{\vec{K}}	\newcommand{\vcL}{\vec{L}}
\newcommand{\vcM}{\vec{M}}	\newcommand{\vcN}{\vec{N}}
\newcommand{\vcO}{\vec{O}}	\newcommand{\vcP}{\vec{P}}
\newcommand{\vcQ}{\vec{Q}}	\newcommand{\vcR}{\vec{R}}
\newcommand{\vcS}{\vec{S}}	\newcommand{\vcT}{\vec{T}}
\newcommand{\vcU}{\vec{U}}	\newcommand{\vcV}{\vec{V}}
\newcommand{\vcW}{\vec{W}}	\newcommand{\vcX}{\vec{X}}
\newcommand{\vcY}{\vec{Y}}	\newcommand{\vcZ}{\vec{Z}}
%Small Letters
\newcommand{\vca}{\vec{a}}	\newcommand{\vcb}{\vec{b}}
\newcommand{\vcc}{\vec{c}}	\newcommand{\vcd}{\vec{d}}
\newcommand{\vce}{\vec{e}}	\newcommand{\vcf}{\vec{f}}
\newcommand{\vcg}{\vec{g}}	\newcommand{\vch}{\vec{h}}
\newcommand{\vci}{\vec{i}}	\newcommand{\vcj}{\vec{j}}
\newcommand{\vck}{\vec{k}}	\newcommand{\vcl}{\vec{l}}
\newcommand{\vcm}{\vec{m}}	\newcommand{\vcn}{\vec{n}}
\newcommand{\vco}{\vec{o}}	\newcommand{\vcp}{\vec{p}}
\newcommand{\vcq}{\vec{q}}	\newcommand{\vcr}{\vec{r}}
\newcommand{\vcs}{\vec{s}}	\newcommand{\vct}{\vec{t}}
\newcommand{\vcu}{\vec{u}}	\newcommand{\vcv}{\vec{v}}
\newcommand{\vcw}{\vec{w}}	\newcommand{\vcx}{\vec{x}}
\newcommand{\vcy}{\vec{y}}	\newcommand{\vcz}{\vec{z}}

%---------------------------------------
% Greek Letters:-
%---------------------------------------
\newcommand{\eps}{\epsilon}
\newcommand{\veps}{\varepsilon}
\newcommand{\lm}{\lambda}
\newcommand{\Lm}{\Lambda}
\newcommand{\gm}{\gamma}
\newcommand{\Gm}{\Gamma}
\newcommand{\vph}{\varphi}
\newcommand{\ph}{\phi}
\newcommand{\om}{\omega}
\newcommand{\Om}{\Omega}

%%%%%%%%%%%%%%%%%%%%%%%%%%%%%%%%%%%%%%%% MACROS %%%%%%%%%%%%%%%%%%%%%%%%%%%%%%%%%%%%%%%%

%%%%%%%%%%%%%%% Link With an Icon %%%%%%%%%%%%%%% 
\newcommand{\link}[1]{
    \href{#1}{\faIcon{link}}
}

%%%%%%%%%%%%%%% Name Template %%%%%%%%%%%%%%% 
\newcommand{\name}[2]{
    % Name
    \Huge % Font size
    \raggedright \textbf{#1} \par

    \vspace*{0.3cm}
    
    % Profession
    \Large % Font size
    \raggedright #2 \par
    \normalsize \normalfont
}

%%%%%%%%%%%%%%% Contact Details %%%%%%%%%%%%%%%
\newcommand{\info}[2]{
    \faIcon{#2} \hspace{0.2em} #1
}

%%%%%%%%%%%%%%% Email %%%%%%%%%%%%%%%
\newcommand{\email}[1]{
    \info{#1}{envelope}
}

%%%%%%%%%%%%%%% Phone Number %%%%%%%%%%%%%%%
\newcommand{\phone}[1]{
    \info{#1}{mobile-alt}
}

%%%%%%%%%%%%%%% Address %%%%%%%%%%%%%%%
\newcommand{\address}[1]{
    \info{#1}{map-marker-alt}
}

%%%%%%%%%%%%%%% GitHub %%%%%%%%%%%%%%%
\newcommand{\github}[2]{
    \info{\href{#1}{\underline{#2}}}{github}
}

%%%%%%%%%%%%%%% LinkedIn %%%%%%%%%%%%%%%
\newcommand{\linkedin}[2]{
    \info{\href{#1}{\underline{#2}}}{linkedin}
}

%%%%%%%%%%%%%%% ResearchGate %%%%%%%%%%%%%%%
\newcommand{\researchgate}[2]{
    \info{\href{#1}{\underline{#2}}}{researchgate}
}

%%%\newcommand*{\Researchgate}[1]{\sociallink{\researchgatesocialsymbol}{http://www.#1}{#1}}

%%%%%%%%%%%%%%% Website %%%%%%%%%%%%%%%
\newcommand{\website}[1]{
    \info{#1}{link}
}

%%%%%%%%%%%%%%% Draw Skill Bars %%%%%%%%%%%%%%% 
\newcommand{\drawskillbars}[1]{
    \begin{tikzpicture}
        % Draw 5 gray bars
        \foreach \i in {0, 1, 2, 3, 4}{
            \fill[lightgray] (\i * 0.7 + 0.2 *\i,0) rectangle (0.7 + \i * 0.7 + \i * 0.2,0.1);
        }
        
        % Draw number of black bars depending on the skill level
        \foreach \i in {#1}{
            \fill[blue!40] (\i * 0.7 + 0.2 *\i,0) rectangle (0.7 + \i * 0.7 + \i * 0.2,0.1);
            %\fill[title] (\i * 0.7 + 0.2 *\i,0) rectangle (0.7 + \i * 0.7 + \i * 0.2,0.1);
        }
    \end{tikzpicture} \par
}
    
%%%%%%%%%%%%%%% Skills %%%%%%%%%%%%%%%
\newcommand{\skill}[3]{
    % Name of the skill
    \large
    \noindent \hangafter=0
    \adjustbox{valign=t}{\begin{minipage}{0.72\textwidth}
        \large \noindent \hangafter=0
        % Name of the skill
        \textmd{#1} 
        \normalsize \par 
        \vspace{1em}
         % Description
        \noindent \small \color{subtitle} \parbox{1\linewidth}{\textsl{#3}} \par
        \normalsize \par
        \end{minipage}}
    \adjustbox{valign=t}{\begin{minipage}{0.2\textwidth}
        % Skill bars
        \large \hangafter=0
        %\noindent 
        \drawskillbars{#2}
        \end{minipage}}
    \normalsize \par 
    % Skill bars
    %%\drawskillbars{#2}
    %%\vspace{0.5em}
    
    \vspace{1.0em}
    \normalsize \color{black} \par
}

%%%%%%%%%%%%%%% Software %%%%%%%%%%%%%%%
\newcommand{\soft}[2]{
    \adjustbox{valign=t}{\begin{minipage}{0.40\textwidth}
        \large \noindent \hangafter=0
        % Name of the skill
        \textmd{#1} 
        \normalsize \par 
        \vspace{1em}
        \end{minipage}}
    \adjustbox{valign=t}{\begin{minipage}{0.5\textwidth}
        % Skill bars
        \large \noindent \hangafter=0
        \drawskillbars{#2}
        \end{minipage}}
    \normalsize \par 
    \vspace{1em}
}

%%%%%%%%%%%%%%% Personal details %%%%%%%%%%%%%%%
\newcommand{\details}[2]{
    % Name of the language
    \large
    \noindent \hangafter=0 \color{black}
    \adjustbox{valign=t}{\parbox{0.27\linewidth}{#1}}  \adjustbox{valign=t}{\parbox{0.55\linewidth}{#2}} \par
    \vspace{.3em}
    \normalsize \color{black} \par
 }

%%%%%%%%%%%%%%% Language %%%%%%%%%%%%%%%
\newcommand{\lan}[2]{
    % Name of the language
    \large
    \noindent \hangafter=0 \color{black}
    \parbox{0.3\linewidth}{\textmd{#1}}   \color{subtitle} \parbox{0.4\linewidth}{\textsl{#2}} \par
    %\large English \color{subtitle} \textit{Advanced} 
    %\normalsize \par 
    % Knowledge level
    %\noindent \small \color{subtitle} \parbox{1\linewidth}{\textsl{#2}} \par
    \vspace{1.0em}
    \normalsize \color{black} \par
 }

%%%%%%%%%%%%%%% Education %%%%%%%%%%%%%%%
\newcommand{\education}[4]{
    % Name of the studies
    \noindent \large \parbox{.65\linewidth}{\textbf{#1}}
    % Duration in a Box
    \hfill \small
    \tcbox[enhanced,nobeforeafter,box align=base,colback=title,colframe=title,size=fbox,arc=0mm, valign=bottom]{{\textbf{#2}}} \par
    \vspace{0.3em}
    % School Name 
    \normalsize
    \noindent \color{subtitle} \parbox{.9\linewidth}{\textsl{#3}} \par
    % Description
    \normalsize \color{black}
    \vspace*{0.3em}
    \small #4 
    \normalsize \par
    \vspace*{0.5em}
}

%%%%%%%%%%%%%%% Work Experience %%%%%%%%%%%%%%%
\newcommand{\work}[4]{
    % Name of the Job
    \noindent \large \parbox{.65\linewidth}{\textbf{#1}}
    % Duration in a Box 
    \hfill \small
    \tcbox[enhanced,box align=base,nobeforeafter,colback=title,colframe=title,size=fbox,arc=0mm]{\textbf{#2}} \par
    \vspace{0.3em}
    % Name of the Employer
    \noindent \large \color{subtitle} \parbox{.9\linewidth}{\textsl{#3}} \par
    % Description of the job
    \vspace*{0.3em} \color{black}
    \small #4 
    \normalsize \par
}

%%%%%%%%%%%%%%% Teaching %%%%%%%%%%%%%%%
\newcommand{\teaching}[3]{
    % What, Topic and Who/Where/when
    \noindent \adjustbox{valign=t}{\parbox{.99\linewidth}{\text{#1} \text{#2} \textbf{#3}}} 
    \vspace{0.5em}
    \vspace*{1em} 
}

%%%%%%%%%%%%%%% Publications %%%%%%%%%%%%%%%
\newcommand{\publ}[4]{
    % Authors, Title and journal
    \noindent \parbox{.99\linewidth}{\textsl{#3}. \textbf{#1} \textsl{#2} \link{#4}}
    \vspace{0.5em}
    \vspace*{1em} \color{black}
    }

%%%%%%%%%%%%%%% Talks %%%%%%%%%%%%%%%
\newcommand{\talk}[3]{
    % Authors, Title and journal
    \noindent \parbox{.99\linewidth}{\textsl{#3}. \textbf{#1} \textsl{#2}}
    \vspace{0.5em}
    \vspace*{1em} \color{black}
}

%%%%%%%%%%%%%%% Events %%%%%%%%%%%%%%%
\newcommand{\event}[3]{
    \noindent \parbox{.99\linewidth}{\textbf{#1} \textsl{#3} \textsl{#2}}
    \vspace{0.5em}
    \vspace*{1em} \color{black}
}

\definecolor{mytheorembg}{HTML}{F2F2F9}
\definecolor{mytheoremfr}{HTML}{00007B}


\tcbuselibrary{theorems,skins,hooks}
\newtcbtheorem{problem}{Problem}
{%
	enhanced,
	breakable,
	colback = mytheorembg,
	frame hidden,
	boxrule = 0sp,
	borderline west = {2pt}{0pt}{mytheoremfr},
	sharp corners,
	detach title,
	before upper = \tcbtitle\par\smallskip,
	coltitle = mytheoremfr,
	fonttitle = \bfseries\sffamily,
	description font = \mdseries,
	separator sign none,
	segmentation style={solid, mytheoremfr},
}
{p}

\setlength{\parindent}{0pt}

%%%%%%%%%%%%%%%%%%%%%%%%%%%%%%%%%%%%%%%%%%%%%%%%%%%%%%%%%%%%%%%%%%%%%%%%%%%%%%%%%%%%%%%%%%%%%%%%%%%%%%%%%%%%%%%%%%%%%%%%%%%%%%%%%%%%%%%%

\begin{document}
	%Header-Make sure you update this information!!!!
	
	%%%%%%%%%%%%%%%%%%%%%%%%%%%%%%%%%%%%%%%%%%%%%%%%%%%%%%%%%%%%%%%%%%%%%%%%%%%%%%%%%%%%%%%%%%%%%%%%%%%%%%%%%%%%%%%%%%%%%%%%%%%%%%%%%%%%%%%%
	
	\textsf{\noindent \large\textbf{Soham Chatterjee} \hfill \textbf{Assignment - 2}\\
		Email: \href{sohamc@cmi.ac.in}{sohamc@cmi.ac.in} \hfill Roll: BMC202175\\
		\normalsize Course: Probability Theory \hfill Date: April 8, 2022 \\
		\noindent\rule{7in}{2.8pt}}
	
	%%%%%%%%%%%%%%%%%%%%%%%%%%%%%%%%%%%%%%%%%%%%%%%%%%%%%%%%%%%%%%%%%%%%%%%%%%%%%%%%%%%%%%%%%%%%%%%%%%%%%%%%%%%%%%%%%%%%%%%%%%%%%%%%%%%%%%%%
	% Problem 1
	%%%%%%%%%%%%%%%%%%%%%%%%%%%%%%%%%%%%%%%%%%%%%%%%%%%%%%%%%%%%%%%%%%%%%%%%%%%%%%%%%%%%%%%%%%%%%%%%%%%%%%%%%%%%%%%%%%%%%%%%%%%%%%%%%%%%%%%%
	
	\begin{problem}{}{p1}
		Suppose that $X_1,X_2,X_3$ are identical independent random variables taking values in positive integers where $P(X_j=k)=(1-p)p^{k-1}$ (where $0 < p < 1$). Find the probability of that $X_1\leq X_2\leq X_3$.
		%Probleem		
	\end{problem}
	
	\solve{
		%Solution
		We have, $$P(X_j=k)=(1-p)p^{k-1}\ \forall\ k\in \bbN,\ j\in\{1,2,3\}$$Given that $X_1,X_2,X_3$ are independent.  Now\begin{align*}
			P(X_1\leq X_2\leq X_3) = & \sum_{k=1}^{\infty}P(X_1\leq k, X_2=k, X_3\geq k)                                                                 \\
			=                        & \sum_{k=1}^{\infty}P(X_1\leq k)P(X_2=k)P(X_3\geq k)                                                               \\
			=                        & \sum_{k=1}^{\infty} \lt[\lt(\sum_{i=1}^{k}P(X_1=i)\rt)P(X_2=k)\lt(\sum_{j=k}^{\infty}P(X_3=j)\rt)\rt]             \\
			=                        & \sum_{k=1}^{\infty} \lt[\lt(\sum_{i=1}^{k}(1-p)p^{i-1}\rt)(1-p)p^{k-1}\lt(\sum_{j=k}^{\infty}(1-p)p^{j-1}\rt)\rt] \\
			=                        & (1-p)^3\sum_{k=1}^{\infty} \lt[\lt(\sum_{i=0}^{k-1}p^{i}\rt)p^{k-1}\lt(\sum_{j=k-1}^{\infty}p^{j}\rt)\rt]         \\
			=                        & (1-p)^3\sum_{k=1}^{\infty} \lt[\lt(\sum_{i=0}^{k-1}p^{i}\rt)p^{k-1}\lt(p^{k-1}\sum_{j=0}^{\infty}p^{j}\rt)\rt]    \\
			=                        & (1-p)^3\sum_{k=1}^{\infty} \lt[\frac{1-p^k}{1-p}\, p^{k-1}\, \frac{p^{k-1}}{1-p}\rt]                              \\
			=                        & (1-p)\sum_{k=1}^{\infty} p^{2(k-1)}(1-p^k)                                                                        \\
			=                        & (1-p)\sum_{k=1}^{\infty} \lt(p^{2(k-1)}-p\times p^{3(k-1)}\rt)                                                    \\
			=                        & (1-p)\sum_{k=0}^{\infty} p^{2k}-p \sum_{k=0}^{\infty}p^{3k}                                                       \\
			=                        & (1-p)\lt[\sum_{k=0}^{\infty} p^{2k}-p \sum_{k=0}^{\infty}p^{3k}\rt]                                               \\
			=                        & (1-p)\lt[ \frac{1}{1-p^2}-\frac{p}{1-p^3}\rt]                                                                     \\
			=                        & \frac1{1+p}-\frac{p}{1+p+p^2}                                                                                     \\
			=                        & \frac{1+p+p^2-p(1+p)}{(1+p)(1+p+p^2)}=\frac{1}{(1+p)(1+p+p^2)}
		\end{align*}
	}
	
	
	%%%%%%%%%%%%%%%%%%%%%%%%%%%%%%%%%%%%%%%%%%%%%%%%%%%%%%%%%%%%%%%%%%%%%%%%%
	% Problem 2
	%%%%%%%%%%%%%%%%%%%%%%%%%%%%%%%%%%%%%%%%%%%%%%%%%%%%%%%%%%%%%%%%%%%%%%%%%
	
	\begin{problem}{}{%Problem Name
		}
		 Suppose that three players $A$, $B$, $C$ take turns to throw a biased coin successively in	cyclic order $A, B, C, A, B, C,\dots.$ Let $P(H) = p$. Find the probability that $A$ is the first person to throw heads, next $B$ and finally $C$.
	\end{problem}
	
	\solve{
		%Solution
		Let $q=1-p$. Let $X_A$, $X_B$, $X_C$ are three random variables where $X_A=n$ if $A$ gets his first head at $n-$th turn. Similarly $X_B=n$  if $B$ gets his first head at $n-th$ turn and $X_C=n$ if $C$ gets his first head at $n-th$ turn. Hence $P(X_A=n)=(1-p)^{n-1}p=(1-q)q^{n-1}$. Similarly $P(X_B=n)=(1-p)^{n-1}=(1-q)q^{n-1}$ and $P(X_C=n)=(1-p)^{n-1}p=(1-q)q^{n-1}$Now $X_A, X_B, X_C$ are independent. Hence we have to find the probability of $X_A\leq X_B\leq X_C$. Hence by applying the result in \hyperref[p:p1]{Problem \ref{p:p1}} we get $$P(X_A\leq X_B\leq X_C)=\frac1{(1+q)(1+q+q^2)}=\frac1{1+(1-p)(1+(1-p)+(1-p)^2)}=\frac{1}{(2-p)(3-3p+p^2)}$$Therefore probability of $A$ getting the first head then $B$ then $C$ is $\frac{1}{(2-p)(3-3p+p^2)}$
	}
	
	
	%%%%%%%%%%%%%%%%%%%%%%%%%%%%%%%%%%%%%%%%%%%%%%%%%%%%%%%%%%%%%%%%%%%%%%%%%
	% Problem 3
	%%%%%%%%%%%%%%%%%%%%%%%%%%%%%%%%%%%%%%%%%%%%%%%%%%%%%%%%%%%%%%%%%%%%%%%%%
	
	\begin{problem}{}{%Problem Name
		}
		%Problem		
		Each member of a group of $n$ students is assigned a number at random from the set $\{0, 1, 2, \dots, 9\}$. The sum of the numbers assigned is the total score that the group obtains. Find the mean of the total score. You may assume that any number is equally likely to be assigned to any student.
	\end{problem}
	
	\solve{%Solution
				Let $X$ be the random variable which takes the values of all possible total scores obtained by adding all the numbers obtained from each student. Hence we need to find $E(X)$. Let $X_1,X_2,\dots, X_{n}$ random variables where each $X_i$ takes values from the set $\{0,1,2,\dots, 9\}$ which represents that the number assigned to $i-$th student is $X_i$. Since all numbers in the set $\{0,1,2,\dots, 9\}$ are equally likely, $P(X_i=k)=\frac1{10}$ where $k\in\{0,1,2,\dots, 9\}$. Hence $X=\sum_{i=1}^nX_i$ and therefore $$E(X)=\sum_{i=1}^nE(X_i)=\sum_{i=1}^n\lt[\sum_{k=0}^9kP(X_i=k)\rt]=\sum_{i=1}^n\lt[\sum_{k=0}^9 k\, \frac1{10}\rt]=\sum_{i=1}^n \frac{45}{10}=\frac{9n}{2}$$
	}
	
	
	%%%%%%%%%%%%%%%%%%%%%%%%%%%%%%%%%%%%%%%%%%%%%%%%%%%%%%%%%%%%%%%%%%%%%%%%%
	% Problem 4
	%%%%%%%%%%%%%%%%%%%%%%%%%%%%%%%%%%%%%%%%%%%%%%%%%%%%%%%%%%%%%%%%%%%%%%%%%
	
	\begin{problem}{}{%Problem Name
		}
		%Problem
		Suppose that an urn contains $n$ balls numbered from 1 to $n$. In an experiment $k$ balls are drawn at random and their numbers are added to get $S$. Find the variance of $S$.
	\end{problem}
	
	\solve{
		%Solution
		Let $X$ be the random variable which takes the value of the sum obtained by adding the numbers on the $k$ balls randomly picked up from the urn. Hence we need find. Let $X_1,X_2,\cdots, X_n$ are the random variables such that for any $X_i$ $$X_i=\begin{cases*}
			1 & \text{when }i-\text{th ball was picked up}\\
			0 & \text{when }i-\text{th ball was not picked up}
		\end{cases*} $$Therefore for any $X_i$ $$P(X_i=1)= \frac{\text{The number of ways to choose other }k-1\text{ balls from }n-1\text{ balls}}{\text{The number of ways to choose }k \text{ balls from }n\text{ balls}}=\frac{\binom{n-1}{k-1}}{\binom{n}{k}}=\frac{k}{n}$$Hence $X=\sum\limits_{i=1}^n iX_i$. Hence 
		$$E(X)  =\sum_{i=1}^niE(X_i) =\sum_{i=1}^ni\frac{k}{n} =\frac{k}{n}\, \frac{n(n+1)}{2}=\frac{k(n+1)}{2}$$
		
		Now $$X^2=\lt[\sum\limits_{i=1}^niX_i\rt]^2=\sum_{i=1}^ni^2X_i^2+\sum_{1\leq i<j\leq n}2ijX_iX_j\implies E(X^2)=\sum_{i=1}^ni^2E(X_i^2)+\sum_{1\leq i<j\leq n}2ijE(X_iX_j)$$
		$$\sum_{i=1}^ni^2E(X_i^2)=\sum_{i=1}^ni^2\frac{k}{n}=\frac{k}{n}\, \frac{n(n+1)(2n+1)}{6}=\frac{k(n+1)(2n+1)}{6}$$Now, \begin{align*}
			E(X_iX_j) & =P(X_iX_j=1)                                                                                                                                                      \\
			          & =\text{Probability of both }i-\text{th and }j-\text{th ball were picked up }                                                                                      \\
			          & =\frac{\text{The number of ways to choose other }k-2\text{ balls from }n-2\text{ balls}}{\text{The number of ways to choose }k \text{ balls from }n\text{ balls}} \\
			          & =\frac{\binom{n-2}{k-2}}{\binom{n}{k}}=\frac{k(k-1)}{n(n-1)}
		\end{align*}
	Therefore \begin{align*}
		\sum_{1\leq i<j\leq n}2ijE(X_iX_j) & =\sum_{1\leq i<j\leq n}2ij\,\frac{k(k-1)}{n(n-1)}                                    \\
		                                   & =\frac{k(k-1)}{n(n-1)}\lt[\sum_{1\leq i,j\leq n}ij-\sum_{i=1}^ni^2\rt]               \\
		                                   & =\frac{k(k-1)}{n(n-1)}\lt[ \lt( \sum_{i=1}^ni\rt)^2-\sum_{i=1}^ni^2\rt]              \\
		                                   & =\frac{k(k-1)}{n(n-1)}\lt[ \lt( \frac{n(n+1)}{2} \rt)^2 -\frac{n(n+1)(2n+1)}{6} \rt] \\
		                                   & =\frac{k(k-1)}{n(n-1)}\lt[ \frac{n^2(n+1)^2}{4} -\frac{n(n+1)(2n+1)}{6} \rt]         \\
		                                   & =\frac{k(k-1)(n+1)}{2(n-1)} \lt[ \frac{n(n+1)}{2}-\frac{2n+1}{3} \rt]                \\
		                                   & =\frac{k(k-1)(n+1)}{2(n-1)}\, \frac{3n(n+1)-2(2n+1)}{6}                              \\
		                                   & =\frac{k(k-1)(n+1)}{2(n-1)}\, \frac{3n^2-n-2}{6}                                     \\
		                                   & =\frac{k(k-1)(n+1)}{2(n-1)}\, \frac{(n-1)(3n+2)}{6}                                  \\
		                                   & =\frac{k(k-1)(n+1)(3n+2)}{12}
	\end{align*}
	Hence$$E(X^2)=\sum_{i=1}^ni^2E(x_i^2)+\sum_{1\leq i<j\leq n}ijE(X_iX_j)=\frac{k(n+1)(2n+1)}{6}+\frac{k(k-1)(n+1)(3n+2)}{12}$$
	Therefore finally we get the variance\begin{align*}
		Var(X) &=E(X^2)-E(X)^2\\
		&=\frac{k(n+1)(2n+1)}{6}+\frac{k(k-1)(n+1)(3n+2)}{12}-\frac{k^2(n+1)^2}{4}\\
		&=\frac{k(n+1)}{2}\lt[ \frac{2n+1}{3} +\frac{(k-1)(3n+2)}{6} -\frac{k(n+1)}{2} \rt]\\
		&=\frac{k(n+1)}{2}\lt[ \frac{2n+1}{3} +\frac{(k-1)(3n+2)}{6} -\frac{k(n+1)}{2} \rt]\\
		&=\frac{k(n+1)}2 \,\frac{2(2n+1)+(k-1)(3n+2) 3k(n+1)}{6}\\
		&=\frac{k(n+1)}2 \,\frac{(4n+2)+(3nk+2k-3n-2) - (3nk+3k)}{6}\\
		&=\frac{k(n+1)}{2}\, \frac{n-k}{6}=\frac{k(n+1)(n-k)}{12}
	\end{align*}So $Var(X)=\dfrac{k(n+1)(n-k)}{12}$
	}
	
\end{document}