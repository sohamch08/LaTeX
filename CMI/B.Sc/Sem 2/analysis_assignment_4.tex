\documentclass[a4paper, 11pt]{article}
\usepackage{comment} % enables the use of multi-line comments (\ifx \fi) 
\usepackage{fullpage} % changes the margin
\usepackage[a4paper, total={7in, 10in}]{geometry}
\usepackage{amsmath,mathtools}
\usepackage{amssymb,amsthm}  % assumes amsmath package installed
\usepackage{float}
\usepackage{xcolor}
\usepackage{mdframed}
\usepackage[shortlabels]{enumitem}
\usepackage{indentfirst}
\usepackage{hyperref}
\hypersetup{
	colorlinks=true,
	linkcolor=blue,
	filecolor=magenta,
	urlcolor=blue!70!red,
	pdftitle={Analysis Assignment 4}, %%%%%%%%%%%%%%%%   WRITE ASSIGNMENT PDF NAME  %%%%%%%%%%%%%%%%%%%%
}
\usepackage[most,many,breakable]{tcolorbox}



\definecolor{mytheorembg}{HTML}{F2F2F9}
\definecolor{mytheoremfr}{HTML}{00007B}


\tcbuselibrary{theorems,skins,hooks}
\newtcbtheorem{problem}{Problem}
{%
	enhanced,
	breakable,
	colback = mytheorembg,
	frame hidden,
	boxrule = 0sp,
	borderline west = {2pt}{0pt}{mytheoremfr},
	sharp corners,
	detach title,
	before upper = \tcbtitle\par\smallskip,
	coltitle = mytheoremfr,
	fonttitle = \bfseries\sffamily,
	description font = \mdseries,
	separator sign none,
	segmentation style={solid, mytheoremfr},
}
{p}

% To give references for any problem use like this
% suppose the problem number is p3 then 2 options either 
% \hyperref[p:p3]{<text you want to use to hyperlink> \ref{p:p3}}
%                  or directly 
%                   \ref{p:p3}



%---------------------------------------
% BlackBoard Math Fonts :-
%---------------------------------------

%Captital Letters
\newcommand{\bbA}{\mathbb{A}}	\newcommand{\bbB}{\mathbb{B}}
\newcommand{\bbC}{\mathbb{C}}	\newcommand{\bbD}{\mathbb{D}}
\newcommand{\bbE}{\mathbb{E}}	\newcommand{\bbF}{\mathbb{F}}
\newcommand{\bbG}{\mathbb{G}}	\newcommand{\bbH}{\mathbb{H}}
\newcommand{\bbI}{\mathbb{I}}	\newcommand{\bbJ}{\mathbb{J}}
\newcommand{\bbK}{\mathbb{K}}	\newcommand{\bbL}{\mathbb{L}}
\newcommand{\bbM}{\mathbb{M}}	\newcommand{\bbN}{\mathbb{N}}
\newcommand{\bbO}{\mathbb{O}}	\newcommand{\bbP}{\mathbb{P}}
\newcommand{\bbQ}{\mathbb{Q}}	\newcommand{\bbR}{\mathbb{R}}
\newcommand{\bbS}{\mathbb{S}}	\newcommand{\bbT}{\mathbb{T}}
\newcommand{\bbU}{\mathbb{U}}	\newcommand{\bbV}{\mathbb{V}}
\newcommand{\bbW}{\mathbb{W}}	\newcommand{\bbX}{\mathbb{X}}
\newcommand{\bbY}{\mathbb{Y}}	\newcommand{\bbZ}{\mathbb{Z}}

%---------------------------------------
% MathCal Fonts :-
%---------------------------------------

%Captital Letters
\newcommand{\mcA}{\mathcal{A}}	\newcommand{\mcB}{\mathcal{B}}
\newcommand{\mcC}{\mathcal{C}}	\newcommand{\mcD}{\mathcal{D}}
\newcommand{\mcE}{\mathcal{E}}	\newcommand{\mcF}{\mathcal{F}}
\newcommand{\mcG}{\mathcal{G}}	\newcommand{\mcH}{\mathcal{H}}
\newcommand{\mcI}{\mathcal{I}}	\newcommand{\mcJ}{\mathcal{J}}
\newcommand{\mcK}{\mathcal{K}}	\newcommand{\mcL}{\mathcal{L}}
\newcommand{\mcM}{\mathcal{M}}	\newcommand{\mcN}{\mathcal{N}}
\newcommand{\mcO}{\mathcal{O}}	\newcommand{\mcP}{\mathcal{P}}
\newcommand{\mcQ}{\mathcal{Q}}	\newcommand{\mcR}{\mathcal{R}}
\newcommand{\mcS}{\mathcal{S}}	\newcommand{\mcT}{\mathcal{T}}
\newcommand{\mcU}{\mathcal{U}}	\newcommand{\mcV}{\mathcal{V}}
\newcommand{\mcW}{\mathcal{W}}	\newcommand{\mcX}{\mathcal{X}}
\newcommand{\mcY}{\mathcal{Y}}	\newcommand{\mcZ}{\mathcal{Z}}



%---------------------------------------
% Bold Math Fonts :-
%---------------------------------------

%Captital Letters
\newcommand{\bmA}{\boldsymbol{A}}	\newcommand{\bmB}{\boldsymbol{B}}
\newcommand{\bmC}{\boldsymbol{C}}	\newcommand{\bmD}{\boldsymbol{D}}
\newcommand{\bmE}{\boldsymbol{E}}	\newcommand{\bmF}{\boldsymbol{F}}
\newcommand{\bmG}{\boldsymbol{G}}	\newcommand{\bmH}{\boldsymbol{H}}
\newcommand{\bmI}{\boldsymbol{I}}	\newcommand{\bmJ}{\boldsymbol{J}}
\newcommand{\bmK}{\boldsymbol{K}}	\newcommand{\bmL}{\boldsymbol{L}}
\newcommand{\bmM}{\boldsymbol{M}}	\newcommand{\bmN}{\boldsymbol{N}}
\newcommand{\bmO}{\boldsymbol{O}}	\newcommand{\bmP}{\boldsymbol{P}}
\newcommand{\bmQ}{\boldsymbol{Q}}	\newcommand{\bmR}{\boldsymbol{R}}
\newcommand{\bmS}{\boldsymbol{S}}	\newcommand{\bmT}{\boldsymbol{T}}
\newcommand{\bmU}{\boldsymbol{U}}	\newcommand{\bmV}{\boldsymbol{V}}
\newcommand{\bmW}{\boldsymbol{W}}	\newcommand{\bmX}{\boldsymbol{X}}
\newcommand{\bmY}{\boldsymbol{Y}}	\newcommand{\bmZ}{\boldsymbol{Z}}
%Small Letters
\newcommand{\bma}{\boldsymbol{a}}	\newcommand{\bmb}{\boldsymbol{b}}
\newcommand{\bmc}{\boldsymbol{c}}	\newcommand{\bmd}{\boldsymbol{d}}
\newcommand{\bme}{\boldsymbol{e}}	\newcommand{\bmf}{\boldsymbol{f}}
\newcommand{\bmg}{\boldsymbol{g}}	\newcommand{\bmh}{\boldsymbol{h}}
\newcommand{\bmi}{\boldsymbol{i}}	\newcommand{\bmj}{\boldsymbol{j}}
\newcommand{\bmk}{\boldsymbol{k}}	\newcommand{\bml}{\boldsymbol{l}}
\newcommand{\bmm}{\boldsymbol{m}}	\newcommand{\bmn}{\boldsymbol{n}}
\newcommand{\bmo}{\boldsymbol{o}}	\newcommand{\bmp}{\boldsymbol{p}}
\newcommand{\bmq}{\boldsymbol{q}}	\newcommand{\bmr}{\boldsymbol{r}}
\newcommand{\bms}{\boldsymbol{s}}	\newcommand{\bmt}{\boldsymbol{t}}
\newcommand{\bmu}{\boldsymbol{u}}	\newcommand{\bmv}{\boldsymbol{v}}
\newcommand{\bmw}{\boldsymbol{w}}	\newcommand{\bmx}{\boldsymbol{x}}
\newcommand{\bmy}{\boldsymbol{y}}	\newcommand{\bmz}{\boldsymbol{z}}

%---------------------------------------
% Scr Math Fonts :-
%---------------------------------------

\newcommand{\sA}{{\mathscr{A}}}   \newcommand{\sB}{{\mathscr{B}}}
\newcommand{\sC}{{\mathscr{C}}}   \newcommand{\sD}{{\mathscr{D}}}
\newcommand{\sE}{{\mathscr{E}}}   \newcommand{\sF}{{\mathscr{F}}}
\newcommand{\sG}{{\mathscr{G}}}   \newcommand{\sH}{{\mathscr{H}}}
\newcommand{\sI}{{\mathscr{I}}}   \newcommand{\sJ}{{\mathscr{J}}}
\newcommand{\sK}{{\mathscr{K}}}   \newcommand{\sL}{{\mathscr{L}}}
\newcommand{\sM}{{\mathscr{M}}}   \newcommand{\sN}{{\mathscr{N}}}
\newcommand{\sO}{{\mathscr{O}}}   \newcommand{\sP}{{\mathscr{P}}}
\newcommand{\sQ}{{\mathscr{Q}}}   \newcommand{\sR}{{\mathscr{R}}}
\newcommand{\sS}{{\mathscr{S}}}   \newcommand{\sT}{{\mathscr{T}}}
\newcommand{\sU}{{\mathscr{U}}}   \newcommand{\sV}{{\mathscr{V}}}
\newcommand{\sW}{{\mathscr{W}}}   \newcommand{\sX}{{\mathscr{X}}}
\newcommand{\sY}{{\mathscr{Y}}}   \newcommand{\sZ}{{\mathscr{Z}}}


%---------------------------------------
% Math Fraktur Font
%---------------------------------------

%Captital Letters
\newcommand{\mfA}{\mathfrak{A}}	\newcommand{\mfB}{\mathfrak{B}}
\newcommand{\mfC}{\mathfrak{C}}	\newcommand{\mfD}{\mathfrak{D}}
\newcommand{\mfE}{\mathfrak{E}}	\newcommand{\mfF}{\mathfrak{F}}
\newcommand{\mfG}{\mathfrak{G}}	\newcommand{\mfH}{\mathfrak{H}}
\newcommand{\mfI}{\mathfrak{I}}	\newcommand{\mfJ}{\mathfrak{J}}
\newcommand{\mfK}{\mathfrak{K}}	\newcommand{\mfL}{\mathfrak{L}}
\newcommand{\mfM}{\mathfrak{M}}	\newcommand{\mfN}{\mathfrak{N}}
\newcommand{\mfO}{\mathfrak{O}}	\newcommand{\mfP}{\mathfrak{P}}
\newcommand{\mfQ}{\mathfrak{Q}}	\newcommand{\mfR}{\mathfrak{R}}
\newcommand{\mfS}{\mathfrak{S}}	\newcommand{\mfT}{\mathfrak{T}}
\newcommand{\mfU}{\mathfrak{U}}	\newcommand{\mfV}{\mathfrak{V}}
\newcommand{\mfW}{\mathfrak{W}}	\newcommand{\mfX}{\mathfrak{X}}
\newcommand{\mfY}{\mathfrak{Y}}	\newcommand{\mfZ}{\mathfrak{Z}}
%Small Letters
\newcommand{\mfa}{\mathfrak{a}}	\newcommand{\mfb}{\mathfrak{b}}
\newcommand{\mfc}{\mathfrak{c}}	\newcommand{\mfd}{\mathfrak{d}}
\newcommand{\mfe}{\mathfrak{e}}	\newcommand{\mff}{\mathfrak{f}}
\newcommand{\mfg}{\mathfrak{g}}	\newcommand{\mfh}{\mathfrak{h}}
\newcommand{\mfi}{\mathfrak{i}}	\newcommand{\mfj}{\mathfrak{j}}
\newcommand{\mfk}{\mathfrak{k}}	\newcommand{\mfl}{\mathfrak{l}}
\newcommand{\mfm}{\mathfrak{m}}	\newcommand{\mfn}{\mathfrak{n}}
\newcommand{\mfo}{\mathfrak{o}}	\newcommand{\mfp}{\mathfrak{p}}
\newcommand{\mfq}{\mathfrak{q}}	\newcommand{\mfr}{\mathfrak{r}}
\newcommand{\mfs}{\mathfrak{s}}	\newcommand{\mft}{\mathfrak{t}}
\newcommand{\mfu}{\mathfrak{u}}	\newcommand{\mfv}{\mathfrak{v}}
\newcommand{\mfw}{\mathfrak{w}}	\newcommand{\mfx}{\mathfrak{x}}
\newcommand{\mfy}{\mathfrak{y}}	\newcommand{\mfz}{\mathfrak{z}}

%---------------------------------------
% Bar
%---------------------------------------

%Captital Letters
\newcommand{\ovA}{\overline{A}}	\newcommand{\ovB}{\overline{B}}
\newcommand{\ovC}{\overline{C}}	\newcommand{\ovD}{\overline{D}}
\newcommand{\ovE}{\overline{E}}	\newcommand{\ovF}{\overline{F}}
\newcommand{\ovG}{\overline{G}}	\newcommand{\ovH}{\overline{H}}
\newcommand{\ovI}{\overline{I}}	\newcommand{\ovJ}{\overline{J}}
\newcommand{\ovK}{\overline{K}}	\newcommand{\ovL}{\overline{L}}
\newcommand{\ovM}{\overline{M}}	\newcommand{\ovN}{\overline{N}}
\newcommand{\ovO}{\overline{O}}	\newcommand{\ovP}{\overline{P}}
\newcommand{\ovQ}{\overline{Q}}	\newcommand{\ovR}{\overline{R}}
\newcommand{\ovS}{\overline{S}}	\newcommand{\ovT}{\overline{T}}
\newcommand{\ovU}{\overline{U}}	\newcommand{\ovV}{\overline{V}}
\newcommand{\ovW}{\overline{W}}	\newcommand{\ovX}{\overline{X}}
\newcommand{\ovY}{\overline{Y}}	\newcommand{\ovZ}{\overline{Z}}
%Small Letters
\newcommand{\ova}{\overline{a}}	\newcommand{\ovb}{\overline{b}}
\newcommand{\ovc}{\overline{c}}	\newcommand{\ovd}{\overline{d}}
\newcommand{\ove}{\overline{e}}	\newcommand{\ovf}{\overline{f}}
\newcommand{\ovg}{\overline{g}}	\newcommand{\ovh}{\overline{h}}
\newcommand{\ovi}{\overline{i}}	\newcommand{\ovj}{\overline{j}}
\newcommand{\ovk}{\overline{k}}	\newcommand{\ovl}{\overline{l}}
\newcommand{\ovm}{\overline{m}}	\newcommand{\ovn}{\overline{n}}
\newcommand{\ovo}{\overline{o}}	\newcommand{\ovp}{\overline{p}}
\newcommand{\ovq}{\overline{q}}	\newcommand{\ovr}{\overline{r}}
\newcommand{\ovs}{\overline{s}}	\newcommand{\ovt}{\overline{t}}
\newcommand{\ovu}{\overline{u}}	\newcommand{\ovv}{\overline{v}}
\newcommand{\ovw}{\overline{w}}	\newcommand{\ovx}{\overline{x}}
\newcommand{\ovy}{\overline{y}}	\newcommand{\ovz}{\overline{z}}

%---------------------------------------
% Tilde
%---------------------------------------

%Captital Letters
\newcommand{\tdA}{\tilde{A}}	\newcommand{\tdB}{\tilde{B}}
\newcommand{\tdC}{\tilde{C}}	\newcommand{\tdD}{\tilde{D}}
\newcommand{\tdE}{\tilde{E}}	\newcommand{\tdF}{\tilde{F}}
\newcommand{\tdG}{\tilde{G}}	\newcommand{\tdH}{\tilde{H}}
\newcommand{\tdI}{\tilde{I}}	\newcommand{\tdJ}{\tilde{J}}
\newcommand{\tdK}{\tilde{K}}	\newcommand{\tdL}{\tilde{L}}
\newcommand{\tdM}{\tilde{M}}	\newcommand{\tdN}{\tilde{N}}
\newcommand{\tdO}{\tilde{O}}	\newcommand{\tdP}{\tilde{P}}
\newcommand{\tdQ}{\tilde{Q}}	\newcommand{\tdR}{\tilde{R}}
\newcommand{\tdS}{\tilde{S}}	\newcommand{\tdT}{\tilde{T}}
\newcommand{\tdU}{\tilde{U}}	\newcommand{\tdV}{\tilde{V}}
\newcommand{\tdW}{\tilde{W}}	\newcommand{\tdX}{\tilde{X}}
\newcommand{\tdY}{\tilde{Y}}	\newcommand{\tdZ}{\tilde{Z}}
%Small Letters
\newcommand{\tda}{\tilde{a}}	\newcommand{\tdb}{\tilde{b}}
\newcommand{\tdc}{\tilde{c}}	\newcommand{\tdd}{\tilde{d}}
\newcommand{\tde}{\tilde{e}}	\newcommand{\tdf}{\tilde{f}}
\newcommand{\tdg}{\tilde{g}}	\newcommand{\tdh}{\tilde{h}}
\newcommand{\tdi}{\tilde{i}}	\newcommand{\tdj}{\tilde{j}}
\newcommand{\tdk}{\tilde{k}}	\newcommand{\tdl}{\tilde{l}}
\newcommand{\tdm}{\tilde{m}}	\newcommand{\tdn}{\tilde{n}}
\newcommand{\tdo}{\tilde{o}}	\newcommand{\tdp}{\tilde{p}}
\newcommand{\tdq}{\tilde{q}}	\newcommand{\tdr}{\tilde{r}}
\newcommand{\tds}{\tilde{s}}	\newcommand{\tdt}{\tilde{t}}
\newcommand{\tdu}{\tilde{u}}	\newcommand{\tdv}{\tilde{v}}
\newcommand{\tdw}{\tilde{w}}	\newcommand{\tdx}{\tilde{x}}
\newcommand{\tdy}{\tilde{y}}	\newcommand{\tdz}{\tilde{z}}

%---------------------------------------
% Vec
%---------------------------------------

%Captital Letters
\newcommand{\vcA}{\vec{A}}	\newcommand{\vcB}{\vec{B}}
\newcommand{\vcC}{\vec{C}}	\newcommand{\vcD}{\vec{D}}
\newcommand{\vcE}{\vec{E}}	\newcommand{\vcF}{\vec{F}}
\newcommand{\vcG}{\vec{G}}	\newcommand{\vcH}{\vec{H}}
\newcommand{\vcI}{\vec{I}}	\newcommand{\vcJ}{\vec{J}}
\newcommand{\vcK}{\vec{K}}	\newcommand{\vcL}{\vec{L}}
\newcommand{\vcM}{\vec{M}}	\newcommand{\vcN}{\vec{N}}
\newcommand{\vcO}{\vec{O}}	\newcommand{\vcP}{\vec{P}}
\newcommand{\vcQ}{\vec{Q}}	\newcommand{\vcR}{\vec{R}}
\newcommand{\vcS}{\vec{S}}	\newcommand{\vcT}{\vec{T}}
\newcommand{\vcU}{\vec{U}}	\newcommand{\vcV}{\vec{V}}
\newcommand{\vcW}{\vec{W}}	\newcommand{\vcX}{\vec{X}}
\newcommand{\vcY}{\vec{Y}}	\newcommand{\vcZ}{\vec{Z}}
%Small Letters
\newcommand{\vca}{\vec{a}}	\newcommand{\vcb}{\vec{b}}
\newcommand{\vcc}{\vec{c}}	\newcommand{\vcd}{\vec{d}}
\newcommand{\vce}{\vec{e}}	\newcommand{\vcf}{\vec{f}}
\newcommand{\vcg}{\vec{g}}	\newcommand{\vch}{\vec{h}}
\newcommand{\vci}{\vec{i}}	\newcommand{\vcj}{\vec{j}}
\newcommand{\vck}{\vec{k}}	\newcommand{\vcl}{\vec{l}}
\newcommand{\vcm}{\vec{m}}	\newcommand{\vcn}{\vec{n}}
\newcommand{\vco}{\vec{o}}	\newcommand{\vcp}{\vec{p}}
\newcommand{\vcq}{\vec{q}}	\newcommand{\vcr}{\vec{r}}
\newcommand{\vcs}{\vec{s}}	\newcommand{\vct}{\vec{t}}
\newcommand{\vcu}{\vec{u}}	\newcommand{\vcv}{\vec{v}}
\newcommand{\vcw}{\vec{w}}	\newcommand{\vcx}{\vec{x}}
\newcommand{\vcy}{\vec{y}}	\newcommand{\vcz}{\vec{z}}

%---------------------------------------
% Greek Letters:-
%---------------------------------------
\newcommand{\eps}{\epsilon}
\newcommand{\veps}{\varepsilon}
\newcommand{\lm}{\lambda}
\newcommand{\Lm}{\Lambda}
\newcommand{\gm}{\gamma}
\newcommand{\Gm}{\Gamma}
\newcommand{\vph}{\varphi}
\newcommand{\ph}{\phi}
\newcommand{\om}{\omega}
\newcommand{\Om}{\Omega}


%%%%%%%%%%%%%%%%%%%%%%%%%%%%%%%%%%%%%%%% MACROS %%%%%%%%%%%%%%%%%%%%%%%%%%%%%%%%%%%%%%%%

%%%%%%%%%%%%%%% Link With an Icon %%%%%%%%%%%%%%% 
\newcommand{\link}[1]{
    \href{#1}{\faIcon{link}}
}

%%%%%%%%%%%%%%% Name Template %%%%%%%%%%%%%%% 
\newcommand{\name}[2]{
    % Name
    \Huge % Font size
    \raggedright \textbf{#1} \par

    \vspace*{0.3cm}
    
    % Profession
    \Large % Font size
    \raggedright #2 \par
    \normalsize \normalfont
}

%%%%%%%%%%%%%%% Contact Details %%%%%%%%%%%%%%%
\newcommand{\info}[2]{
    \faIcon{#2} \hspace{0.2em} #1
}

%%%%%%%%%%%%%%% Email %%%%%%%%%%%%%%%
\newcommand{\email}[1]{
    \info{#1}{envelope}
}

%%%%%%%%%%%%%%% Phone Number %%%%%%%%%%%%%%%
\newcommand{\phone}[1]{
    \info{#1}{mobile-alt}
}

%%%%%%%%%%%%%%% Address %%%%%%%%%%%%%%%
\newcommand{\address}[1]{
    \info{#1}{map-marker-alt}
}

%%%%%%%%%%%%%%% GitHub %%%%%%%%%%%%%%%
\newcommand{\github}[2]{
    \info{\href{#1}{\underline{#2}}}{github}
}

%%%%%%%%%%%%%%% LinkedIn %%%%%%%%%%%%%%%
\newcommand{\linkedin}[2]{
    \info{\href{#1}{\underline{#2}}}{linkedin}
}

%%%%%%%%%%%%%%% ResearchGate %%%%%%%%%%%%%%%
\newcommand{\researchgate}[2]{
    \info{\href{#1}{\underline{#2}}}{researchgate}
}

%%%\newcommand*{\Researchgate}[1]{\sociallink{\researchgatesocialsymbol}{http://www.#1}{#1}}

%%%%%%%%%%%%%%% Website %%%%%%%%%%%%%%%
\newcommand{\website}[1]{
    \info{#1}{link}
}

%%%%%%%%%%%%%%% Draw Skill Bars %%%%%%%%%%%%%%% 
\newcommand{\drawskillbars}[1]{
    \begin{tikzpicture}
        % Draw 5 gray bars
        \foreach \i in {0, 1, 2, 3, 4}{
            \fill[lightgray] (\i * 0.7 + 0.2 *\i,0) rectangle (0.7 + \i * 0.7 + \i * 0.2,0.1);
        }
        
        % Draw number of black bars depending on the skill level
        \foreach \i in {#1}{
            \fill[blue!40] (\i * 0.7 + 0.2 *\i,0) rectangle (0.7 + \i * 0.7 + \i * 0.2,0.1);
            %\fill[title] (\i * 0.7 + 0.2 *\i,0) rectangle (0.7 + \i * 0.7 + \i * 0.2,0.1);
        }
    \end{tikzpicture} \par
}
    
%%%%%%%%%%%%%%% Skills %%%%%%%%%%%%%%%
\newcommand{\skill}[3]{
    % Name of the skill
    \large
    \noindent \hangafter=0
    \adjustbox{valign=t}{\begin{minipage}{0.72\textwidth}
        \large \noindent \hangafter=0
        % Name of the skill
        \textmd{#1} 
        \normalsize \par 
        \vspace{1em}
         % Description
        \noindent \small \color{subtitle} \parbox{1\linewidth}{\textsl{#3}} \par
        \normalsize \par
        \end{minipage}}
    \adjustbox{valign=t}{\begin{minipage}{0.2\textwidth}
        % Skill bars
        \large \hangafter=0
        %\noindent 
        \drawskillbars{#2}
        \end{minipage}}
    \normalsize \par 
    % Skill bars
    %%\drawskillbars{#2}
    %%\vspace{0.5em}
    
    \vspace{1.0em}
    \normalsize \color{black} \par
}

%%%%%%%%%%%%%%% Software %%%%%%%%%%%%%%%
\newcommand{\soft}[2]{
    \adjustbox{valign=t}{\begin{minipage}{0.40\textwidth}
        \large \noindent \hangafter=0
        % Name of the skill
        \textmd{#1} 
        \normalsize \par 
        \vspace{1em}
        \end{minipage}}
    \adjustbox{valign=t}{\begin{minipage}{0.5\textwidth}
        % Skill bars
        \large \noindent \hangafter=0
        \drawskillbars{#2}
        \end{minipage}}
    \normalsize \par 
    \vspace{1em}
}

%%%%%%%%%%%%%%% Personal details %%%%%%%%%%%%%%%
\newcommand{\details}[2]{
    % Name of the language
    \large
    \noindent \hangafter=0 \color{black}
    \adjustbox{valign=t}{\parbox{0.27\linewidth}{#1}}  \adjustbox{valign=t}{\parbox{0.55\linewidth}{#2}} \par
    \vspace{.3em}
    \normalsize \color{black} \par
 }

%%%%%%%%%%%%%%% Language %%%%%%%%%%%%%%%
\newcommand{\lan}[2]{
    % Name of the language
    \large
    \noindent \hangafter=0 \color{black}
    \parbox{0.3\linewidth}{\textmd{#1}}   \color{subtitle} \parbox{0.4\linewidth}{\textsl{#2}} \par
    %\large English \color{subtitle} \textit{Advanced} 
    %\normalsize \par 
    % Knowledge level
    %\noindent \small \color{subtitle} \parbox{1\linewidth}{\textsl{#2}} \par
    \vspace{1.0em}
    \normalsize \color{black} \par
 }

%%%%%%%%%%%%%%% Education %%%%%%%%%%%%%%%
\newcommand{\education}[4]{
    % Name of the studies
    \noindent \large \parbox{.65\linewidth}{\textbf{#1}}
    % Duration in a Box
    \hfill \small
    \tcbox[enhanced,nobeforeafter,box align=base,colback=title,colframe=title,size=fbox,arc=0mm, valign=bottom]{{\textbf{#2}}} \par
    \vspace{0.3em}
    % School Name 
    \normalsize
    \noindent \color{subtitle} \parbox{.9\linewidth}{\textsl{#3}} \par
    % Description
    \normalsize \color{black}
    \vspace*{0.3em}
    \small #4 
    \normalsize \par
    \vspace*{0.5em}
}

%%%%%%%%%%%%%%% Work Experience %%%%%%%%%%%%%%%
\newcommand{\work}[4]{
    % Name of the Job
    \noindent \large \parbox{.65\linewidth}{\textbf{#1}}
    % Duration in a Box 
    \hfill \small
    \tcbox[enhanced,box align=base,nobeforeafter,colback=title,colframe=title,size=fbox,arc=0mm]{\textbf{#2}} \par
    \vspace{0.3em}
    % Name of the Employer
    \noindent \large \color{subtitle} \parbox{.9\linewidth}{\textsl{#3}} \par
    % Description of the job
    \vspace*{0.3em} \color{black}
    \small #4 
    \normalsize \par
}

%%%%%%%%%%%%%%% Teaching %%%%%%%%%%%%%%%
\newcommand{\teaching}[3]{
    % What, Topic and Who/Where/when
    \noindent \adjustbox{valign=t}{\parbox{.99\linewidth}{\text{#1} \text{#2} \textbf{#3}}} 
    \vspace{0.5em}
    \vspace*{1em} 
}

%%%%%%%%%%%%%%% Publications %%%%%%%%%%%%%%%
\newcommand{\publ}[4]{
    % Authors, Title and journal
    \noindent \parbox{.99\linewidth}{\textsl{#3}. \textbf{#1} \textsl{#2} \link{#4}}
    \vspace{0.5em}
    \vspace*{1em} \color{black}
    }

%%%%%%%%%%%%%%% Talks %%%%%%%%%%%%%%%
\newcommand{\talk}[3]{
    % Authors, Title and journal
    \noindent \parbox{.99\linewidth}{\textsl{#3}. \textbf{#1} \textsl{#2}}
    \vspace{0.5em}
    \vspace*{1em} \color{black}
}

%%%%%%%%%%%%%%% Events %%%%%%%%%%%%%%%
\newcommand{\event}[3]{
    \noindent \parbox{.99\linewidth}{\textbf{#1} \textsl{#3} \textsl{#2}}
    \vspace{0.5em}
    \vspace*{1em} \color{black}
}

\setlength{\parindent}{0pt}

%%%%%%%%%%%%%%%%%%%%%%%%%%%%%%%%%%%%%%%%%%%%%%%%%%%%%%%%%%%%%%%%%%%%%%%%%%%%%%%%%%%%%%%%%%%%%%%%%%%%%%%%%%%%%%%%%%%%%%%%%%%%%%%%%%%%%%%%

\begin{document}

%%%%%%%%%%%%%%%%%%%%%%%%%%%%%%%%%%%%%%%%%%%%%%%%%%%%%%%%%%%%%%%%%%%%%%%%%%%%%%%%%%%%%%%%%%%%%%%%%%%%%%%%%%%%%%%%%%%%%%%%%%%%%%%%%%%%%%%%

\textsf{\noindent \large\textbf{Soham Chatterjee} \hfill \textbf{Assignment - 4}\\
	Email: \href{sohamc@cmi.ac.in}{sohamc@cmi.ac.in} \hfill Roll: BMC202175\\
	\normalsize Course: Analysis 2 \hfill Date: May 15, 2022 \\
	\noindent\rule{7in}{2.8pt}}

%%%%%%%%%%%%%%%%%%%%%%%%%%%%%%%%%%%%%%%%%%%%%%%%%%%%%%%%%%%%%%%%%%%%%%%%%%%%%%%%%%%%%%%%%%%%%%%%%%%%%%%%%%%%%%%%%%%%%%%%%%%%%%%%%%%%%%%%
% Problem 1
%%%%%%%%%%%%%%%%%%%%%%%%%%%%%%%%%%%%%%%%%%%%%%%%%%%%%%%%%%%%%%%%%%%%%%%%%%%%%%%%%%%%%%%%%%%%%%%%%%%%%%%%%%%%%%%%%%%%%%%%%%%%%%%%%%%%%%%%

\begin{problem}{%problem statement
Rudin Chapt. 9 Problem 18
}{p1% problem reference text
}
%Problem	
Answer analogous questions for the mapping defined by
$$
	u=x^{2}-y^{2}, \quad v=2 x y
$$
\end{problem}

\solve{
	%Solution
	\begin{enumerate}[label=(\alph*)]
		\item Let the mapping defined by $f=(f_1,f_2)$ where $f_1(x,y)=x^2-y^2$ and $f_2(x,y)=2xy$. Hence $f(x,y)=(u,v)$. Then the range of $f$ is the whole $\bbR^2$ plane. For any point $(u,v)\in\bbR^2$ not equal to the origin $f$ maps two distinct maps to $(u,v)$\[
			      \begin{array}{ccl}
				      \lt( \sqrt{\frac{\sqrt{u^2+v^2}+u}{2}},\sqrt{\frac{\sqrt{u^2+v^2}-u}{2}} \rt)  & \lt( -\sqrt{\frac{\sqrt{u^2+v^2}+u}{2}},-\sqrt{\frac{\sqrt{u^2+v^2}-u}{2}} \rt) & \text{when }v\text{ is positive} \\
				      \lt( \sqrt{\frac{\sqrt{u^2+v^2}+u}{2}},-\sqrt{\frac{\sqrt{u^2+v^2}-u}{2}} \rt) & \lt( -\sqrt{\frac{-\sqrt{u^2+v^2}+u}{2}},\sqrt{\frac{\sqrt{u^2+v^2}-u}{2}} \rt) & \text{when }v\text{ is negative}
			      \end{array}
		      \]

		\item The matrix of $f'(x,y)$ is \[ f'(x,y) =\lt[ \begin{matrix}
					      2x & -2y \\ 2y & 2x
				      \end{matrix}\rt] \] Therefore the jacobian of $f$ is $4(x^2+y^2)$. Hence at any point $(u,v)\in \bbR^2$ except the orijin $\mcJ$ is nonzero. Hence if we exclude the origin in both planes, $f$ becomes locally one-one and globally two-one.
		\item Let $r=\sqrt{u^2+v^2}$ which is the distance from origin to the point $(u,v)$. Let $\textbf{a}=\lt(0,\frac{\pi}{3}\rt)$, then $f$ maps the pointto  $\textbf{b}=\lt(-\frac{\pi^2}{9},0\rt)$. Hence locally $f$ has the inverse function, $g(u,v)=\lt(\sqrt{\frac{r+u}{2}},\sqrt{\frac{w-u}{2}} \rt)$. Hence $$ g'(u,v)=\lt[ \begin{matrix}
					      \frac{u+r}{4r}\sqrt{\frac{2}{u+r}} & \frac{v}{4r}\sqrt{\frac{2}{u+r}} \\
					      \frac{r-u}{4r}\sqrt{\frac{2}{r-u}} & \frac{v}{4r}\sqrt{\frac{2}{r-u}}
				      \end{matrix}\rt]$$  Therefore $$g(u,0)=\lt[ \begin{matrix}
					      \sqrt{\frac{u+|u|}{2}} & \sqrt{\frac{|u|-u}{2}} \end{matrix}\rt]=\begin{cases}
				      (\sqrt{u},0) & \text{ when } u\geq 0\\ (0,\sqrt{-u}) & \text{ when }u<0\end{cases}$$Hence
		      \begin{align*}
			      \lt.\del{g_1(u,v)}{u}\rt|_{\lt( -\frac{\pi^2}{9}, 0\rt)} & =\lim_{h\to 0}\frac{g_1\lt(-\frac{\pi^2}{9}+h,0\rt) - g_1\lt(-\frac{\pi^2}{9},0\rt)}{h}                                                                                \\
			                                                               & = \lim_{h\to 0}\frac{0-0}{h}=0                                                                                                                                         \\
			      \lt.\del{g_1(u,v)}{v}\rt|_{\lt( -\frac{\pi^2}{9}, 0\rt)} & =\lim_{k\to 0 } \frac{g_1\lt(-\frac{\pi^2}{9},k\rt) - g_1\lt(-\frac{\pi^2}{9},0\rt)}{k}                                                                                \\
			                                                               & = \lim_{k\to 0}\frac1k \sqrt{\frac{\sqrt{\frac{\pi^4}{81}+k^2}-\frac{\pi^2}{9}}{2}} =\lim_{k\to 0}\sqrt{ \frac{\sqrt{a^2+k^2}-a}{2}}                                   \\
			                                                               & = \frac1{\sqrt{2}}\lim_{\theta\to 0}\frac{\sqrt{\sqrt{a^2+a^2\tan^2\theta}-a }}{a\tan\theta}\qquad\lt[\text{Assume }k=a\tan\theta,a=\frac{\pi^2}{9}\rt]                \\
			                                                               & =\frac1{\sqrt{2a}}\lim_{\theta\to 0}\frac{\sqrt{\sec\theta-1}}{\tan\theta}=\frac1{\sqrt{2a}}\lim_{\theta\to 0}\sqrt{\cos\theta}\frac{\sqrt{1-\cos\theta}}{\sin\theta }
		      \end{align*}\begin{align*}
			      \qquad\qquad\quad                                        & = \frac1{\sqrt{2a}}\lim_{\theta\to 0} \sqrt{\cos\theta}\sqrt{\frac{2\sin^2\frac{\theta}{2}}{4\sin^2\frac{\theta}2\cos^2\frac{\theta}2}}=\frac1{\sqrt{2a}}\lim_{\theta\to 0} \sqrt{\cos\theta}\sqrt{\frac1{2\cos^2\frac{\theta}2}  }             \\
			                                                               & = \frac1{2\sqrt{a}}=\frac3{2\pi}                                                                                                                                                                                                                \\
			      \lt.\del{g_2(u,v)}{u}\rt|_{\lt( -\frac{\pi^2}{9}, 0\rt)} & =\lim_{h\to 0}\frac{g_2\lt(-\frac{\pi^2}{9}+h,0\rt) - g_2\lt(-\frac{\pi^2}{9},0\rt)}{h}                                                                                                                                                         \\
			                                                               & = \lim_{h\to 0} \frac{\sqrt{\frac{\pi^2}{9}-h}-\frac{\pi}{3}}{h}=\frac{d}{dx}\lt.\lt( \sqrt{\frac{\pi^2}{9}-x}\rt)\rt|_{x=0}                                                                                                                    \\
			                                                               & = \lt.\frac12 \frac{-1}{\sqrt{\frac{\pi^2}{9}-x}}\rt|_{x=0}=-\frac{3}{2\pi}                                                                                                                                                                     \\
			      \lt.\del{g_2(u,v)}{v}\rt|_{\lt( -\frac{\pi^2}{9}, 0\rt)} & =  \lim_{k\to 0 } \frac{g_2\lt(-\frac{\pi^2}{9},k\rt) - g_2\lt(-\frac{\pi^2}{9},0\rt)}{k}                                                                                                                                                       \\
			                                                               & = \lim_{k\to 0 } \frac{ \sqrt{ \frac{ \sqrt{ \frac{pi^4}{81}+k^2 } +\frac{\pi^2}{9} }{2} }-\frac{\pi}{3} }{k}                                                                                                                                   \\
			                                                               & = \lim_{\theta\to 0 } \frac{ \frac1{\sqrt{2}} \sqrt{a\sec\theta +a}-\sqrt{a} }{a\tan\theta}\lt[\text{ Assume }\frac{\pi^2}{9}=a, k=a\tan\theta \rt]                                                                                             \\
			                                                               & = \frac1{\sqrt{a}}\lim_{\theta\to 0 } \frac{ \frac1{\sqrt{2}}\sqrt{\frac{1+\cos\theta}{\cos\theta}}-1 }{\tan\theta}=\frac1{\sqrt{a}}\lim_{\theta\to 0 } \frac{\sqrt{2}\sqrt{\frac{1+\cos\theta}{\cos\theta}}-2 }{2\tan\theta}                   \\
			                                                               & = \frac1{\sqrt{a}}\lim_{\theta\to 0 } \frac{\cos\theta\lt(\sqrt{2}\sqrt{\frac{1+\cos\theta}{\cos\theta}}-2\rt)\lt(\sqrt{2}\sqrt{\frac{1+\cos\theta}{\cos\theta}}+2 \rt) }{2\sin\theta\lt(\sqrt{2}\sqrt{\frac{1+\cos\theta}{\cos\theta}}+2 \rt)} \\
			                                                               & = \frac1{\sqrt{a}}\lim_{\theta\to 0 } \frac{ \cos\theta \lt( 2\frac{1+\cos\theta}{\cos\theta}-4 \rt)}{2\sin\theta\lt(\sqrt{2}\sqrt{\frac{1+\cos\theta}{\cos\theta}}+2 \rt)}                                                                     \\
			                                                               & = \frac1{\sqrt{a}}\lim_{\theta\to 0 } \frac{2(1-\cos\theta)}{2\sin\theta\lt(\sqrt{2}\sqrt{\frac{1+\cos\theta}{\cos\theta}}+2 \rt)}                                                                                                              \\
			                                                               & = \frac1{\sqrt{a}}\lim_{\theta\to0} \frac{2\sin^2\frac{\theta}{2}}{2\sin\frac{\theta}{2}\cos\frac{\theta}{2}\lt(\sqrt{2}\sqrt{\frac{1+\cos\theta}{\cos\theta}}+2 \rt)}                                                                          \\
			                                                               & = \frac1{\sqrt{a}}\lim_{\theta\to0} \frac{\sin\frac{\theta}{2}}{\cos\frac{\theta}{2}\lt(\sqrt{2}\sqrt{\frac{1+\cos\theta}{\cos\theta}}+2 \rt)}=0
		      \end{align*}Therefore $$g'\lt(-\frac{\pi^2}{9},0\rt)=\lt[ \begin{matrix}
					      0 & \frac{3}{2\pi} \\ \frac{3}{2\pi} & 0
				      \end{matrix} \rt]$$Now we previously determined $f'(x,y)=\lt[ \begin{matrix}
					      2x & -2y \\ 2y & 2x
				      \end{matrix} \rt]$Therefore \[
			      f'(\textbf{a})\circ g'(\textbf{b})=\lt[ \begin{matrix}
					      0 & -\frac{2\pi}{3} \\ \frac{2\pi}{3} & 0 \end{matrix}\rt] \lt[ \begin{matrix}
					      0 & \frac{3}{2\pi} \\ -\frac{3}{2\pi} & 0 \end{matrix}\rt] = \lt[ \begin{matrix}
					      1 & 0 \\ 0 & 1\end{matrix}\rt]
		      \]
		\item Thepoints parallel to $x-$axis are of the form $(x,c)$ and the points of the form parallel to $y-$axix are $(c,y)$. Image of $f$ of the lines parallel to $x-$ axis and $y-$axis are $$f(x,c)=\lt[ \begin{matrix}2x & -2c\\ 2c & 2x\end{matrix}\rt]\qquad f(c,y)=\lt[ \begin{matrix}2c & -2y\\ 2y& 2c\end{matrix}\rt]$$
	\end{enumerate}
}


%%%%%%%%%%%%%%%%%%%%%%%%%%%%%%%%%%%%%%%%%%%%%%%%%%%%%%%%%%%%%%%%%%%%%%%%%
% Problem 2
%%%%%%%%%%%%%%%%%%%%%%%%%%%%%%%%%%%%%%%%%%%%%%%%%%%%%%%%%%%%%%%%%%%%%%%%%

\begin{problem}{%problem statement
Rudin Chapt. 9 Problem 19
}{p2% problem reference text
}
%Problem		
Show that the system of equations
$$
	\begin{array}{r}
		3 x+y-z+u^{2}=0 \\
		x-y+2 z+u=0     \\
		2 x+2 y-3 z+2 u=0
	\end{array}
$$
can be solved for $x, y, u$ in terms of $z$; for $x, z, u$ in terms of $y ;$ for $y, z, u$ in terms of $x$; but not for $x, y, z$ in terms of $u$.
\end{problem}

\solve{
%Solution
Let $f$ be the map from $\bbR^4$ to $\bbR^3$ such that $$f(x,y,z,u)=(3 x+y-z+u^{2},x-y+2 z+u,2 x+2 y-3 z+2 u)=(f_1(x,y,z,u),f_2(x,y,z,u),f_3(x,y,z,u))$$where each $f_i:\bbR^4\to\bbR$ function. Now $f(0,0,0,0)=(0,0,0)$. Now the matrix of $f'(x,y,z,u)$ is $$f'(x,y,z,u)=\lt[ \begin{matrix}
			3 & 1 & -1 & 2u \\ 1 & -1 & 2 & 1 \\ 2 & 2 & -3 & 2
		\end{matrix} \rt]$$

The determinant of the part $x,y,u$ is $8u-12\neq 0$ near $\textbf{0}$. So by implicit function theorem there exists a solution of $f(x(z),y(z),z,u(z))=\textbf{0}$ near \textbf{0}.


The determinant of the part $x,z,u$ is $21-14u\neq 0$ near $\textbf{0}$. So by implicit function theorem there exists a solution of $f(x(y),y,z(y),u(y))=\textbf{0}$ near \textbf{0}.

The determinant of the part $y,z,u$ is $3-2u\neq 0$ near $\textbf{0}$. So by implicit function theorem there exists a solution of $f(x,y(x),z(x),u(x))=\textbf{0}$ near \textbf{0}.

But in case of $x,y,z$ the determinant is 0. Therefore there exists infinite solutions and they does not depend on $u$. Hence $x,y,z$ can not be expressed in terms of $u$
}


%%%%%%%%%%%%%%%%%%%%%%%%%%%%%%%%%%%%%%%%%%%%%%%%%%%%%%%%%%%%%%%%%%%%%%%%%
% Problem 3
%%%%%%%%%%%%%%%%%%%%%%%%%%%%%%%%%%%%%%%%%%%%%%%%%%%%%%%%%%%%%%%%%%%%%%%%%

\begin{problem}{%problem statement
Rudin Chapt. 9 Problem 23
}{p3% problem reference text
}
%Problem		
Define $f$ in $R^{3}$ by
$$
	f\left(x, y_{1}, y_{2}\right)=x^{2} y_{1}+e^{x}+y_{2} \text {. }
$$
Show that $f(0,1,-1)=0,\left(D_{1} f\right)(0,1,-1) \neq 0$, and that there exists therefore a differentiable function $g$ in some neighborhood of $(1,-1)$ in $R^{2}$, such that $g(1,-1)=0$ and
$$
	f\left(g\left(y_{1}, y_{2}\right), y_{1}, y_{2}\right)=0
$$Find $(D_1g)(1,-1)$ and $(D_2g)(1,-1)$.
\end{problem}

\solve{
%Solution
$$f(0,1,-1)=0^2\times 1+e^0-1=0$$Hence $f(0,1,-1)=0$. Now $D_1f(x,y_1,y_2)=2xy_1+e^x$. Then $$D_1f(0,1,-1)=2\times 0\times 1+e^0\neq 0$$Hence by Implicit Function Theorem there exists a $C^1$ function $g:\bbR^2\to \bbR$ and a neighbourhood near $(1,-1)$ such that for any point $(y_1,y_2)\in U$ $f(g(y_1,y_2),y_1,y_2)=0$ and $g(1,-1)=0$

Let $$h(y_1,y_2)=f(g(y_1y_2),y_1,y_2)=g^2(y_1,y_2)y_1+e^{g(y_1,y_2)}+y_2=0$$where $(y_1,y_2)\in U$. Therefore $$0=D_1h(y_1,y_2)=\lt(2g(y_1,y_2)y_1+e^{g(y_1,y_2)}\rt)D_1g(y_1,y_2)+g(y_1,y_2)^2 $$ Similarly $$0=D_2h(y_1,y_2)=\lt(2g(y_1,y_2)y_1+e^{g(y_1,y_2)}\rt)D_2g(y_1,y_2)+1$$Putting $y_1=1,y_2=-1,g(y_1,g_2)=0$ we get $$D_1g(1,-1)=0\qquad D_2g(1.-1)=-1$$
}


%%%%%%%%%%%%%%%%%%%%%%%%%%%%%%%%%%%%%%%%%%%%%%%%%%%%%%%%%%%%%%%%%%%%%%%%%
% Problem 4
%%%%%%%%%%%%%%%%%%%%%%%%%%%%%%%%%%%%%%%%%%%%%%%%%%%%%%%%%%%%%%%%%%%%%%%%%

\begin{problem}{%problem statement
}{p4% problem reference text
}
%Problem		
Find max/min of $x+y+z$ subject to $x^2-y^2=1$ and $2x+z=1.$
\end{problem}

\solve{
%Solution
The constraint $x^2-y^2=1$ is a hyperbolic cylinder. Hence it can be parametrized as $(\theta, z)\to (\pm \cosh \theta,\sinh\theta,z)$. Now given the contraint $2x+z=1\iff z=1-2x$ the points can be parametrized as $h_{\pm}:\theta \to (\pm\cosh\theta,\sinh\theta, 1\mp 2\cosh\theta)$. We define function $g_{\pm}=f\circ h_{\pm}$. Hence $$g(\theta)=\pm \cosh\theta +\sin\theta+1+\mp\cosh\theta=1+\sinh\theta\mp\cosh\theta$$Hence $g_{\pm}'(\theta)=\cosh\theta-\mp\sinh\theta=e^{\mp\theta}$ which has no extrema for all $\theta\in \bbR$. Hence $x+y+z$ has no extrema under the constraints $x^2-y^2=1$ and $2x+z=1$
}


%%%%%%%%%%%%%%%%%%%%%%%%%%%%%%%%%%%%%%%%%%%%%%%%%%%%%%%%%%%%%%%%%%%%%%%%%
% Problem 5
%%%%%%%%%%%%%%%%%%%%%%%%%%%%%%%%%%%%%%%%%%%%%%%%%%%%%%%%%%%%%%%%%%%%%%%%%

\begin{problem}{%problem statement
}{p5% problem reference text
}
%Problem		
Show that tangent vectors can be realized as velocity vectors of curves. More precisely, let $U$ be an open set in $\mathbb{R}^{n}$. Let $g$ be a $C^{1}$ map $U \rightarrow \mathbb{R}^{m}$. Let $c$ a point in the image of $g$, $M=g^{-1}(c)$ and $p \in M$ such that $g^{\prime}(p)$ is surjective. Recall that $T_{p} M=$ the kernel of $g^{\prime}(p)$ is called the tangent space of $M$ at $p$. Show that this tangent space is spanned by the velocity vectors of all $C^{1}$ paths $\gamma$ in $M$ based at $p$, i.e., by $\gamma^{\prime}(0)$, where $\gamma:(-\epsilon, \epsilon) \rightarrow M$ is a $C^{1}$ function with $\gamma(0)=p$.
\end{problem}

\solve{
%Solution
Let $p=(p_1,p_2,\dots,p_n)$ and $d=n-m$. Now since $g'(p)$ is surjective it spans $\bbR^m$. Hence the matrix of $g'(p)$ has $m$ linearly independent columns. Suppose the last $m$ columns are linearly independent. Let $A$ be the matrix of $g'(p)$ and $$A=[A_d\mid A_m]$$ where $A_d$ is a $d\times m$ and $A_m$ is a $m\times m$ matrix. Hence $A_m$ is invertible. Let $p=(P_d,P_m)$ where $$P_d=(p_1,p_2,\dots,p_d)\text{ and }P_m=(p_{d+1},p_{d+2},\dots, p_{m})$$By implicit function theorem there $\exs$ a open ball $V\subset \bbR^d$ containing $P_d$, an open ball $W\subset \bbR^m$ containing $P_m$ and a $C_1$ function $h:\bbR^d\to \bbR^m$ such that $h(x)=y$ and $g(x,y)=c$ and $h(P_d)=P_m$

Now let $v=(v_d,v_m)\in T_pM$ where $v_d\in \bbR^d$ and $v_m\in \bbR^m$. We define a function $a:\bbR\to \bbR^d$ such that $$a(t)=P_d+tv_d$$ and a function $c$ such that $$c(t)=(a(t),h(a(t)))$$ Since $V$ is an open ball we can bound $|t|$ by $\eps>0$ such that $c(-\eps,\eps)\subset V$.

Now $$c(0)=(a(0),h(a(0)))=(p_d,h(p_d))=(p_d,p_m)$$. Now we need to show $c'(0)=v$. Notice that $$c'(0)=\lt( a'(t),\lt.\frac{d}{dt}h(a(t))\rt|_{t=0} \rt)=\lt(v_d,\lt.\frac{d}{dt}h(a(t))\rt|_{t=0} \rt)$$. Now we will find $\lt.\frac{d}{dt}h(a(t))\rt|_{t=0}$.\begin{align*}
	\lt.\frac{d}{dt}h(a(t))\rt|_{t=0}  = h'(a(0))a'(0)=h'(P_d)v_d
\end{align*}Now by implicit theorem we also have $h'(P_d)=-A_m^{-1}A_d$. Hence \begin{multline*}
	h'(P_d)v_d=-A_m^{-1}A_dv_d=-A_m^{-1}(A_dv_d)=-A_m^{-1}([A_d\mid 0]v)=-A_m^{-1}(g'(p)v-[0\mid A_m]v) \\
	=-A_m^{-1}(-[0\mid A_m]v)=-A_m^{-1}(-A_mv_m)=v_m
\end{multline*}Therefore $$c'(0)=(v_d,v_m)=v$$Hence there exists a $C^1$ function $c:(-\eps,\eps)\to M$ such that $c(0)=p$ and $c'(0)=v$ where $v\in T_pM$
}


\end{document}
