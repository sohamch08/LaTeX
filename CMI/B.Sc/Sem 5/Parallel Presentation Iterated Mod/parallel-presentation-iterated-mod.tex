\documentclass[article,10pt]{beamer}% para se tiver muitas sections
%\documentclass[11pt,compress,xcolor=dvipsnames]{beamer}
%---------------------------------------------------------------------
% Color and themes
%---------------------------------------------------------------------
\definecolor{ipb}{rgb}{0.36, 0.54,0.66}
%\definecolor{royalazure}{rgb}{0.07, 0.04, 0.56}
\definecolor{royalazure}{RGB}{8,71,155}
\setbeamercolor{footline}{bg=ipb}
\setbeamercolor{frametitle}{bg=ipb,fg=white}
\setbeamercolor{title}{bg=ipb}
\setbeamerfont{frametitle}{size=\large}
\setbeamertemplate{bibliography item}[text]
\setbeamertemplate{caption}[numbered]
\setbeamertemplate{blocks}[rounded][shadow]
\setbeamercolor{palette primary}{use=structure,fg=white,bg=structure.fg}
\setbeamercolor{palette secondary}{use=structure,fg=white,bg=structure.fg!75!black}
\setbeamercolor{palette tertiary}{use=structure,fg=white,bg=structure.fg!50!black}
\setbeamercolor{palette quaternary}{fg=white,bg=structure.fg!50!black}
\setbeamercolor*{sidebar}{use=structure,bg=structure.fg}
\setbeamercolor{titlelike}{parent=palette primary}
\setbeamercolor{block title}{bg=ipb,fg=white}
\setbeamercolor*{block title example}{use={normal text,example text},bg=white,fg=ipb}
\setbeamercolor{fine separation line}{}
\setbeamercolor{item projected}{fg=black}
\setbeamercolor{palette sidebar primary}{use=normal text,fg=normal text.fg}
\setbeamercolor{palette sidebar quaternary}{use=structure,fg=structure.fg}
\setbeamercolor{palette sidebar secondary}{use=structure,fg=structure.fg}
\setbeamercolor{palette sidebar tertiary}{use=normal text,fg=normal text.fg}
\setbeamercolor{palette sidebar quaternary}{fg=ipb}
%\setbeamercolor{section in sidebar}{fg=brown}
%\setbeamercolor{section in sidebar shaded}{fg=grey}
\setbeamercolor{sidebar}{bg=ipb}
\setbeamercolor{sidebar}{parent=palette primary}
\setbeamercolor{structure}{fg=ipb}
%\setbeamercolor{subsection in sidebar}{fg=brown}
%\setbeamercolor{subsection in sidebar shaded}{fg=grey}
\setbeamercolor{section in head/foot}{fg=white,bg=royalazure}
%\setbeamercolor{subsection in head/foot}{fg=white,bg=royalazure}
%\usepackage[
%backend=biber,
%style=alphabetic,
%sorting=ynt
%]{biblatex}
%\addbibresource{bibliography.bib}
%\renewcommand*{\bibfont}{\normalfont\small}
\usetheme{Warsaw}
%---------------------------------------------------------------------
% Footline
%---------------------------------------------------------------------
\setbeamertemplate{footline}
 {\leavevmode
\hbox{
   \begin{beamercolorbox}[wd=0.49\paperwidth,ht=2.25ex,dp=1ex,leftskip=0.3cm]{author in head/foot}%
     \usebeamerfont{author in head/foot}\textcolor{white}{Soham Chatterjee}
   \end{beamercolorbox}%
   \begin{beamercolorbox}[wd=0.34\paperwidth,ht=2.25ex,dp=1ex,center]{title in head/foot}
     \usebeamerfont{title in head/foot}\textcolor{white}{\insertshorttitle}
   \end{beamercolorbox}%
   \begin{beamercolorbox}[wd=0.17\paperwidth,ht=2.25ex,dp=1ex,leftskip=0.3cm,rightskip=0.3cm]{title in head/foot}%
   \hfill\usebeamerfont{page number in head/foot}
   \insertframenumber{} / \textcolor{white}{\inserttotalframenumber}
   \end{beamercolorbox}}
}
%---------------------------------------------------------------------
% Packages
%---------------------------------------------------------------------
\usefonttheme[]{serif}
\usepackage{amsmath, latexsym, color, graphicx, amssymb, bm, here}
\usepackage{epsf, epsfig, pifont,tikz,subfigure}
\usepackage{graphics, calrsfs}
\usepackage{times}
\usepackage{fancybox,calc}
\usepackage{palatino,mathpazo,mathtools}
\usepackage{amsfonts}
\usepackage{wrapfig}
\usepackage{multicol}
\usepackage{sidecap}
\usepackage{academicons}
%\usepackage{pdfauthor}
%\usepackage{pdfcreator}
\usepackage{hyperref}
\usepackage{listings}
\usepackage{babel}
\usepackage[]{hyperref}
%---------------------------------------------------------------------
% Definir
%---------------------------------------------------------------------
\def\inst#1{\unskip$^{#1}$}
\def\orcidID#1{\unskip$^{[#1]}$}
\def\fnmsep{\unskip$^,$}
\def\email#1{{\tt#1}}

%---------------------------------------------------------------------
% Dados
%---------------------------------------------------------------------
\title{The Iterated Mod Problem}
%opçao para 1 autor
%\author{Primeiro Autor~\orcidID{a12345}}
%opçao para 2 autor
\author{\LARGE{Soham Chatterjee}}
%opcao para 3 autor
%\author{Primeiro Autor~\orcidID{a12345}\\
%        Segundo Autor~\orcidID{a12345}\\
 %       Terceiro Autor~\orcidID{a12345}
  %      }
%\institute{Instituto Politécnico de Bragança- Escola Superior de Tecnologia e Gestão\\
            %\vspace{0.3cm}
            %Licenciatura em Curso}
\institute{\large{Chennai Mathematical Institute}}
%\date{\vfill\scriptsize{\today}\\\vspace{0cm}\includegraphics[scale=0.4]{Imagens/logo.png}}
%---------------------------------------------------------------------
% Index
%---------------------------------------------------------------------
%\AtBeginSection[]
%{
%  \begin{frame}{Contents}
%    \tableofcontents[currentsection]
%  \end{frame}
%}
\definecolor{azuel}{rgb}{0.07, 0.04, 0.56}
	\definecolor{backcolour}{rgb}{0.95,0.95,0.92}
	\definecolor{codegreen}{rgb}{0,0.6,0}
	\definecolor{mygreen}{RGB}{28,172,0}
	\definecolor{mylilas}{RGB}{170,55,241}
	\definecolor{codegray}{rgb}{0.5,0.5,0.5}
	\definecolor{codepurple}{rgb}{0.58,0,0.82}
	\lstdefinestyle{list2}{language=Matlab,% lst
		basicstyle=\color{black},
		backgroundcolor=\color{white},
		breaklines=true,
		breakatwhitespace=false,
		keepspaces=true,
		morekeywords={matlab2tikz},
		showspaces=false, 
		showtabs=false,
		tabsize=2,
		rulecolor=\color{black},
		frame=single,
		keywordstyle={\scriptsize\color{blue}},
		morekeywords=[2]{1}, 
		keywordstyle=[2]{\scriptsize\color{black}},
		identifierstyle={\scriptsize\color{black}},%
		stringstyle={\scriptsize\color{mylilas}},
		commentstyle={\scriptsize\color{mygreens}},
		showstringspaces=false,
		numbers=left,%
		numberstyle={\scriptsize\color{black}},
		numbersep=10pt, 
		emph=[1]{for,end,break},emphstyle=[1]\scriptsize\color{blue}, 
	}
	%\renewcommand{\lstlistingname}{Alg.}
\newcommand{\bbF}{\mathbb{F}}
\begin{document}


\maketitle


\begin{frame}
    \frametitle{Contents}
    \tableofcontents
\end{frame}



\begin{frame}
    \frametitle{Introduction}
\begin{itemize}
    \item This paper is  Iterated Mod Problem by Karloff and Ruzzo \cite{iteratedmod}
    \item  Sequential algorithm for computing $gcd$ is based on Euclidean Algorithm
    $r_0=a$, $r_1=b$. Then $$r_2=r_0\bmod{r_1},\quad r_{3}=r_{1}\bmod{r_2}, \quad \cdots$$ $gcd$ is the last nonzero $r_i$.
    \item But parallel complexity of $gcd$ is poorly understood. Fastest parallel  algorithm takes $O\left(\frac{n}{\log n}\right)$ time \cite{gcdfastpar}
    \item $gcd$ for polynomials is in $NC$
    \item  The problem we will study related to the $gcd$ problem. It is given  integers or polynomials $x, m_n,m_{n-1},\dots, m_1$  find if $$((x\bmod{m_{n}})\bmod{m_{n-1}})\cdots)\bmod{m_{1}})=0$$
\end{itemize}
\end{frame}
%%%%%%%%%%%%%%%%%%%%%%%%%%%%%%Frame 4
\section{Iterated Integer Mod $(IIM)$ Problem}
\subsection{Introduction}
\begin{frame}
	\frametitle{Iterated Integer Mod Problem}
	\framesubtitle{Introduction}
%	\begin{itemize}
	\textbf{Problem:}
	
		 Given positive integers $x, m_n,m_{n-1},\dots, m_1$  find if $$((x\bmod{m_{n}})\bmod{m_{n-1}})\cdots)\bmod{m_{1}})=0$$
%		\item We will show this problem is $P$-complete.
%	\end{itemize}
%	First we have
\vspace{3mm}
		 \begin{theorem}
		Iterated Iinteger Mod $\in P$
	\end{theorem}
	For any 2 numbers $a$ and $b$, $a\bmod{b}$ is in $P$. Here we are doing $n$ iterated mods. So it still takes polynomial time. So $IIM\in P$.
\end{frame}
\subsection{Circuit Value Problem}

%%%%%%%%%%%%%%%%%%%%%%%%%%%%%%Frame 5

\begin{frame}
\frametitle{Circuit Value Problem}
%To show $IIM$ is $P$-complete. We will use this theorem.
\begin{theorem}[\cite{cvppcomp}]
	Circuit Value Problem is $P$-complete.
\end{theorem}
\vspace{5mm}


\begin{itemize}
	\item Enough to take $CVP$ for circuits with only $NAND$ gates, $NANDCVP$
\end{itemize}
$$\text{Gates$\in [G]$}$$ $$\text{ Input Variables$\coloneqq y_i,i\in[r]$, Input Bits$\coloneqq Y_i,i\in[r]$} $$ 



%\begin{itemize}
%	\item A $NANDCVP$ circuit the $r$ nodes $y_1,\dots, y_r$ of indegree 0 are the inputs 
%	\item The $G$ many nodes  with indegree 2 are the gates. The gates are numbers $1,\dots, G$. The gates are numbered in reverse topological order i.e. every edge is directed from a higher numbered gate to a lower numbered gate and the last gate with gate number 1 is the output with the edge going out of it is 0th edge.  
%	\item The edges $E=2G+1$ are numbered so that the two gates into gate $g$ are numbered $2g $ and $2g-1$.
%\end{itemize}
\end{frame}

\subsection{$NANDCVP\leq_l IIM$}

\begin{frame}

\frametitle{$NANDCVP\leq_l IIM$}
\framesubtitle{Log-Space Reduction}
Let $n=2G$. %The reduction from $NANDCVP$ to the integer iterated mod problem is as follows:
\begin{itemize}
	\item $x$ is $n+1$-bit integer whose $i$th bit is $Y_j$ if the $i$th edge is incident from the input $y_j$. Otherwise it is 1.
	\item $1\leq g\leq G$  $$m_{2g}=2^{2g}+2^{2g-1}+\sum_{\substack{j\text{th edge}\\ \text{out-edge from }g}}2^j\text{ and }m_{2g-1}=2^{2g-1}$$
\end{itemize}
%This reduction is a log-space reduction from $NANDCVP$ to Integer Iterated Mod problem. 
%\begin{itemize}
\vspace{2mm}
\textbf{Remark:} Here $m_{2g}$ and $m_{2g-1}$ simulate the gate $g$
%\end{itemize}
%The next theorem proves that the output gate of the $CVP$ instance is 0 iff $$((\cdots ((x\bmod{m_{2G}})\bmod{m_{2g-1}})\cdots))=0$$
\end{frame}
\begin{frame}[allowframebreaks]
\frametitle{$NANDCVP\leq_l IIM$}
\framesubtitle{Correctness}
\begin{theorem}
	Let $x_{G+1}=x$. And for all $1\leq g\leq G$ $x_g=((\cdots ((x\bmod{m_{2G}})\bmod{m_{2g-1}})\cdots\bmod{m_{2g}})\bmod{m_{2g-1}})=0$. Then:\begin{enumerate}
		\item For all $1\leq g\leq G+1$, $x_g\leq 2^{2g-1}$
		\item For all $1\leq g\leq G+1$, $0\leq j\leq 2g-1$ if the $j$th edge is an outgoing edge from an input node or from a gate $h$ such that $h\geq g$ then $x_g$'s $j$th bit is the value carried by $j$th edge 		otherwise 1
	\end{enumerate}
\end{theorem}
\framebreak


\textbf{Prove by downward induction:}
\vspace{3mm}

 Base Case ($g=G+1$): We have $x< 2^{2(G+1)-1}=2^{2G+1}=2^{n}$. True as $x$ is $n$-bit number. And second condition follows by constuction.
%And by constuction $i$th bit has $Y_j$ if edge $i$ is going out from $y_j$ otherwise 1. So base case follows
Let the theorem holds for all $g>k$. \framebreak

\textbf{Part (a)}:

\hspace{0.5cm} $x_k=(x_{k+1}\bmod{m_{2k}})\bmod{m_{2g-1}}$. $m_{2k-1}=2^{2k-1}$. So $x_k$ has $2k-1$ bits so $ x_k<2^{2k-1}$. So Part (a) is proved.
\framebreak

\textbf{Part (b):}
\begin{itemize}
	\item The only bits differ between $x_{k+1}$ and $x_k$ are the bits corresponding to edges incident on $k$th vertex (in and out). In $x_{k+1}$ the $j$th bits are 1 if $j$th edge going out from gate $k$.
	\item The $2k$ and $2k-1$th edges are in edges of gate $k$. So in $x_{k+1}$ the $(2k)$th and $(2k-1)$th bits are the value carried by the $(2k)$ and $(2k-1)$th edges. Two cases to consider:
\end{itemize}
\framebreak

\textbf{Both $(2k)$ and $(2k+1)$th bits are 1}: 

$m_{2k}\leq x_{k+1}< 2m_{2k}$. So $$(x_{k+1}\bmod{m_{m_{2k}}})\bmod m_{2k-1}=x_{k+1}-m_{2k}$$ 
So in $x_{2k}$ at output bits position of $m_{2k}$  the 1 in replaced  by  $0$\vspace{2mm}

\textbf{At least one of the bits is 0}: 

$$x_{k+1}<m_{2k}\implies x_{k+1}\bmod{m_{2k}}=x_{k+1}$$ So in $x_{2k}$ at output bits position of $m_{2k}$  has 1.

\end{frame}

\begin{frame}
	\frametitle{$IIM$ is $P$-complete}

$x_1< 2^{1}$ is the value carried by the $0$th edge, value of the $CVP$ instance. 
\begin{theorem}
%	\begin{center}
		$NANDCVP\leq_l $ Iterated Integer Mod
%	\end{center}
\end{theorem}
	 	\vspace{5mm}
	 	
	\begin{theorem}
		Integer Iterated Mod Problem is $P$-complete
	\end{theorem}
\end{frame}

\section{Super Increasing 0-1 Knapsack Problem}
\subsection{Introduction}
\begin{frame}
\frametitle{Super Increasing Knaspsack Problem (SIK)}
\framesubtitle{Introduction}
\begin{definition}[0-1 Knapsack Problem]
	Given an integer $w$ and a sequence of integers $w_1,w_2,\dots, w_n$ is there a sequence of $0-1$ valued variables $x_1,\dots x_n$ such that $w=\sum\limits_{i=1}^n x_iw_i$.
\end{definition}
\begin{itemize}
	\item 0-1 Knapsack Problem is known to be $NP$-complete. \cite{guidetonpc}
	\item A knapsack instance is called super increasing ($SIK$) if each weight $w_i$ is larger than the sum of the previous weights i.e. for all $2\leq i\leq n$ we have $w_i>\sum\limits_{j=1}^{i-1}w_j$
\end{itemize}
\end{frame}
\begin{frame}
	\frametitle{Super Increasing Knaspsack Problem (SIK)}
	\framesubtitle{Introduction}
\begin{theorem}
	Super Increasing Knaspsack Problem $\in P$
\end{theorem}\vspace{5mm}

Greedy strategy considering the $w_i'$ in decreasing order gives a linear time algorithm for solving super increasing knapsack problem.
\end{frame}

\subsection{Super Increasing Knaspsack Problem is $P$-complete}
\begin{frame}[allowframebreaks]
\frametitle{$SIK$ is $P$-complete}
%\begin{theorem}
%	Super Increasing Knapsack Problem is $P$-complete
%\end{theorem}
\begin{theorem}
	If $w_1,\dots, w_n$ are such that $\forall \ i\in [n-1]$ $\sum\limits_{k=1}^{i}w_k<w_{i+1}$ then there is a 0-1 sequence of variables $x_1,\dots, x_n$ such that $\sum\limits_{i=1}^nx_iw_i=w$ iff $$((\cdots((w\bmod{w_n})\bmod{w_{n-1}})\cdots )\bmod{w_2})\bmod{w_1}=1$$
\end{theorem}
\framebreak

\textbf{Observe:} The previous reduction the modulo numbers doesn't satisfy super increasing knapsack condition.
\vspace{5mm}

\begin{itemize}
	\item Need to find another reduction of $NANDCVP$ to $IIM$ where modulo numbers are super increasing to work with above theorem !!
\end{itemize}

\framebreak

\begin{itemize}
	\item Let $x$ is $n+1$-length base 4 number whose $i$th digit is $Y_j$ if the $i$th edge is incident from the input $y_j$. Otherwise it is 1.
	\item $1\leq g\leq G$  $$m_{2g}=4^{2g}+4^{2g-1}+\sum_{\substack{j\text{th edge}\\ \text{out-edge from }g}}4^j$$ $$m_{2g-0.5}=4^{2g}-4^{2g-1},\ \ m_{2g-1}=4^{2g-1}$$
\end{itemize}
\framebreak
Define for all $1\leq g\leq  G$,\\
 $x_g=(((\cdots (((x\bmod{m_{2G}})\bmod{m_{2G-0.5}})\bmod{m_{2G-1}})\cdots)\bmod{m_{2g}})\bmod{m_{2g-0.5}})\bmod{m_{2g-1}}=0$ and $x_{G+1}=x$.\vspace{5mm}
\begin{itemize}
	\item $x_g\leq 4^{2g-1}$ for all $1\leq g \leq G+1$
\end{itemize}
\framebreak


	\begin{theorem}
		For all $1\leq g\leq G+1$, $0\leq j\leq 2g-1$ if the $j$th edge is an outgoing edge from an input node or from a gate $h$ such that $h\geq g$ then $x_g$'s $j$th bit is the value carried by $j$th edge 		otherwise 1
	\end{theorem}
	\framebreak
	
%	\vspace{5mm}
\begin{itemize}	
	\item Prove by downward induction. Base case $g=G+1$ is true.
	\item $x_{k+1}$ and $x_k$ differs at the positions corresponding to the edges incident on $k$th vertex.
	\item $2k$ and $2k-1$th edges are in-edges of vertex $k$ so they are the values carried by $2k$ and $2k-1$th edges
	
	\end{itemize}
%	\item The theorem is true and with that we get $SIK$ is $P$-complete.
%\end{itemize}
\framebreak

 \textbf{If both of them 1}:
 
  $$4m_{2k}>x_{k+1}\geq m_{2k}\implies x_{k+1}\bmod{m_{2k}}=x_{k+1}-m_{2k}<4^{2k-1}$$  $$(x_{k+1}-m_{2k}\bmod{m_{2k-0.5}})\bmod{m_{2k-1}}=x_{k+1}-m_{2k}$$ In $x_k$ the positions where $m_{2k}$ has 1 will have 0.
  \framebreak
  
\textbf{If at least one of them 0}:

 $x_{k+1}\bmod{m_{2k}}=x_{k+1}$. In $x_k$ positions where $m_{2k}$ has 1 will have 1. $$x_{k+1}=a\times 4^{2k}+b\times 4^{2k-1}+c\text{ where }a,b\in \{0,1\}$$\begin{itemize}
	\item $a=1$, $b=0$: $$(x_{k+1}\bmod{m_{2k-0.5}})\bmod{m_{2k-1}}=1\times 4^{2k-1}+c\bmod{m_{2k-1}}=c$$
	\item $b=0,1$: $$(x_{k+1}\bmod{m_{2k-0.5}})\bmod{m_{2k-1}}=b\times 4^{2k-1}+c\bmod{m_{2k-1}}=c$$
\end{itemize}
\framebreak

After $m_1$, $x_1<2^1$ is the value carried by the 0th edge, the value of the $CVP$.

\begin{itemize}
	\item  \textbf{Notice}: The modulos  satisfies the super increasing knapsack problem. 
\end{itemize} 
 Since $$\sum\limits_{g=1}^{k}m_{2g}+m_{2g-0.5}+m_{2g-1}=\sum\limits_{g=1}^{k}m_{2g}+4^{2g}<4^{2k+1}=m_{2(k+1)-1}$$ 
 \framebreak
 
	\begin{enumerate}
		\item \hspace{1ex} Sum of weights till $m_{2k}$ is strictly $<m_{2(k+1)-1}$
		\item \begin{tabular}[t]{rl}
			& Sum of weights till $m_{2(k+1)-1}$ \\ 
			$=$ &(sum of weights till $m_{2k})+m_{2(k+1)-1}$ \\
		 $<$ & $2\times 4^{2(k+1)-1}<3\times 4^{2(k+1)-1}=m_{2(k+1)-0.5}$
		\end{tabular}
		
		\item \begin{tabular}[t]{rl}
			& Sum of weights till $m_{2(k+1)-0.5}$ \\
			$=$ & (sum of weights till $m_{2k})+m_{2(k+1)-1}+m_{2(k+1)-0.5}$\\
			 $<$ & $2\times 4^{2(k+1)-1}+3\times 4^{2(k+1)+1}$\\
			 $=$ & $ 4^{2(k+1)}+4^{2(k+1)-1}<m_{2(k+1)}$
		\end{tabular}
	\end{enumerate}


\framebreak

\begin{theorem}
	$NANDCVP\leq_l$ Super Increasing Knapsack
\end{theorem}\vspace{5mm}

\begin{theorem}
	Super Increasing Knapsack Problem is $P$-complete.
\end{theorem}
\end{frame}

\section{Polynomial Iterated Mod Problem (PIM)}
\subsection{Introduction}
%%%%%%%%%%%%%%%%%%%%%%%%%%%%%%Frame 15
\begin{frame}
\frametitle{Polynomial Iterated Mod Problem}
\framesubtitle{Introduction}
\begin{definition}[Polynomial Iterated Mod Problem]
	Given univariate polynomials $a(x)$, $b_1(x),\dots, b_n(x)$ over a field $\bbF$ compute the residue $((\cdots ((a(x)\bmod{b_1(x)})\bmod{b_2(x)})\cdots)\bmod{b_{n-1}(x)})\bmod{b_{n}(x)}$
\end{definition}
\begin{itemize}
	\item A polynomial mod can't test for two bits $$(10)_2\bmod{(11)_2}=(10)_2\text{ but }(x^2+0 x)\bmod{(x^2+x)}=0x^2-x$$
\end{itemize}
\begin{theorem}
	Polynomial Iterated Mod Problem is in $P$
\end{theorem}
\end{frame}

\subsection{Matrix Inversion}
\begin{frame}
\frametitle{Lower Triangular Matrix Inversion}



\begin{theorem}[\cite{hellwematinv},\cite{hellensurvey}]
	For any field $\bbF$, lower triangular matrix inversion is in $Arithmetic-NC$
\end{theorem}\vspace{5mm}

\begin{theorem}[\cite{fastparmatricgcd},\cite{paralwellrings}]
	Lower triangular matrix inversion is in $NC$ over finite fields and $\mathbb{Q}$
\end{theorem}

\end{frame}
\subsection{$PIM$ is in $NC$}
\begin{frame}[allowframebreaks]
	\frametitle{Reduction}
	Given $a(x),b_1(x),\dots, b_n(x)$ over $\bbF$.\\
	$b_0(x)=r_0(x)=a(x)$ and $d_i=\deg b_i(x)$ for all $0\leq i\leq n$.\\
	 Assume $d_0\geq d_1> \cdots > d_n$
	 \begin{align*}
	 	a(x) & = q_1(x)b_1(x)+r_1(x)\\
	 	& = q_1(x)b_1(x)+q_2(x)b_2(x)+r_2(x)\\
	 	& \qquad \vdots\\
	 	& = q_1(x)b_1(x)+\cdots+q_n(x)b_n(x)+r_n(x)
	 \end{align*}
	 $r_{i-1}(x)=q_i(x)\cdot b_i(x)+r_i(x)$ with 	 $	\deg r_i<\deg b_i=d_i$ or $r_i=0$ 
	\framebreak
	
	The coefficient of $x^j$ in $a(x), b_i(x), q_i(x), r_i(x)$ are $a_j$, $b_{i,j}$, $q_{i,j}$, $ r_{i,j}$. 
	\begin{itemize}
		\item $\deg q_1=d_0-d_1$, $\deg q_i\leq d_{i-1}-d_{i}-1$
		\item Compare the coefficients of $x^j$ in both direction.
		\item $(d_0+1)\times (d_0+1)$  matrix $M$. Denote the variable matrix for coefficients of $q_i$ and $r_n$ as $X$
	\end{itemize}\framebreak
	
 $d_0-i$-th entry of $MX$ is coefficient of  degree $i$. $d_{k}\leq i<d_{k-1}$. 
 
 $r_n(x)+\sum\limits_{i=K+1}^nq_i(x)b_i(x)$ doesn't take part in coefficient of $x^i$.
 
 $$i=d_k +  (d_{k-1}-d_{k}-1-(d_{k-1}-1-i))=d_k + (i-d_{k})$$Can't go lower  $(d_{k-1}-d_{k}-1-(d_{k-1}-1-i))$ for coefficient of $q_k$
 $$d_0-i=(d_0-d_1+1)+(d_1-d_2)+\cdots (d_{k-2}-d_{k-1})+(d_{k-1}-1-i)$$
 So $M$ has at $(d_0-i,d_0-i)$th entry $b_{k,d_k}$ and after that all entries are 0 in that row. Hence $M$ is lower triangular.\vspace{2mm}
 
Matrix is non-singular since the diagonal entries are the leading coefficients of $b_i(x)$ \framebreak


We need to inverse $M$ which is in $Arithmetic-NC$ for general fields and for finite fields, $\mathbb{Q}$ it is in $NC$. \vspace{5mm}

\begin{theorem}
	Iterated Polynomial Mod Problem is in $NC$ for finite field and $\mathbb{Q}$ and in $Arithmetic-NC$ for general field.
\end{theorem}
	
\end{frame}
%\section{Open Problem}
%\begin{frame}
%\frametitle{Open Problem}
%\begin{itemize}
%	\item The $gcd$ problem we still don't know if it is in $NC$. 
%	\item Modular Powering i.e. $a^e\bmod b$ where $a,b,e$ are $n$-bit integers are not knowen to be $P$-complete or in $NC$. 
%\end{itemize}
%\end{frame}
\begin{frame}
\begin{center}
		\Huge{Thank You!}

	\end{center}\end{frame}
\begin{frame}[allowframebreaks]
\frametitle{References}
\bibliographystyle{alpha}
\bibliography{bibliography}
%    \nocite{*} % used here because no citation happens in slides
\end{frame}

\end{document}