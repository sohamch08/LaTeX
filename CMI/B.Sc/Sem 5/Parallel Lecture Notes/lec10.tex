\section{Maximal Independent Set ($MIS$)}
\begin{theorem}
	$MIS\in P$
\end{theorem}

\subsection{Matching and Independent Set of Line Graph}

\begin{definition}[Line Graph]
	The line graph of the graph $G$, written $L(G)$  is the graph whose vertices are the edges of $G$, with $(e_1,e_2)\in E(L(G))$ when $e_1\cap e_2\neq \phi$
\end{definition}

\begin{theorem}
	Given a graph $G$ a set of edges $S\subseteq E$ is a matching if and only if it is a independent set in the line graph $L(G)$
\end{theorem}
\begin{proof}
	$(\Rightarrow)$: Let $S$ be a matching of $G$. Therefore for all $e_1,e_2\in S$ we have $e_1\cap e_2-\phi$. Hence $e_1,e_2$ are not adjacent in $L(G)$. Hence $S$ is an independent set of $L(G)$.
	
	$(\Leftarrow)$: Let $S$ be a independent set in $L(G)$. Then for all $e_1,e_2\in S$, $e_1$ and $e_2$ are not adjacent. Therefore $e_1\cap e_2=\phi$. Hence the set $S$ is a set of edges of $G$ where none of them shares any endpoint. Hence $S$ is a matching in $G$.
\end{proof}

\begin{fact}
	Maximal (Maximum) Matching in $G$ is an Maximal (Maximum) Independent Set  in the line graph $L(G)$.
\end{fact}

\subsection{Luby's Algorithm}
\begin{definition}[$RNC^{k}$]
	The class $RNC^{k}$ is the class of problems that can be solved by a randomized algorithm that runs in $O(\log^k n)$ time with a polynomial number of processors. 
\end{definition}
\begin{remark}
	Therefore $RNC$ is the randomized counterpart of NC.\parinn
\end{remark}

%%%%%%%%%%%%%%%%%%%%%%%%
% Luby's Algorithm
%%%%%%%%%%%%%%%%%%%%%%%%






%%%%%%%%%%%%%%%%%%%%%%%%
\parinf Here we denote for any $v\in V$ $d(v)\coloneqq \deg(v)$. Now we will define good and bad vertices for us based on certain properties of vertices like this:\parinn

	A vertex $ v\in V$ is \underline{good} if $$\sum_{u\in N(v)}\frac{1}{2d(u)}\geq \frac16$$A pair of vertices $u,v\in V$ is said to be good if $$good(u,v)\iff \llbracket(u,v)\in E\rrbracket\wedge\llbracket good(u)\vee good(v)\rrbracket$$

Therefore in case of bad vertex $bad(v)=\neg good(v)$ and for a pair of vertices $u,v\in V$ we have $bad(u,v)=\neg good(u,v)$

We 

\begin{lemma}
	For any $v\in V$	$$bad(v)\implies d^+(v)\geq 2d^-(v)\iff  d^-(v)\leq \frac{d(v)}{3}\iff d^+(v)\geq \frac{2d(v)}{3}$$
\end{lemma}
\begin{proof}
	content...
\end{proof}

\begin{lemma}
	At least half of the edges in the graph are good
\end{lemma}
\begin{proof}
	content...
\end{proof}

\begin{lemma}
	Let $X$ ve the random variable representing the number of deleted edges. Then $$\bbE[X]\geq \frac{|E|}{72}$$
\end{lemma}
\begin{proof}
	content...
\end{proof}

\begin{lemma}
		For any $v\in V$ $$Pr[v\notin I\mid v\in S]\leq \frac12$$
\end{lemma}
\begin{proof}
	content...
\end{proof}

\begin{lemma}
		For any $v\in V$ $$Pr[v\in I]\geq \frac{1}{4d(u)}$$
\end{lemma}
\begin{proof}
	content...
\end{proof}

\begin{lemma}
	If $v\in V$ is good then $$Pr[v\in N(I)]\geq \frac1{36}$$
\end{lemma}