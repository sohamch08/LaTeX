%% This is a template for Assignments 
%% My name is Soham Chatterjee
%% I have created this theme.
%% Thanks for using it.


\documentclass[a4paper, 11pt]{article}
\usepackage{comment} % enables the use of multi-line comments (\ifx \fi) 
\usepackage{fullpage} % changes the margin
\usepackage[a4paper, total={7in, 10in}]{geometry}
\usepackage{amsmath,mathtools}
\usepackage{amssymb,amsthm}  % assumes amsmath package installed
\usepackage{float}
\usepackage{graphicx}
\graphicspath{{./images/}}
\usepackage{xcolor}
\usepackage{mdframed}
\usepackage[shortlabels]{enumitem}
\usepackage{indentfirst}
\usepackage{hyperref}
\hypersetup{
	colorlinks=true,
	linkcolor=blue,
	filecolor=magenta,      
	urlcolor=blue!70!red,
	pdftitle={Assignment}, %%%%%%%%%%%%%%%%   WRITE ASSIGNMENT PDF NAME  %%%%%%%%%%%%%%%%%%%%
}
\usepackage[most,many,breakable]{tcolorbox}
%\usepackage{mathpazo}
\usepackage{kpfonts}

\definecolor{mytheorembg}{HTML}{F2F2F9}
\definecolor{mytheoremfr}{HTML}{00007B}


\tcbuselibrary{theorems,skins,hooks}
\newtcbtheorem{problem}{Problem}
{%
	enhanced,
	breakable,
	colback = mytheorembg,
	frame hidden,
	boxrule = 0sp,
	borderline west = {2pt}{0pt}{mytheoremfr},
	sharp corners,
	detach title,
	before upper = \tcbtitle\par\smallskip,
	coltitle = mytheoremfr,
	fonttitle = \bfseries\sffamily,
	description font = \mdseries,
	separator sign none,
	segmentation style={solid, mytheoremfr},
}
{p}

% To give references for any problem use like this
% suppose the problem number is p3 then 2 options either 
% \hyperref[p:p3]{<text you want to use to hyperlink> \ref{p:p3}}
%                  or directly 
%                   \ref{p:p3}



%---------------------------------------
% BlackBoard Math Fonts :-
%---------------------------------------

%Captital Letters
\newcommand{\bbA}{\mathbb{A}}	\newcommand{\bbB}{\mathbb{B}}
\newcommand{\bbC}{\mathbb{C}}	\newcommand{\bbD}{\mathbb{D}}
\newcommand{\bbE}{\mathbb{E}}	\newcommand{\bbF}{\mathbb{F}}
\newcommand{\bbG}{\mathbb{G}}	\newcommand{\bbH}{\mathbb{H}}
\newcommand{\bbI}{\mathbb{I}}	\newcommand{\bbJ}{\mathbb{J}}
\newcommand{\bbK}{\mathbb{K}}	\newcommand{\bbL}{\mathbb{L}}
\newcommand{\bbM}{\mathbb{M}}	\newcommand{\bbN}{\mathbb{N}}
\newcommand{\bbO}{\mathbb{O}}	\newcommand{\bbP}{\mathbb{P}}
\newcommand{\bbQ}{\mathbb{Q}}	\newcommand{\bbR}{\mathbb{R}}
\newcommand{\bbS}{\mathbb{S}}	\newcommand{\bbT}{\mathbb{T}}
\newcommand{\bbU}{\mathbb{U}}	\newcommand{\bbV}{\mathbb{V}}
\newcommand{\bbW}{\mathbb{W}}	\newcommand{\bbX}{\mathbb{X}}
\newcommand{\bbY}{\mathbb{Y}}	\newcommand{\bbZ}{\mathbb{Z}}

%---------------------------------------
% MathCal Fonts :-
%---------------------------------------

%Captital Letters
\newcommand{\mcA}{\mathcal{A}}	\newcommand{\mcB}{\mathcal{B}}
\newcommand{\mcC}{\mathcal{C}}	\newcommand{\mcD}{\mathcal{D}}
\newcommand{\mcE}{\mathcal{E}}	\newcommand{\mcF}{\mathcal{F}}
\newcommand{\mcG}{\mathcal{G}}	\newcommand{\mcH}{\mathcal{H}}
\newcommand{\mcI}{\mathcal{I}}	\newcommand{\mcJ}{\mathcal{J}}
\newcommand{\mcK}{\mathcal{K}}	\newcommand{\mcL}{\mathcal{L}}
\newcommand{\mcM}{\mathcal{M}}	\newcommand{\mcN}{\mathcal{N}}
\newcommand{\mcO}{\mathcal{O}}	\newcommand{\mcP}{\mathcal{P}}
\newcommand{\mcQ}{\mathcal{Q}}	\newcommand{\mcR}{\mathcal{R}}
\newcommand{\mcS}{\mathcal{S}}	\newcommand{\mcT}{\mathcal{T}}
\newcommand{\mcU}{\mathcal{U}}	\newcommand{\mcV}{\mathcal{V}}
\newcommand{\mcW}{\mathcal{W}}	\newcommand{\mcX}{\mathcal{X}}
\newcommand{\mcY}{\mathcal{Y}}	\newcommand{\mcZ}{\mathcal{Z}}



%---------------------------------------
% Bold Math Fonts :-
%---------------------------------------

%Captital Letters
\newcommand{\bmA}{\boldsymbol{A}}	\newcommand{\bmB}{\boldsymbol{B}}
\newcommand{\bmC}{\boldsymbol{C}}	\newcommand{\bmD}{\boldsymbol{D}}
\newcommand{\bmE}{\boldsymbol{E}}	\newcommand{\bmF}{\boldsymbol{F}}
\newcommand{\bmG}{\boldsymbol{G}}	\newcommand{\bmH}{\boldsymbol{H}}
\newcommand{\bmI}{\boldsymbol{I}}	\newcommand{\bmJ}{\boldsymbol{J}}
\newcommand{\bmK}{\boldsymbol{K}}	\newcommand{\bmL}{\boldsymbol{L}}
\newcommand{\bmM}{\boldsymbol{M}}	\newcommand{\bmN}{\boldsymbol{N}}
\newcommand{\bmO}{\boldsymbol{O}}	\newcommand{\bmP}{\boldsymbol{P}}
\newcommand{\bmQ}{\boldsymbol{Q}}	\newcommand{\bmR}{\boldsymbol{R}}
\newcommand{\bmS}{\boldsymbol{S}}	\newcommand{\bmT}{\boldsymbol{T}}
\newcommand{\bmU}{\boldsymbol{U}}	\newcommand{\bmV}{\boldsymbol{V}}
\newcommand{\bmW}{\boldsymbol{W}}	\newcommand{\bmX}{\boldsymbol{X}}
\newcommand{\bmY}{\boldsymbol{Y}}	\newcommand{\bmZ}{\boldsymbol{Z}}
%Small Letters
\newcommand{\bma}{\boldsymbol{a}}	\newcommand{\bmb}{\boldsymbol{b}}
\newcommand{\bmc}{\boldsymbol{c}}	\newcommand{\bmd}{\boldsymbol{d}}
\newcommand{\bme}{\boldsymbol{e}}	\newcommand{\bmf}{\boldsymbol{f}}
\newcommand{\bmg}{\boldsymbol{g}}	\newcommand{\bmh}{\boldsymbol{h}}
\newcommand{\bmi}{\boldsymbol{i}}	\newcommand{\bmj}{\boldsymbol{j}}
\newcommand{\bmk}{\boldsymbol{k}}	\newcommand{\bml}{\boldsymbol{l}}
\newcommand{\bmm}{\boldsymbol{m}}	\newcommand{\bmn}{\boldsymbol{n}}
\newcommand{\bmo}{\boldsymbol{o}}	\newcommand{\bmp}{\boldsymbol{p}}
\newcommand{\bmq}{\boldsymbol{q}}	\newcommand{\bmr}{\boldsymbol{r}}
\newcommand{\bms}{\boldsymbol{s}}	\newcommand{\bmt}{\boldsymbol{t}}
\newcommand{\bmu}{\boldsymbol{u}}	\newcommand{\bmv}{\boldsymbol{v}}
\newcommand{\bmw}{\boldsymbol{w}}	\newcommand{\bmx}{\boldsymbol{x}}
\newcommand{\bmy}{\boldsymbol{y}}	\newcommand{\bmz}{\boldsymbol{z}}

%---------------------------------------
% Scr Math Fonts :-
%---------------------------------------

\newcommand{\sA}{{\mathscr{A}}}   \newcommand{\sB}{{\mathscr{B}}}
\newcommand{\sC}{{\mathscr{C}}}   \newcommand{\sD}{{\mathscr{D}}}
\newcommand{\sE}{{\mathscr{E}}}   \newcommand{\sF}{{\mathscr{F}}}
\newcommand{\sG}{{\mathscr{G}}}   \newcommand{\sH}{{\mathscr{H}}}
\newcommand{\sI}{{\mathscr{I}}}   \newcommand{\sJ}{{\mathscr{J}}}
\newcommand{\sK}{{\mathscr{K}}}   \newcommand{\sL}{{\mathscr{L}}}
\newcommand{\sM}{{\mathscr{M}}}   \newcommand{\sN}{{\mathscr{N}}}
\newcommand{\sO}{{\mathscr{O}}}   \newcommand{\sP}{{\mathscr{P}}}
\newcommand{\sQ}{{\mathscr{Q}}}   \newcommand{\sR}{{\mathscr{R}}}
\newcommand{\sS}{{\mathscr{S}}}   \newcommand{\sT}{{\mathscr{T}}}
\newcommand{\sU}{{\mathscr{U}}}   \newcommand{\sV}{{\mathscr{V}}}
\newcommand{\sW}{{\mathscr{W}}}   \newcommand{\sX}{{\mathscr{X}}}
\newcommand{\sY}{{\mathscr{Y}}}   \newcommand{\sZ}{{\mathscr{Z}}}


%---------------------------------------
% Math Fraktur Font
%---------------------------------------

%Captital Letters
\newcommand{\mfA}{\mathfrak{A}}	\newcommand{\mfB}{\mathfrak{B}}
\newcommand{\mfC}{\mathfrak{C}}	\newcommand{\mfD}{\mathfrak{D}}
\newcommand{\mfE}{\mathfrak{E}}	\newcommand{\mfF}{\mathfrak{F}}
\newcommand{\mfG}{\mathfrak{G}}	\newcommand{\mfH}{\mathfrak{H}}
\newcommand{\mfI}{\mathfrak{I}}	\newcommand{\mfJ}{\mathfrak{J}}
\newcommand{\mfK}{\mathfrak{K}}	\newcommand{\mfL}{\mathfrak{L}}
\newcommand{\mfM}{\mathfrak{M}}	\newcommand{\mfN}{\mathfrak{N}}
\newcommand{\mfO}{\mathfrak{O}}	\newcommand{\mfP}{\mathfrak{P}}
\newcommand{\mfQ}{\mathfrak{Q}}	\newcommand{\mfR}{\mathfrak{R}}
\newcommand{\mfS}{\mathfrak{S}}	\newcommand{\mfT}{\mathfrak{T}}
\newcommand{\mfU}{\mathfrak{U}}	\newcommand{\mfV}{\mathfrak{V}}
\newcommand{\mfW}{\mathfrak{W}}	\newcommand{\mfX}{\mathfrak{X}}
\newcommand{\mfY}{\mathfrak{Y}}	\newcommand{\mfZ}{\mathfrak{Z}}
%Small Letters
\newcommand{\mfa}{\mathfrak{a}}	\newcommand{\mfb}{\mathfrak{b}}
\newcommand{\mfc}{\mathfrak{c}}	\newcommand{\mfd}{\mathfrak{d}}
\newcommand{\mfe}{\mathfrak{e}}	\newcommand{\mff}{\mathfrak{f}}
\newcommand{\mfg}{\mathfrak{g}}	\newcommand{\mfh}{\mathfrak{h}}
\newcommand{\mfi}{\mathfrak{i}}	\newcommand{\mfj}{\mathfrak{j}}
\newcommand{\mfk}{\mathfrak{k}}	\newcommand{\mfl}{\mathfrak{l}}
\newcommand{\mfm}{\mathfrak{m}}	\newcommand{\mfn}{\mathfrak{n}}
\newcommand{\mfo}{\mathfrak{o}}	\newcommand{\mfp}{\mathfrak{p}}
\newcommand{\mfq}{\mathfrak{q}}	\newcommand{\mfr}{\mathfrak{r}}
\newcommand{\mfs}{\mathfrak{s}}	\newcommand{\mft}{\mathfrak{t}}
\newcommand{\mfu}{\mathfrak{u}}	\newcommand{\mfv}{\mathfrak{v}}
\newcommand{\mfw}{\mathfrak{w}}	\newcommand{\mfx}{\mathfrak{x}}
\newcommand{\mfy}{\mathfrak{y}}	\newcommand{\mfz}{\mathfrak{z}}

%---------------------------------------
% Bar
%---------------------------------------

%Captital Letters
\newcommand{\ovA}{\overline{A}}	\newcommand{\ovB}{\overline{B}}
\newcommand{\ovC}{\overline{C}}	\newcommand{\ovD}{\overline{D}}
\newcommand{\ovE}{\overline{E}}	\newcommand{\ovF}{\overline{F}}
\newcommand{\ovG}{\overline{G}}	\newcommand{\ovH}{\overline{H}}
\newcommand{\ovI}{\overline{I}}	\newcommand{\ovJ}{\overline{J}}
\newcommand{\ovK}{\overline{K}}	\newcommand{\ovL}{\overline{L}}
\newcommand{\ovM}{\overline{M}}	\newcommand{\ovN}{\overline{N}}
\newcommand{\ovO}{\overline{O}}	\newcommand{\ovP}{\overline{P}}
\newcommand{\ovQ}{\overline{Q}}	\newcommand{\ovR}{\overline{R}}
\newcommand{\ovS}{\overline{S}}	\newcommand{\ovT}{\overline{T}}
\newcommand{\ovU}{\overline{U}}	\newcommand{\ovV}{\overline{V}}
\newcommand{\ovW}{\overline{W}}	\newcommand{\ovX}{\overline{X}}
\newcommand{\ovY}{\overline{Y}}	\newcommand{\ovZ}{\overline{Z}}
%Small Letters
\newcommand{\ova}{\overline{a}}	\newcommand{\ovb}{\overline{b}}
\newcommand{\ovc}{\overline{c}}	\newcommand{\ovd}{\overline{d}}
\newcommand{\ove}{\overline{e}}	\newcommand{\ovf}{\overline{f}}
\newcommand{\ovg}{\overline{g}}	\newcommand{\ovh}{\overline{h}}
\newcommand{\ovi}{\overline{i}}	\newcommand{\ovj}{\overline{j}}
\newcommand{\ovk}{\overline{k}}	\newcommand{\ovl}{\overline{l}}
\newcommand{\ovm}{\overline{m}}	\newcommand{\ovn}{\overline{n}}
\newcommand{\ovo}{\overline{o}}	\newcommand{\ovp}{\overline{p}}
\newcommand{\ovq}{\overline{q}}	\newcommand{\ovr}{\overline{r}}
\newcommand{\ovs}{\overline{s}}	\newcommand{\ovt}{\overline{t}}
\newcommand{\ovu}{\overline{u}}	\newcommand{\ovv}{\overline{v}}
\newcommand{\ovw}{\overline{w}}	\newcommand{\ovx}{\overline{x}}
\newcommand{\ovy}{\overline{y}}	\newcommand{\ovz}{\overline{z}}

%---------------------------------------
% Tilde
%---------------------------------------

%Captital Letters
\newcommand{\tdA}{\tilde{A}}	\newcommand{\tdB}{\tilde{B}}
\newcommand{\tdC}{\tilde{C}}	\newcommand{\tdD}{\tilde{D}}
\newcommand{\tdE}{\tilde{E}}	\newcommand{\tdF}{\tilde{F}}
\newcommand{\tdG}{\tilde{G}}	\newcommand{\tdH}{\tilde{H}}
\newcommand{\tdI}{\tilde{I}}	\newcommand{\tdJ}{\tilde{J}}
\newcommand{\tdK}{\tilde{K}}	\newcommand{\tdL}{\tilde{L}}
\newcommand{\tdM}{\tilde{M}}	\newcommand{\tdN}{\tilde{N}}
\newcommand{\tdO}{\tilde{O}}	\newcommand{\tdP}{\tilde{P}}
\newcommand{\tdQ}{\tilde{Q}}	\newcommand{\tdR}{\tilde{R}}
\newcommand{\tdS}{\tilde{S}}	\newcommand{\tdT}{\tilde{T}}
\newcommand{\tdU}{\tilde{U}}	\newcommand{\tdV}{\tilde{V}}
\newcommand{\tdW}{\tilde{W}}	\newcommand{\tdX}{\tilde{X}}
\newcommand{\tdY}{\tilde{Y}}	\newcommand{\tdZ}{\tilde{Z}}
%Small Letters
\newcommand{\tda}{\tilde{a}}	\newcommand{\tdb}{\tilde{b}}
\newcommand{\tdc}{\tilde{c}}	\newcommand{\tdd}{\tilde{d}}
\newcommand{\tde}{\tilde{e}}	\newcommand{\tdf}{\tilde{f}}
\newcommand{\tdg}{\tilde{g}}	\newcommand{\tdh}{\tilde{h}}
\newcommand{\tdi}{\tilde{i}}	\newcommand{\tdj}{\tilde{j}}
\newcommand{\tdk}{\tilde{k}}	\newcommand{\tdl}{\tilde{l}}
\newcommand{\tdm}{\tilde{m}}	\newcommand{\tdn}{\tilde{n}}
\newcommand{\tdo}{\tilde{o}}	\newcommand{\tdp}{\tilde{p}}
\newcommand{\tdq}{\tilde{q}}	\newcommand{\tdr}{\tilde{r}}
\newcommand{\tds}{\tilde{s}}	\newcommand{\tdt}{\tilde{t}}
\newcommand{\tdu}{\tilde{u}}	\newcommand{\tdv}{\tilde{v}}
\newcommand{\tdw}{\tilde{w}}	\newcommand{\tdx}{\tilde{x}}
\newcommand{\tdy}{\tilde{y}}	\newcommand{\tdz}{\tilde{z}}

%---------------------------------------
% Vec
%---------------------------------------

%Captital Letters
\newcommand{\vcA}{\vec{A}}	\newcommand{\vcB}{\vec{B}}
\newcommand{\vcC}{\vec{C}}	\newcommand{\vcD}{\vec{D}}
\newcommand{\vcE}{\vec{E}}	\newcommand{\vcF}{\vec{F}}
\newcommand{\vcG}{\vec{G}}	\newcommand{\vcH}{\vec{H}}
\newcommand{\vcI}{\vec{I}}	\newcommand{\vcJ}{\vec{J}}
\newcommand{\vcK}{\vec{K}}	\newcommand{\vcL}{\vec{L}}
\newcommand{\vcM}{\vec{M}}	\newcommand{\vcN}{\vec{N}}
\newcommand{\vcO}{\vec{O}}	\newcommand{\vcP}{\vec{P}}
\newcommand{\vcQ}{\vec{Q}}	\newcommand{\vcR}{\vec{R}}
\newcommand{\vcS}{\vec{S}}	\newcommand{\vcT}{\vec{T}}
\newcommand{\vcU}{\vec{U}}	\newcommand{\vcV}{\vec{V}}
\newcommand{\vcW}{\vec{W}}	\newcommand{\vcX}{\vec{X}}
\newcommand{\vcY}{\vec{Y}}	\newcommand{\vcZ}{\vec{Z}}
%Small Letters
\newcommand{\vca}{\vec{a}}	\newcommand{\vcb}{\vec{b}}
\newcommand{\vcc}{\vec{c}}	\newcommand{\vcd}{\vec{d}}
\newcommand{\vce}{\vec{e}}	\newcommand{\vcf}{\vec{f}}
\newcommand{\vcg}{\vec{g}}	\newcommand{\vch}{\vec{h}}
\newcommand{\vci}{\vec{i}}	\newcommand{\vcj}{\vec{j}}
\newcommand{\vck}{\vec{k}}	\newcommand{\vcl}{\vec{l}}
\newcommand{\vcm}{\vec{m}}	\newcommand{\vcn}{\vec{n}}
\newcommand{\vco}{\vec{o}}	\newcommand{\vcp}{\vec{p}}
\newcommand{\vcq}{\vec{q}}	\newcommand{\vcr}{\vec{r}}
\newcommand{\vcs}{\vec{s}}	\newcommand{\vct}{\vec{t}}
\newcommand{\vcu}{\vec{u}}	\newcommand{\vcv}{\vec{v}}
\newcommand{\vcw}{\vec{w}}	\newcommand{\vcx}{\vec{x}}
\newcommand{\vcy}{\vec{y}}	\newcommand{\vcz}{\vec{z}}

%---------------------------------------
% Greek Letters:-
%---------------------------------------
\newcommand{\eps}{\epsilon}
\newcommand{\veps}{\varepsilon}
\newcommand{\lm}{\lambda}
\newcommand{\Lm}{\Lambda}
\newcommand{\gm}{\gamma}
\newcommand{\Gm}{\Gamma}
\newcommand{\vph}{\varphi}
\newcommand{\ph}{\phi}
\newcommand{\om}{\omega}
\newcommand{\Om}{\Omega}


%%%%%%%%%%%%%%%%%%%%%%%%%%%%%%%%%%%%%%%% MACROS %%%%%%%%%%%%%%%%%%%%%%%%%%%%%%%%%%%%%%%%

%%%%%%%%%%%%%%% Link With an Icon %%%%%%%%%%%%%%% 
\newcommand{\link}[1]{
    \href{#1}{\faIcon{link}}
}

%%%%%%%%%%%%%%% Name Template %%%%%%%%%%%%%%% 
\newcommand{\name}[2]{
    % Name
    \Huge % Font size
    \raggedright \textbf{#1} \par

    \vspace*{0.3cm}
    
    % Profession
    \Large % Font size
    \raggedright #2 \par
    \normalsize \normalfont
}

%%%%%%%%%%%%%%% Contact Details %%%%%%%%%%%%%%%
\newcommand{\info}[2]{
    \faIcon{#2} \hspace{0.2em} #1
}

%%%%%%%%%%%%%%% Email %%%%%%%%%%%%%%%
\newcommand{\email}[1]{
    \info{#1}{envelope}
}

%%%%%%%%%%%%%%% Phone Number %%%%%%%%%%%%%%%
\newcommand{\phone}[1]{
    \info{#1}{mobile-alt}
}

%%%%%%%%%%%%%%% Address %%%%%%%%%%%%%%%
\newcommand{\address}[1]{
    \info{#1}{map-marker-alt}
}

%%%%%%%%%%%%%%% GitHub %%%%%%%%%%%%%%%
\newcommand{\github}[2]{
    \info{\href{#1}{\underline{#2}}}{github}
}

%%%%%%%%%%%%%%% LinkedIn %%%%%%%%%%%%%%%
\newcommand{\linkedin}[2]{
    \info{\href{#1}{\underline{#2}}}{linkedin}
}

%%%%%%%%%%%%%%% ResearchGate %%%%%%%%%%%%%%%
\newcommand{\researchgate}[2]{
    \info{\href{#1}{\underline{#2}}}{researchgate}
}

%%%\newcommand*{\Researchgate}[1]{\sociallink{\researchgatesocialsymbol}{http://www.#1}{#1}}

%%%%%%%%%%%%%%% Website %%%%%%%%%%%%%%%
\newcommand{\website}[1]{
    \info{#1}{link}
}

%%%%%%%%%%%%%%% Draw Skill Bars %%%%%%%%%%%%%%% 
\newcommand{\drawskillbars}[1]{
    \begin{tikzpicture}
        % Draw 5 gray bars
        \foreach \i in {0, 1, 2, 3, 4}{
            \fill[lightgray] (\i * 0.7 + 0.2 *\i,0) rectangle (0.7 + \i * 0.7 + \i * 0.2,0.1);
        }
        
        % Draw number of black bars depending on the skill level
        \foreach \i in {#1}{
            \fill[blue!40] (\i * 0.7 + 0.2 *\i,0) rectangle (0.7 + \i * 0.7 + \i * 0.2,0.1);
            %\fill[title] (\i * 0.7 + 0.2 *\i,0) rectangle (0.7 + \i * 0.7 + \i * 0.2,0.1);
        }
    \end{tikzpicture} \par
}
    
%%%%%%%%%%%%%%% Skills %%%%%%%%%%%%%%%
\newcommand{\skill}[3]{
    % Name of the skill
    \large
    \noindent \hangafter=0
    \adjustbox{valign=t}{\begin{minipage}{0.72\textwidth}
        \large \noindent \hangafter=0
        % Name of the skill
        \textmd{#1} 
        \normalsize \par 
        \vspace{1em}
         % Description
        \noindent \small \color{subtitle} \parbox{1\linewidth}{\textsl{#3}} \par
        \normalsize \par
        \end{minipage}}
    \adjustbox{valign=t}{\begin{minipage}{0.2\textwidth}
        % Skill bars
        \large \hangafter=0
        %\noindent 
        \drawskillbars{#2}
        \end{minipage}}
    \normalsize \par 
    % Skill bars
    %%\drawskillbars{#2}
    %%\vspace{0.5em}
    
    \vspace{1.0em}
    \normalsize \color{black} \par
}

%%%%%%%%%%%%%%% Software %%%%%%%%%%%%%%%
\newcommand{\soft}[2]{
    \adjustbox{valign=t}{\begin{minipage}{0.40\textwidth}
        \large \noindent \hangafter=0
        % Name of the skill
        \textmd{#1} 
        \normalsize \par 
        \vspace{1em}
        \end{minipage}}
    \adjustbox{valign=t}{\begin{minipage}{0.5\textwidth}
        % Skill bars
        \large \noindent \hangafter=0
        \drawskillbars{#2}
        \end{minipage}}
    \normalsize \par 
    \vspace{1em}
}

%%%%%%%%%%%%%%% Personal details %%%%%%%%%%%%%%%
\newcommand{\details}[2]{
    % Name of the language
    \large
    \noindent \hangafter=0 \color{black}
    \adjustbox{valign=t}{\parbox{0.27\linewidth}{#1}}  \adjustbox{valign=t}{\parbox{0.55\linewidth}{#2}} \par
    \vspace{.3em}
    \normalsize \color{black} \par
 }

%%%%%%%%%%%%%%% Language %%%%%%%%%%%%%%%
\newcommand{\lan}[2]{
    % Name of the language
    \large
    \noindent \hangafter=0 \color{black}
    \parbox{0.3\linewidth}{\textmd{#1}}   \color{subtitle} \parbox{0.4\linewidth}{\textsl{#2}} \par
    %\large English \color{subtitle} \textit{Advanced} 
    %\normalsize \par 
    % Knowledge level
    %\noindent \small \color{subtitle} \parbox{1\linewidth}{\textsl{#2}} \par
    \vspace{1.0em}
    \normalsize \color{black} \par
 }

%%%%%%%%%%%%%%% Education %%%%%%%%%%%%%%%
\newcommand{\education}[4]{
    % Name of the studies
    \noindent \large \parbox{.65\linewidth}{\textbf{#1}}
    % Duration in a Box
    \hfill \small
    \tcbox[enhanced,nobeforeafter,box align=base,colback=title,colframe=title,size=fbox,arc=0mm, valign=bottom]{{\textbf{#2}}} \par
    \vspace{0.3em}
    % School Name 
    \normalsize
    \noindent \color{subtitle} \parbox{.9\linewidth}{\textsl{#3}} \par
    % Description
    \normalsize \color{black}
    \vspace*{0.3em}
    \small #4 
    \normalsize \par
    \vspace*{0.5em}
}

%%%%%%%%%%%%%%% Work Experience %%%%%%%%%%%%%%%
\newcommand{\work}[4]{
    % Name of the Job
    \noindent \large \parbox{.65\linewidth}{\textbf{#1}}
    % Duration in a Box 
    \hfill \small
    \tcbox[enhanced,box align=base,nobeforeafter,colback=title,colframe=title,size=fbox,arc=0mm]{\textbf{#2}} \par
    \vspace{0.3em}
    % Name of the Employer
    \noindent \large \color{subtitle} \parbox{.9\linewidth}{\textsl{#3}} \par
    % Description of the job
    \vspace*{0.3em} \color{black}
    \small #4 
    \normalsize \par
}

%%%%%%%%%%%%%%% Teaching %%%%%%%%%%%%%%%
\newcommand{\teaching}[3]{
    % What, Topic and Who/Where/when
    \noindent \adjustbox{valign=t}{\parbox{.99\linewidth}{\text{#1} \text{#2} \textbf{#3}}} 
    \vspace{0.5em}
    \vspace*{1em} 
}

%%%%%%%%%%%%%%% Publications %%%%%%%%%%%%%%%
\newcommand{\publ}[4]{
    % Authors, Title and journal
    \noindent \parbox{.99\linewidth}{\textsl{#3}. \textbf{#1} \textsl{#2} \link{#4}}
    \vspace{0.5em}
    \vspace*{1em} \color{black}
    }

%%%%%%%%%%%%%%% Talks %%%%%%%%%%%%%%%
\newcommand{\talk}[3]{
    % Authors, Title and journal
    \noindent \parbox{.99\linewidth}{\textsl{#3}. \textbf{#1} \textsl{#2}}
    \vspace{0.5em}
    \vspace*{1em} \color{black}
}

%%%%%%%%%%%%%%% Events %%%%%%%%%%%%%%%
\newcommand{\event}[3]{
    \noindent \parbox{.99\linewidth}{\textbf{#1} \textsl{#3} \textsl{#2}}
    \vspace{0.5em}
    \vspace*{1em} \color{black}
}

\setlength{\parindent}{0pt}

%%%%%%%%%%%%%%%%%%%%%%%%%%%%%%%%%%%%%%%%%%%%%%%%%%%%%%%%%%%%%%%%%%%%%%%%%%%%%%%%%%%%%%%%%%%%%%%%%%%%%%%%%%%%%%%%%%%%%%%%%%%%%%%%%%%%%%%%

\begin{document}

%%%%%%%%%%%%%%%%%%%%%%%%%%%%%%%%%%%%%%%%%%%%%%%%%%%%%%%%%%%%%%%%%%%%%%%%%%%%%%%%%%%%%%%%%%%%%%%%%%%%%%%%%%%%%%%%%%%%%%%%%%%%%%%%%%%%%%%%

\textsf{\noindent \large\textbf{Soham Chatterjee} \hfill \textbf{Assignment - 1}\\
    Email: \href{sohamc@cmi.ac.in}{sohamc@cmi.ac.in} \hfill Roll: BMC202175\\
    \normalsize Course: Algorithmic Coding Theory \hfill Date: September 10, 2023}

%%%%%%%%%%%%%%%%%%%%%%%%%%%%%%%%%%%%%%%%%%%%%%%%%%%%%%%%%%%%%%%%%%%%%%%%%%%%%%%%%%%%%%%%%%%%%%%%%%%%%%%%%%%%%%%%%%%%%%%%%%
% Problem 1
%%%%%%%%%%%%%%%%%%%%%%%%%%%%%%%%%%%%%%%%%%%%%%%%%%%%%%%%%%%%%%%%%%%%%%%%%%%%%%%%%%%%%%%%%%%%%%%%%%%%%%%%%%%%%%%%%%%%%%%%%%

\begin{problem}{%problem statement
		Chapter 1
}{p1
% problem reference text
}
%Problem		
Ex 1.18
\end{problem}

\solve{
%Solution
\begin{enumerate}[label=(\alph*)]
	\item If all the first $n-1$ people pass and the last person guess his hat color then the last person gets it right with probability $\frac12$. Hence They win with probability $\frac12$. Hence the $n$ people can win with probability at least $\frac12$.
	\item We can only consider the graphs which for $u,v\in V$ do not contain both the edges $u\to v$ and $v\to u$ cause the graph doesn't have a larger $K(G)$ than the graph which has neither of the edges. We will form a bijection with the strategies for guessing with the directed subgraphs of hypercube. Now in the hypercube every vertex represents a configuration of the hat colors of the $n$ people. Let in the original graph both the edges $u\to v$ and $v\to u$ are there. So $u,v$ differ in at one positions, let at $i$-th position. Now in a strategy if the $i$-th player among $u,v$ guesses $u$ then we keep the edge $v\to u$ and if he guesses $v$ then we keep the edge $v\to u$ and if he passes then we dont draw any edge. This forms a bijection between the strategies for guessing and the subgraphs of the hypercube. So in such a subgraph the vertices with 0 outdegree are the winning positions. Hence winning probability becomes $\frac{K(G)}{2^n}$. Therefore over all subgraphs the maximum of $\frac{K(G)}{2^n}$ is the winning probability of the hat problem.
	\item Let $I$ be the set of vertices which have out-degree 0 and in-degree at least 1. So $$\sum_{v\in V}in-degree(v)\geq \sum_{v\in I}in-degree{v}\geq |S|=K(G)$$And we also have $$\sum_{v\in V}out-degree(v)=\sum_{v\in V-I}out-degree(v)\leq n(2^n-K(G))$$Since $\sum\limits_{v\in V} in-degree(v)=\sum\limits_{v\in V} out-degree(v)$ we have $$K(G)\leq n(2^n-K(G))\implies (n+1)K(G)\leq n2^n\implies \frac{K(G)}{2^n}\leq \frac{n}{n+1}$$Hence the value of $\frac{K(G)}{2^n}$ is atmost $\frac{n}{n+1}$.
	\item First take a code $C\subseteq \{0,1\}^n$ where the distance is 3. Then for any $u\in C$ add the edges $u\to v$ for all $v\notin C$ such that $\Delta(u,v)=1$ i.e. $u,v$ differ in one position. So for every $u\in C$, $out-degree(u)=n$. So for no pair of vertices $x,y$ both the edges $x\to y$ and $y\to x$ are in the graph. So $K(G)=n|C|$. Now if we take $C$ to be the hamming code $[2^l-1,2^l-l-1,3]$ then $|C|=\frac{2^{2^l-1}}{2^l}=\frac{2^n}{n+1}$. Hence $\frac{K(G)}{n}=\frac{2^n}{n+1}\iff \frac{K(G)}{2^n}=\frac{n}{n+1}$. 
\end{enumerate}
}


%%%%%%%%%%%%%%%%%%%%%%%%%%%%%%%%%%%%%%%%%%%%%%%%%%%%%%%%%%%%%%%%%%%%%%%%%%%%%%%%%%%%%%%%%%%%%%%%%%%%%%%%%%%%%%%%%%%%%%%%%%
% Problem 2
%%%%%%%%%%%%%%%%%%%%%%%%%%%%%%%%%%%%%%%%%%%%%%%%%%%%%%%%%%%%%%%%%%%%%%%%%%%%%%%%%%%%%%%%%%%%%%%%%%%%%%%%%%%%%%%%%%%%%%%%%%

\begin{problem}{%problem statement
		Chapter 2
}{p2% problem reference text
}
%Problem
Ex 2.13		

\end{problem}

\solve{
%Solution
The parity check matrix $G$ of $C^{\perp}$ is the generator matrix of $C$. Now $C^{\perp}$ has distance $d^{\perp}$. So the smallest set of linearly independent columns of $G$ is of size $d^{\perp}$. Hence for any set of $d^{\perp}-1$ columns of $G$ they are linearly independent.

So for any $I$ with $|I|=d^{\perp}-1$ we take the $i$th columns for the $i$'s in $I$. So we have this new matrix $A$ of dimention $k\times (d^{\perp}-1)$. Since the columns are linearly independent $A$ has full rank. By singelton bound on $C^{\perp}$ we have $$n-k\leq n-d^{\perp}+1\implies d^{\perp}-1\leq k$$
Now we want to show that for any $v\in \bbF_q^{d^{\perp}-1}$ there exists a solution of $A^Tx=v$ in $C$. Since $A$ has full rank there are $d-1$ columns of $A^T$ which are linearly independent. So those $d^{\perp}-1$ columns of $A^T$ can span the $\bbF_q^{d^{\perp}-1}$. Hence there exists a solution of $A^Tx=v$. Now by Ex 2.6(4) for all $v\in \bbF_q^{d^{\perp}-1}$ there are same number of solutions. Hence $C$ is $d^{\perp}-1$ wise independent. 
}
%%%%%%%%%%%%%%%%%%%%%%%%%%%%%%%%%%%%%%%%%%%%%%%%%%%%%%%%%%%%%%%%%%%%%%%%%%%%%%%%%%%%%%%%%%%%%%%%%%%%%%%%%%%%%%%%%%%%%%%%%%
% Problem 3
%%%%%%%%%%%%%%%%%%%%%%%%%%%%%%%%%%%%%%%%%%%%%%%%%%%%%%%%%%%%%%%%%%%%%%%%%%%%%%%%%%%%%%%%%%%%%%%%%%%%%%%%%%%%%%%%%%%%%%%%%%

\begin{problem}{%problem statement
		Chapter 2
	}{p3% problem reference text
	}
	%Problem		
	Ex 2.14
\end{problem}

\solve{
	%Solution
	\begin{enumerate}
		\item Let $G$ be the generator matrix of $C$. So $G$ is a $k\times n$ matrix. Let the columns of $G$ are $c_1,c_2,\dots,c_n$. Take $S=\{c_i\mid i\in[n]\}$. So $|S|=n$. Now for any $x\in \bbF_q^k$, $\langle x,c_i\rangle$ is the $i$-th coordinate of $xG$. We are given that $wt(xG)\in \lt[ \lt(\frac{1-\eps}{2}\rt)n,\lt(\frac{1+\eps}{2}\rt)n \rt]$. Hence $$Pr\lt[\langle x,c_i\rangle =1\rt]\in \lt[ \frac{1-\eps}{2},\frac{1+\eps}{2}\rt]\implies Pr\lt[\langle x,c_i\rangle =0\rt]\in \lt[ \frac{1-\eps}{2},\frac{1+\eps}{2}\rt]$$Therefore $$\lt| \underset{c\in S}{Pr}\lt[\langle x,c_i\rangle =0\rt] - \underset{c\in S}{Pr}\lt[\langle x,c_i\rangle =1\rt]\rt|\leq \lt|\frac{1+\eps}{2}-\frac{1-\eps}{2}  \rt|\leq \eps$$
		
		Now for any $I\subseteq [k]$ take $e_I=\sum\limits_{i\in I} e_i$. Then for any $x\in \bbF_q^K$, $\sum\limits_{i\in I}x_i=\langle x,e_I\rangle$. Hence from above we have$$\lt| \underset{c\in S}{Pr}\lt[\langle c_i,e_I\rangle =0\rt] - \underset{c\in S}{Pr}\lt[\langle c_i,e_I\rangle =1\rt]\rt|=\lt| \underset{c\in S}{Pr}\lt[\sum\limits_{j\in I}c_{i,j}=0\rt] - \underset{c\in S}{Pr}\lt[\sum\limits_{j\in I}c_{i,j}=1\rt]\rt|\leq \eps$$ Hence $S$ is $\eps$-biasesd. 
		\item We have the code $[n,k,\delta n]_2$. Which have hamming weight in the range $\lt(\lt(\frac12-\gm\rt)n,\lt(\frac12+\gm \rt)n \rt)\implies \delta \in \lt(\frac12-\gm,\frac12+\gm\rt)$. Now if we construct a code with relative hamming weight in the range $\lt(  \frac{1-\eps}{2},\frac{1+\eps}{2}\rt)$ whose generator matrix is $k\times n^{O(\gm^{-1}\log \frac1{\eps})}$ then by part (1) we have a code with $\eps$-bias of size $n^{O(\gm^{-1}\log \frac1{\eps})}$. So for that need $m=O\lt(\gm^{-1}\log \frac{1}{\eps}\rt)$\parinn
		
		Now we construct a new code $\lt[n^m,k,\frac12\lt(1-(1-2\delta)^m\rt)n^m  \rt]_2$ by Ex 2.17(e) from $[n,k,d]_2$. We claim that this new code has relative hamming weight in the range $\lt(\frac{1-\eps}{2},\frac{1+\eps}{2}\rt)$. \begin{align*}
			         & \delta \in \lt(\frac12-\gm,\frac+\gm\rt)                                                \\
			\implies & 2\delta \in (1-2\gm,1+2\gm)                                                             \\
			\implies & 1-2\delta\in (=2\gm,2\gm)                                                               \\
			\implies & (1-2\delta)^m\in (-(2\gm)^m,(2\gm)^m)                                                   \\
			\implies & \frac12\lt( 1-(1-2\delta)^m \rt)\in \lt(\frac{1-(2\gm)^m}{2},\frac{1+(2\gm)^m}{2}  \rt)
		\end{align*}We need $$\frac{1-(2\gm)^m}{2}\geq \frac{1-\eps}{2}\iff \eps\geq (2\gm)^m$$Now \begin{align*}
		\eps \geq (2\gm)^m & \iff \log \eps \geq m(\log \gm +1)\\
		& \iff \frac{\log \eps}{1+\log \gm}\geq m=\om \gm^{-1}\log \frac1{\eps}\quad [\text{$\om$ is a constant}]\\
		& \iff \frac{-\gm}{1+\log \gm}\geq \om
	\end{align*}Now since $0<\gm<\frac12$ we have $\log gm<-1$ So $\frac{-\gm}{1+\log \gm}>0$. So we take $\om=\frac{-\gm}{1+\log \gm}$. Hence for this value of $c$ we have the relative hamming weight $\geq \frac{1-\eps}{2}$. Similarly we have the relative hamming weight $\leq \frac{1+\eps}{2}$. Hence this new formed code has relative hamming weight in the range $\lt(  \frac{1-\eps}{2},\frac{1+\eps}{2}\rt)$. The generator matrix of this code is of the dimension $k\times n^{O(\gm^{-1}\log \frac1{\eps})}$. Hence by using part (1) we have a $\eps$-biased space of size $n^{O(\gm^{-1}\log \frac1{\eps})}$. 
	\end{enumerate}
}
%%%%%%%%%%%%%%%%%%%%%%%%%%%%%%%%%%%%%%%%%%%%%%%%%%%%%%%%%%%%%%%%%%%%%%%%%%%%%%%%%%%%%%%%%%%%%%%%%%%%%%%%%%%%%%%%%%%%%%%%%%
% Problem 4
%%%%%%%%%%%%%%%%%%%%%%%%%%%%%%%%%%%%%%%%%%%%%%%%%%%%%%%%%%%%%%%%%%%%%%%%%%%%%%%%%%%%%%%%%%%%%%%%%%%%%%%%%%%%%%%%%%%%%%%%%%

\begin{problem}{%problem statement
		Chapter 2
	}{p3% problem reference text
	}
	%Problem		
	Ex 2.16
\end{problem}

\solve{
	%Solution
	\begin{enumerate}[label=(\alph*)]
		\item Since $G$ has full rank, $rank(G)=k$. Therefore in the reduced column echelon form of $G$ the first $k$ columns forms a identity matrix $I_k$. We denote the matrix formed by the rest $n-k$ columns by $A$. Since the reduced column echelon form of a matrix and the matrix generate the same vector space they are equivalent. And since the reduced column echelon form can be obtained through the Gaussian elimination method we can convert $G$ to a matrix $G'$ of the form $G'=[I_k|A]$ in polynomial time where $G'$ and $G$ are equivalent.
		\item We should have $GH^T=0$ where $G$ is of the form $G=[I_k|A]$. where $A$ is a $k\times (n-k)$ matrix. Take $H=[-A^T|I_{n-k}]$. Suppose we denote $G=(g_{i,j})_{\substack{1\leq i\leq k\\ 1\leq j\leq n}}$ and $H=(h_{i,j})_{\substack{1\leq i\leq n\\ 1\leq j\leq n-k}}$.  Let $C=GH^T=(c_{i,j})_{\substack{1\leq i\leq k\\ 1\leq j\leq n-k}}$\begin{align*}
			c_{i,j}&=\sum_{m=1}^ng_{i,m}h_{m,j}=\sum_{m=1}^{k}\delta_{i,m}h_{m,j}+\sum_{m=k+1}^ng_{i,m}\delta_{m-k,j}=h_{i,j}+g_{i,k+j}=-a_{i,j}+a_{i,j}=0
		\end{align*}So we get every entry of $C$ is 0. Hence $GH^T=0$. Therefore $H$ is the parity check matrix of $G$ and since $H$ is of the form $H=[-A^T|I_{n-k}]$ so it has full rank $n-k$. Hence $H$ is a parity check matrix.
		\item The general parity check matrix $H$ of the hamming code $[2^r,2^r-1-r,3]$ is the the $i$th column is the binary representation of $i$. Now by gaussian elimination we can convert it to the form $H'=[A\mid I_{r}]$. So now in $H'$ for the last $r$ many columns the $i$th columns is the binary representation of $2^i$. In $H$ the $i$th column for which $2^k<i<2^{k+1}$ in $H'$ it is the $(i-k)$th column. So then the generator matrix of the hamming code $[2^r-1,2^r-1-r,3]$ is the matrix $G=[I_{2^r-1-r}\mid -A^T]$ by part (b)
	\end{enumerate}
}
%%%%%%%%%%%%%%%%%%%%%%%%%%%%%%%%%%%%%%%%%%%%%%%%%%%%%%%%%%%%%%%%%%%%%%%%%%%%%%%%%%%%%%%%%%%%%%%%%%%%%%%%%%%%%%%%%%%%%%%%%%
% Problem 4
%%%%%%%%%%%%%%%%%%%%%%%%%%%%%%%%%%%%%%%%%%%%%%%%%%%%%%%%%%%%%%%%%%%%%%%%%%%%%%%%%%%%%%%%%%%%%%%%%%%%%%%%%%%%%%%%%%%%%%%%%%

\begin{problem}{%problem statement
		Chapter 2
	}{p4% problem reference text
	}
	%Problem		
	Ex 2.17
\end{problem}

\solve{
	%Solution
	\begin{enumerate}[label=(\alph*)]
		\item We encode each alphabet in $(n,k,d)_{2^m}$ in binary $\{0,1\}$. So each alphabet takes $m$ bits to encode. So now in the old code to encode each code in binary we have to encode all the $n$ alphabets in binary which takes total $nm$ bits to encode. So in the new code the code length becomes $nm$. \parinn
		
		Initially $|C|=(2^m)^k=2^{mk}$. Hence the new dimention of the code becomes $km$. And the distance becomes at least the same as old one since we are just encoding all the alphabets in binary. So the new distance $d'\geq d$. The new code is $(nm,km,d'\geq d)_2$.
		\item Like the previos part we again encode the alphabets in binary so the new code length becomes $nm.$ $\bbF_{2^m}\cong \frac{\bbF_2[x]}{p(x)}$ where $p(x)$ is a irreducible polybomial of degree $m$. So we can think $\bbF_{2^m}$ as a vector space over $\bbF_2$ where the basis is $x^{n-1},\dots,x,1$. So we can think of an isomorphism $\varphi$ between the vector spaces $\bbF_{2^m}$ and $\lt(  \bbF_2\rt)^m$ by $$\varphi\lt(\sum\limits_{i=0}^{n-1}a_ix^i\rt)=(a_1,\dots,a_{n-1})\qquad \forall\ a_i\in \bbF_2$$ This creates an isomorphism between the vector spaces. So after the binary conversion this still remains as vector space. Like the same logic as for the part (a) we encode all the alphabets in binary which takes $m$ bits. So for each $n$ length old code the new code is of $nm$ length. So the new dimention of the code becomes like before $km$ and the distance is at least $d$. So the new linear code is $[nm,km,d'\geq d]_2$
		\item Since the distance is $d$ hence the minimum weight is $d$. Hence there exists a code $c_0$ for which $wt(c_0)=d$. WLOG suppose in the last $d$ many positions $c_0$ has nonzero words. Then we drop the last $d$ many positions from all the code words. We obtain a new code $C'$ of length $d$.\parinn
		
		Take the  set $S=\{c\in C\mid wt(c)=d\text{ the first $d$ positions have the nonzero elements}\}$. $S$ is also a vector space. By rank nullity theorem we can say $\dim S+\dim C'=k$ since all vectors of $S$ becomes the zero vector in $C'$. Now we claim that $S$ is spanned by one vector. If not let $c_1,c_2\in C$ which are linearly independent. Then there exists an $\alpha\in \bbF_q$ such that the first element of $c_1+\alpha c_2$ becomes zero and this new code is also in $S$. But this code has at most $d-1$ many nonzero elements and at least 1 nonzero element. But we created $S$ where all codes have $d$ many nonzero elements in the first $d$ positions. Hence contradiction. Therefore $\dim S=1$. Hence $\dim C'=k-1$

		We can assume $d>q$ cause if $d\leq q$ then $\lt\lceil \frac{d}{q}\rt\rceil=1$ that means the new code has distance at least 1 which is obviously true for any code. So we take $d>q$. Consider the codes $c+\alpha c_0$ where $\alpha\in\bbF_q$. Since $\alpha $ varies over all elements of the field all the elements of $\alpha c_0$ varies over all elements of $\bbF_q$. So every nonzero element of $c$ becomes zero in $c+\alpha c_0$ for some $\alpha $. Since there are $d$ many symbols by pegion hole principle there exists an $\alpha $ such that  $\lt\lceil \frac{d}{q}\rt\rceil$ many elements of $c+\alpha c_0$ becomes zero. Now this $c+\alpha c_0\in C$. Now $wt(c+\alpha c_0)\geq d-\lt\lceil\frac{d}{q}\rt\rceil +d'$. Since the distance of $C$ is $d$ we have $$d-\lt\lceil \frac{d}{q}\rt\rceil+d'\geq d\implies d'\geq \lt\lceil \frac{d}{q}\rt\rceil$$So we have distance $d'\geq \lt\lceil \frac{d}{q}\rt\rceil$. So this new code becomes $\lt[n-d,k-1,d'\geq  \lt\lceil \frac{d}{q}\rt\rceil \rt]$
		 
		\item For each $c\in C$ where $C$ is the given linear code $[n,k,d]_q$ we form the new code $c^{\otimes m}\coloneqq\underbrace{c\otimes c\otimes \cdots\otimes c}_{m\text{ times}}$. Let the old alphabet set is $\Sigma$. We create the new alphabet set of size $q^m$ which is the set of all possible $m$-tuples i.e. $\Sigma' =\lt\{ (q_1,\dots,q_m)\mid q_i\in \Sigma\ \forall \ i\in [m] \rt\}$. So the new alphabet size becomes $|\Sigma'|=q^m$. 
		Now let $c\in C$ is $c=(q_1,\dots,q_n)$. Now if we expand out the $c^{\otimes m}$ each element of it is a $m$-product of the letters from the set $\{q_1,\dots,q_n\}$. So we can represent each element of it as a $m$-tuple. Now each of this tuple is an element of the alphabet set we created just now. So in the new code number of codes remains same but the alphabet size is now $q^m$. Now originally $|C|=q^k=(q^m)^{\frac{k}{m}}$ So the dimention is $\frac{k}{m}$. Now for the distance initially the distance was $d=\delta n$. So between any two code words at most $n-d$ many positions can be same. So If we tensor each code words $m$ times then at most $(n-d)^m$ many positions can be same. Hence at least $n^m-(n-d)^m$ many positions are different. Now we have to show that there exists a pair of code words which are of $n^m-(n-d)^m$ distance. Take the $0$ code. So $0^{\otimes m}$ is also a $0$ code with $n^m$ many 0's. Now take the code $c_0$ which had weight $d$. Hence $c_0$ has only $d$ many nonzero alphabets and rest $n-d$ alphabets are 0's. So $c_o^{\otimes m}$ has at least $(n-d)^m$ many zeros. Hence it has most $n^m-(n-d)^m$ many nonzero elements. Hence the new distance is $n^m-(n-\delta n)^m=\lt(1-(1-\delta)^m\rt)n^m$. Hence the new code is $\lt( n^m,\frac{k}{m},\lt(1-(1-\delta)^m\rt)n^m \rt)_{q^m}$
		\item Let the generator matrix of $[n,k,d]_2$ $(d=\delta n)$ is $G$ which is a $k\times n$ matrix. Now we create a new generator matrix $G'$ of dimension $n^m\times k$ where we represent a column by a $m$-tuple $(i_1,\dots,i_m)$ where $1\leq i_j\leq n$. We denote the $i$-th column of $G$ by $c_i$. We first represent each bit $b$ by $(-1)^b$ i.e instead of $\{0,1\}$ we will use $\{1,-1\}$. Now in $G'$ the $(i_1,\dots,i_m)$th column is the sum of the $i_1,\dots,i_m$-th columns. So now for any $x$ $xG$ is a $n^m$ length code. Hence the new code has code length $n^m$. Since the number of codes remains same and so is the alphabets we have dimension same as before, $k$. We denote $n^m=N$. Now for any $i\in [n]$, $\langle c_i,x\rangle$ is the $i$-th coordinate of $xG$.  \parinn
		
		So for any $N$-tuple $v$ $$\bbE[x]=\frac{\#\text{0 in $v$}-\#\text{1 in $v$}}{N}$$So \begin{align*}
			\bbE[xG'] & = \frac{1}{N}\sum_{i_1,\dots,i_m\in [n]}(-1)^{\lt\langle\sum\limits_{j=1}^m  c_{i_j},x\rt\rangle} = \frac{1}{N}\sum_{i_1,\dots,i_m\in [n]}(-1)^{\sum\limits_{j=1}^m  \lt\langle c_{i_j},x\rt\rangle} \\
			          & = \frac{1}{N}\sum_{i_1,\dots,i_m\in [n]}\lt[ \prod_{j=1}^m (-1)^{\lt\langle c_{i_j},x\rt\rangle}\rt]  = \prod_{j\in [m]}\lt[\frac1n\sum_{i\in [n]}(-1)^{\langle c_i,x\rangle}  \rt]                  \\
			          & = \prod_{j\in [m]} \bbE[xG]= \lt[\frac{\#\text{0 in $xG$}-\#\text{1 in $xG$}}{n}  \rt]^m\geq \lt[ \frac{n-2d}{n}\rt]^m
		\end{align*}So $\#$0 in $xG'-\#$1 in $xG'\geq n^m\lt[ \frac{n-2d}{n}\rt]^m=(n-2d)^m$. So $\#1$ in $xG'\geq \frac{n^m-(n-2d)^m}{2}=\frac12(1-(1-2\delta)^m)n^m$. Now let $c$ be the code in $[n,k,d]_2$ such that $wt(c)=d$ let $x'$ be such that $x'G=c$. Then $$\bbE[x'G']=\lt[\frac{\#\text{0 in $xG$}-\#\text{1 in $xG$}}{n}  \rt]^m=\lt[\frac{\#\text{0 in $c$}-\#\text{1 in $c$}}{n}  \rt]^m=\lt[\frac{n-d-d}{n}  \rt]^m=\lt[\frac{n-2d}{n}\rt]^m$$Hence $\#1$ in $x'G'$ is $\frac{n^m-(n-2d)^m}{2}$. Hence this new code is $\lt[n^m,k,\frac12(1-(1-2\delta)^m)n^m\rt]_2$
	\end{enumerate}
}

%%%%%%%%%%%%%%%%%%%%%%%%%%%%%%%%%%%%%%%%%%%%%%%%%%%%%%%%%%%%%%%%%%%%%%%%%%%%%%%%%%%%%%%%%%%%%%%%%%%%%%%%%%%%%%%%%%%%%%%%%%
% Problem 5
%%%%%%%%%%%%%%%%%%%%%%%%%%%%%%%%%%%%%%%%%%%%%%%%%%%%%%%%%%%%%%%%%%%%%%%%%%%%%%%%%%%%%%%%%%%%%%%%%%%%%%%%%%%%%%%%%%%%%%%%%%

\begin{problem}{%problem statement
		Chapter 5
	}{p1
		% problem reference text
	}
	%Problem
	Ex 5.4	
\end{problem}


\solve{
	%Solution
Let $\gm$ be the premitive element of $\bbF_q^*$. Then the generator matrix of $RS_{\bbF_q^*}[n,k]$ is the matrix $$G=\mat{1 & 1 & 1& \cdots & 1\\
	0 & \gm & \gm^2 & \cdots & \gm^{n-1} \\
	0 & \gm^2 & \gm^4 & \cdots & \gm^{2(n-1)} \\
	\vdots & \vdots & \vdots & \ddots & \vdots\\
	0 & \gm^{k-1} & \gm^{2(k-1)} & \cdots & \gm^{(n-1)(k-1)}
}$$Now consider this vandermonde matrix which is $(n-k)\times n$ matrix
$$H=\mat{1 & 1 & 1& \cdots & 1\\
	0 & \gm & \gm^2 & \cdots & \gm^{n-1} \\
	0 & \gm^2 & \gm^4 & \cdots & \gm^{2(n-1)} \\
	\vdots & \vdots & \vdots & \ddots & \vdots\\
	0 & \gm^{n-k-1} & \gm^{2(n-k-1)} & \cdots & \gm^{(n-1)(n-k-1)}
}$$If we show that $GH^T=0$ then we can say $H$ is the generator matrix of $C^{\perp}$. And since it is vanderminde matrix $C^{\perp}$ is Reed Solomon Code. Since $Rank(H)=n-k$. $C^{\perp}=RS[n,n-k]$. \parinf

\textbf{\textit{Claim:}} For all $0\leq l\leq n-1$ and  we have $\sum\limits_{i=0}^{n-2}\lt(\gm^i\rt)^l=0$

\textbf{\textit{Proof:}} \begin{align*}
	\sum_{i=0}^{n-2}\lt(\gm^i\rt)^l&=\sum_{i=0}^{n-2}\lt(\gm^l\rt)^i=\frac{1-\lt(\gm^{l}\rt)^{n-1}}{1-\gm^l}=\frac{1-1}{1-1}=0
\end{align*} \qed\parinn 

Now
$$GH^T= \mat{1 & 1 & 1& \cdots & 1\\
	0 & \gm & \gm^2 & \cdots & \gm^{n-1} \\
	0 & \gm^2 & \gm^4 & \cdots & \gm^{2(n-1)} \\
	\vdots & \vdots & \vdots & \ddots & \vdots\\
	0 & \gm^{k-1} & \gm^{2(k-1)} & \cdots & \gm^{(n-1)(k-1)}
} \mat{1 & 0 & 0& \cdots & 0\\
1 & \gm & \gm^2 & \cdots & \gm^{n-k-1} \\
1 & \gm^2 & \gm^4 & \cdots & \gm^{2(n-k-1)} \\
\vdots & \vdots & \vdots & \ddots & \vdots\\
1 & \gm^{n-1} & \gm^{2(n-1)} & \cdots & \gm^{(n-1)(n-k-1)}
}$$


Now denote $G=(g_{i,j})_{\substack{1\leq i\leq k\\ 1\leq j\leq n}}$ and $H=(h_{i,j})_{\substack{1\leq i\leq n-k\\ 1\leq j\leq n}}$. So when $i=1$ $$\sum_{k=1}^ng_{1,k}h_{k,1}=\sum_{k=1}^n1\times 1=0,\quad \sum_{k=1}^ng_{1,k}h_{k,j}=\sum_{k=1}^n h_{k,j}=\sum_{k=2}^n \gm^{(k-1)(j-1)}=\sum_{k=1}^{n-1}\lt(\gm^{k}\rt)^{j-1}=\gm^{j-1}\sum_{k=0}^{n-2}\lt(\gm^{k}\rt)^{j-1}=0 \ $$where the last equality came using the claim. When $i>1$ then 
\begin{multline*}
	\sum_{k=1}^ng_{i,k}h_{k,j}=\sum_{k=2}^n \gm^{(i-1)(k-1)}\gm^{(k-1)(j-1)}=\sum_{k=1}^{n-1}\lt(\gm^{k}\rt)^{(i-1)+(j-1)}\\
	=\gm^{(i-1)+(j-1)}\sum_{k=0}^{n-2}\lt(\gm^{k}\rt)^{(i-1)+(j-1)}=\gm^{(i-1)+(j-1)}\sum_{k=0}^{n-2}\lt(\gm^{k}\rt)^{\om}=0
\end{multline*}Since for any element $\alpha\in \bbF_q$ then $\alpha^{n-1}$ so we can take $\om=(i-1)+(j-1)\pmod{n-1}$.  Hence  again the last equality came using the claim. Hence we get that $GH^T=0$. Therefore $C^{\perp}$ is $RS[n,n-k]$.
}
%%%%%%%%%%%%%%%%%%%%%%%%%%%%%%%%%%%%%%%%%%%%%%%%%%%%%%%%%%%%%%%%%%%%%%%%%%%%%%%%%%%%%%%%%%%%%%%%%%%%%%%%%%%%%%%%%%%%%%%%%%
% Problem 5
%%%%%%%%%%%%%%%%%%%%%%%%%%%%%%%%%%%%%%%%%%%%%%%%%%%%%%%%%%%%%%%%%%%%%%%%%%%%%%%%%%%%%%%%%%%%%%%%%%%%%%%%%%%%%%%%%%%%%%%%%%

\begin{problem}{%problem statement
		Chapter 5
	}{p1
		% problem reference text
	}
	%Problem
	Ex 5.8		
\end{problem}

\solve{
	%Solution
	\begin{enumerate}
		\item Let the set $S\subseteq \bbF_q^n$ is $t$-wise independent source for some $t$, $1\leq t\leq n$. If we take any random vector  $v\in S$ where $v=(v_1,\dots, v_n)$ then for any $I\subseteq [n]$ where $|I|=t$, we have  $Pr[X_i=v_i\ \forall \ i\in I] =\frac{1}{q^t}$. Since the probability is same for all $I$ and for all vectors $v\in S$ that means for all $I\subseteq [n]$ where $|I|=t$, $S$ projected to $I$ has each vectors of $\bbF_q^t$ appear same number of times Hence the set $S$ is $t$-wise independent.\parinn
		
		Let the set $S\subseteq \bbF_q^n$ is $t$-wise independent for some $t$, $1\leq t\leq n$. So for all $I\subseteq [n]$ where $|I|=n$ the set $S$ projected to $I$ has each vector of $\bbF_q^t$ appear same number of times. Let each vector of $\bbF_q^t$ appear $m$ number of times. So $|S|=mq^t$. So if we take any random code $v\in S$ where $v=(v_2,\dots,v_n)$ for all $I\subseteq [n]$ $$Pr[X_i=v_i\ \forall \ i\in I]=\frac{\#\text{vectors for which the $i$-th position element equals to $v_i$ for all $i\in I$}}{mq^t}$$For all vectors $v\in S$ for which the $i$-th position equals to $v_i$ for all $i\in I$. if we project them to $I$ then they give the $\bbF_q^t$ vector $(v_i)_{i\in I}$. This vector in $\bbF^t_q$ appears $m$ number of times. So the number of vectors $v\in S$ for which the $i$-th position equals to $v_i$ for all $i\in I$ is $m$. Hence $Pr[X_i=v_i\ \forall \ i\in I]=\frac{m}{mq^t}=\frac1{q^t}$. Hence the set $S$ is $t$-wise independent source.
		\item Here there should be a condition that the code $C$ has at least one code which nonzero at every position cause otherwise if we take $C=\langle e_q\rangle $ then for all positions execpt the first position it is gurrented to be a 0. \parinn
		
		Now since there is a vector which has nonzero element at every positon if we take any vector randomly then for any random position all alphabets can be present in that position. Because since there is a code in $C$ which has nonzero element at every positon for any position if we multiply the element in that position with all the elements of $\bbF_q$ then it goes through all the elements of $\bbF_q$. So for any random vector and any random position the alphabet $Pr[X=\alpha]=\frac{1}{q}$, $\alpha\in\bbF_q$. Hence any cpde $[n,k]_q$ is 1-wise independent source.
		\item The dual of $MDS$ code is also a $MDS$ code. To prove this let a $MDS$ code is $[n,k]$. Then the dual of this code has dimenstion $n-k$ and distance at most $n-(n-k)+1=k+1$. So if we show that the generator matrix of $[n,k]$ every $k$ columns are linearly independent we are done. Suppose not. WLOG the first $k$ columns are linarly dependent. Hence there is a linear combination of the rows of the generator matrix such that the first $k$ elements becomes 0. Hence this linear combination gives a code which has at least $k$ many zeros. So it has weight at most $n-k$. But the $MDS$ code $[n,k]$ has distance $n-k+1$. Hence contradiction. So the dual of $MDS$ code is also $MDS$ and has distance $k+1$. By Ex 2.13 we have the code $[n,k]$ is $k$ wise independent. Hence by part (a) we have $MDS$ code is $k$-wise independent source.
	
		
		\item Since the $MDS$ code $[n',k]_{2^m}$ can be converted to a linear code $[n'm,km,d]_2$ by Ex 2.17(b). Since $[n',k]_{2^m}$ is $k$-wise independent we claim that $[n'm,km,d]_2$ is also $k$-wise independent. For any code $c\in [n'm,km,d]_2$ and for any $I\subseteq [n'm]$  $$Pr[X_i=v_i\ \forall \ i\in I]=\frac{\#\text{vectors for which the $i$-th position element equals to $c_i$ for all $i\in I$}}{2^{km}}$$ Now since $[n',m]_{2^m}$ is a $MDS$ code there is also codes with no zero elements. Hence for every $X_i=c_i$ there are exactly half of all codes which have the value $c_i$ at $i$-th position. Hence for $X_i=c_i, X_j=c_j$ where $i\neq j$ there exactly half of the codes with $c_i$ at $i$-th position. Hence like this there exactly $2^{km-k} $ many codes which have $X_i=c_i$ for all $i\in I$. Hence $Pr[X_i=v_i\ \forall \ i\in I]=\frac{2^{km-k}}{2^{km}}=\frac{1}{2^k}$. Hence $[n'm,km,d]_2$ is also $k$-wise independent source. \parinn
		
		So now we take $n'=\frac{n}{m}$ and $m=1+\log n\implies 2^m=2\times 2^{\log n}=2n$ then the code $[n'm,km,d]_2=[n,km,d]_2$ has size  $2^{mk}=\lt(2^m\rt)^k=(2n)^k$. Hence this code $[n,km,d]_2$ is $k$-wise independent. Hence $\log((2n)^k)=k\log (2n)=k(1+\log n)$ random bits are enough to compute $n$ random bits that are $k$-wise independent.
		
		We have to have $k(\log n-\log\log n+O(1))$ random bits. If we take $m=\log n -\log\log n+O(1)=\log\lt(\frac{n2^{O(1)}}{\log n}\rt)$ then the field size at first becomes $2^m=\frac{n2^{O(1)}}{\log n}$. Then the size of the code becomes $2^{km}=\lt(2^m\rt)^k=\lt(\frac{n2^{O(1)}}{\log n}\rt)^k$ then the number of random bits needed is $$\log \lt(\lt(\frac{n2^{O(1)}}{\log n}\rt)^k\rt)=k(\log n -\log\log n+O(1))$$ Since in the original $MDS$ code $[n',k]_{2^m}$ where $n'=\frac{n}{m}$, $m=\log n -\log\log n+O(1)$ the code $[n'm=n,km=k(\log n -\log\log n+O(1)),d]_2$ is $k$-wise independent code of size $\lt(\frac{n2^{O(1)}}{\log n}\rt)^k$. So if we take 1 in place of $O(1)$ then we have a code of size $\lt(\frac{2n}{\log n}\rt)^k$ which is $k$-wise independent.
		
		
		
		\item Let $p=\frac{1}{2^k}\implies \log \frac1p=k$ and $X-1,\dots, X_n$ be $tk$ wise independent random variables. Now we group the random variables into $\frac{n}{k}$ groups where each group contains $k$ random variables. Let the groups are $U_1,\dots, U_{\frac{n}{k}}$. Now we create new random variables $Z_1,\dots, Z_{\frac{n}{k}}$ where $Z_i=\bigwedge\limits_{i\in U_i}X_i$. We claim that $Z_i$'s are $t$-wise independent and $p$-biased. They are $p$-biased because $$Pr[Z_i=1]=\prod_{i\in U_i}Pr[X_i=1]=\prod_{i\in U_i}\frac12=\frac1{2^k}=p$$\parinn
		
		So we need to show that $Z_i$'s are $t$-wise independent. For any $I\subseteq [n-k]$ where $|I|=t$ $$Pr\lt[ \bigwedge_{i\in I}(Z_i=z_i) \rt]=Pr\lt[ \bigwedge_{i\in I}\lt(\bigwedge_{j\in U_i}X_j=z_i\rt) \rt]$$since $Z_i$ are disjoint and each $X_i$ are independent. Since the random variables $X_i$ are $tk$-wise independent we can say from here that $z_I$'s are $t$-wise independent.
	\end{enumerate}
}


%%%%%%%%%%%%%%%%%%%%%%%%%%%%%%%%%%%%%%%%%%%%%%%%%%%%%%%%%%%%%%%%%%%%%%%%%%%%%%%%%%%%%%%%%%%%%%%%%%%%%%%%%%%%%%%%%%%%%%%%%%
% Problem 6
%%%%%%%%%%%%%%%%%%%%%%%%%%%%%%%%%%%%%%%%%%%%%%%%%%%%%%%%%%%%%%%%%%%%%%%%%%%%%%%%%%%%%%%%%%%%%%%%%%%%%%%%%%%%%%%%%%%%%%%%%%

\begin{problem}{%problem statement
		Chapter 5
	}{p2% problem reference text
	}
	%Problem		
	Ex 5.15
\end{problem}

\solve{
	%Solution
	Given that $m_1\neq m_2$. Then there exists at least one prime $p_i$ where $i\leq k$ such that $m_1-m_2\pmod{p_i}\neq 0$. Now since $m_1,m_2\in \bbZ_K$ we have $m_1-m_2\in \bbZ_K\implies m_1-m_2\leq K$. So therefore for all primes $p_j$ where $k<j\leq n$ we have $m_1-m_2\pmod{p_j}=m_1-m_2\neq 0$ Hence $b_i=1$ and for all $k<i\leq n$, $b_i=1$. Therefore $$\prod_{i=1}^n p_i^{b_i}\geq p_i\prod_{j=k+1}^np_j>\prod_{j=k+1}^np_j=\frac{N}{K}$$
	
	Therefore $b_j=1$ for at least $n-k+1$ many positions (one for $i$ and the positions form $k+1$ to $n$). Therefore $E(m_1)-E(m_2)\neq 0$ at atleast $n-k+1$ many positions. So $E(m_1)$ and $E(m_2)$ differ at atleast $n-k+1$ many positions.
}
%%%%%%%%%%%%%%%%%%%%%%%%%%%%%%%%%%%%%%%%%%%%%%%%%%%%%%%%%%%%%%%%%%%%%%%%%%%%%%%%%%%%%%%%%%%%%%%%%%%%%%%%%%%%%%%%%%%%%%%%%%
% Problem 7
%%%%%%%%%%%%%%%%%%%%%%%%%%%%%%%%%%%%%%%%%%%%%%%%%%%%%%%%%%%%%%%%%%%%%%%%%%%%%%%%%%%%%%%%%%%%%%%%%%%%%%%%%%%%%%%%%%%%%%%%%%

\begin{problem}{%problem statement
		Chapter 5
	}{p3% problem reference text
	}
	%Problem		
	Ex 5.16
\end{problem}

\solve{
	%Solution
	\begin{enumerate}
		\item We have $f(X+Z)=\sum\limits_{i=0}^{t} r_i(X)Z^i$. Now differentiating $f$ with respect to $Z$ we have $$f'(X+Z)=\sum_{i=0}^{t-1}(i+1)r_{i+1}(X)Z^i$$
		Let for $n=k-1$ we have $$f^{(k-1)}(X+Z)=\sum_{i=0}^{t-k+1} \frac{(i+k-1)!}{i!} r_{i+k-1}(X)Z^i$$
		Denote $\dfrac{(i+k-1)!}{i!} r_{i+k-1}(X)=g_{i}(X)$. Then for $n=k$ we have
		\begin{align*}
			f^{(k)}(X+Z) & = \sum_{i=0}^{t-k} (i+1)g_{i+1}Z^i=\sum_{i=0}^{t-k} \frac{((i+1)+k-1)!}{i!} r_{(i+1)+k-1}(X)Z^i\\
			& = \sum_{i=0}^{t-k}\frac{(i+k)!}{i!} r_{i+k}(X)Z^i
		\end{align*}
		Hence by mathematical induction we have $$f^{(n)}(X+Z)=\sum_{i=0}^{t-n}\frac{(i+n)!}{i!} r_{i+n}(X)Z^i$$
		Therefore $$f^{(n)}(X)=f^{(n)}(X+0)=\sum_{i=0}^{t-n}\frac{(i+n)!}{i!} r_{i+n}(X)0^i=\frac{n!}{0!}r_{n}(X)=n!r_n(X)$$
		\item Let $char(\bbF_q)=m$. So $j\geq m$. Hence $j!=j(j-1)\cdots (m+1)m(m-1)!=mk$ where $k=j(j-1)\cdots (m+1)(m-1)!$. Since $f^{(j)}(X)=j!r_j(X)=m\lt( kr_j(X) \rt)\equiv 0$.
		\item Given that $f(\alpha)=0$. Hence $X-\alpha\mid f(X)$. So $f(X)=(X-\alpha)^mq(X)$ for some $m\in\bbN$ such that $q(\alpha)\neq 0$. Now $f'(X)=m(X-\alpha)^{m-1}q(X)+(X-\alpha)^mq'(X)$. Since $f'(\alpha)=0$ we have $m\geq 2$. Hence $(X-\alpha)^2$ divides $f(X)$. Hence the given statement is true for $j=2$. Let this statement is true form $j=k<char(\bbF_q)$. So for all $0\leq i<k$, $f^{(i)}(\alpha)=0$ and hence $(X-\alpha)^k$ divides $f(X)$. Let for all $0\leq i< k+1$, $f^{(i)}(\alpha)=0$. Since $f(\alpha)=0$ we have $f(X)=(X-\alpha)q(X)$. \begin{align*}
			f'(X)      & =(X-\alpha)q'(X)+q(X)             \\
			f^{(2)}(X) & = (X-\alpha)q^{(2)}(X)+q^{(1)}(X) \\
			f^{(3)}(X) & = (X-\alpha)q^{(3)}(X)+q^{(2)}(X) \\
			\vdots \ \ & \qquad \vdots                     \\
			f^{(k)}(X) & =(X-\alpha)q^{(k)}(X)+q^{(k-1)}(X)
		\end{align*}For all $0\leq i\leq k$ $f^{(i)}(\alpha)=0\implies$ $\forall\ 0\leq i< k$ $q^{(i)}(\alpha)=0$. By induction hypothesis we have $(X-\alpha)^{k}\mid q(X)$. Hence $q(X)=(X-\alpha)^{k}g(X)$. Therefore $f(X)=(X-\alpha)(X-\alpha)^{k}g(X)=(X-\alpha)^{k+1}g(X)$. Hence $(X-\alpha)^{k+1}$ divides $f(X)$. Therefore the given statement is true for all $j\leq char(\bbF_q)$.
		\item Suppose there are more than $\lfloor \frac{t}{m}\rfloor $ distinct elements $\alpha\in \bbF_{q}$ such that $f^{(j)}(\alpha)=0$ for all $0\leq j<m$. Such elements is then at least $\lfloor \frac{t}{m}\rfloor +1$. Let these elements are $\alpha_1,\dots,\alpha_k$. By part (3) for each of these elements $\alpha\in \bbF_q$ we have that $(X-\alpha)^m$ divides $f(X)$. Then we have $g(X)=\prod_{i=1}^k(X-\alpha)^m$ divides $f(X)$. Now $\deg((X-\alpha_i)^m)=m$. So $\deg(g)=km$. Now $$km\geq \lt(\lt\lfloor \frac{t}{m}\rt\rfloor+1  \rt)m\geq t+m>t$$But $\deg(f)=t$. Hence contradiction. Therefore there can be at most $\lfloor \frac{t}{m}\rfloor $ distinct elements satisfying $f^{(j)}(\alpha)=0$ for all $0\leq j<m$.
	\end{enumerate}
}

%%%%%%%%%%%%%%%%%%%%%%%%%%%%%%%%%%%%%%%%%%%%%%%%%%%%%%%%%%%%%%%%%%%%%%%%%%%%%%%%%%%%%%%%%%%%%%%%%%%%%%%%%%%%%%%%%%%%%%%%%%
% Problem 8
%%%%%%%%%%%%%%%%%%%%%%%%%%%%%%%%%%%%%%%%%%%%%%%%%%%%%%%%%%%%%%%%%%%%%%%%%%%%%%%%%%%%%%%%%%%%%%%%%%%%%%%%%%%%%%%%%%%%%%%%%%


\begin{problem}{%problem statement
		Chapter 5
	}{p4% problem reference text
	}
	%Problem		
	Ex 5.17
\end{problem}

\solve{
	%Solution
	Each alphabet of the new code is a $m$-tuple where each element is from $\bbF_q$. So for any polynomial $f_m$ of a message $m$ and a value $\alpha$, $(f_m(\alpha),f_m^{(1)}(\alpha),\dots,f_m^{(m-1)}(\alpha))$ is an alphabet of this code system. Hence code length is $n$.	Since the number of code is still same $|C|=q^k=(q^m)^{\frac{k}{m}}$. So dimension becomes $\frac{k}{m}$.
		
	Consider the polynomial $f_{m_1}(X)-f_{m_2}(X)=f_{12}(X)$. Now $\deg(f_{12}(X))=k-1$. Now take the $m$-tuple $(f_{12}(X),f^{(1)}_{12}(X),\dots,f_{12}^{(m-1)}(X))=u_{12}(X)$. For any $\alpha\in\bbF_q$  $u_{12}(\alpha)=\ov{0}$ when for all $0\leq i\leq m-1$ we have $f^{(i)}_{12}(\alpha)=0$. Such distinct $\alpha\in\bbF_q$ can exist at most $\lt\lfloor \frac{k-1}{m}\rt\rfloor $ many. So distance of this new code is  $d\geq n-\lt\lfloor \frac{k-1}{m}\rt\rfloor$. Now let $l=\lt\lceil\frac{k-1}{m}\rt\rceil$. Then take the polynomial $g(X)=\prod\limits_{i=1}^l(X-\alpha_i)^m$. So $(g(X),g^{(1)}(X),\dots,g^{(m-1)}(X))=\ov{0}$ for $l$ many values. Hence in the code formed from $g(X)$ the weight is $n-(f_{12}(X),f^{(1)}_{12}(X),\dots,f_{12}^{(m-1)}(X))$. Therefore the new code is $\lt[n,\frac{k}{m},n-\lt\lfloor \frac{k-1}{m}\rt\rfloor  \rt]$.
}
\end{document}
