\documentclass[12pt]{article}
\usepackage{amsmath}
\usepackage{anyfontsize}
\usepackage{amsfonts}
\usepackage[paperheight=7in,paperwidth=7in,margin=0.5in]{geometry}
\usepackage[x11names]{xcolor}
\usepackage{tikz}
\usepackage{tcolorbox}
\usepackage{graphicx}
\graphicspath{{images/}}




\begin{document}


\thispagestyle{empty}

\begin{tikzpicture}[remember picture,overlay]
\coordinate [below=12cm] (midpoint) at (current page.north);
\node at (current page.north west){\begin{tikzpicture}[remember picture,overlay]
\node[anchor=north west,inner sep=0pt] at (0,0) {\includegraphics[width=\paperwidth]{2.png}}; 
\end{tikzpicture}};
\end{tikzpicture}



\vspace*{2cm}

\huge{Let us define a sequence $\{b_k\}_{k\geq 1} $ such that $k\alpha \equiv b_k\ (\text{ mod } 2\pi)$ for all $k\in\{1,\cdots,n\}$. Now we can consider $\frac{V_n (\alpha)}{n}$ as the probability of sign change occurrence between $k\alpha$ and $(k+1)\alpha$. Hence it is same as the probability of occurrence of $b_k$ in $\mathcal{I}$ which is equal to $$\dfrac{\text{Length of }(\mathcal{I})}{\text{Length of }[0,2\pi]}=\dfrac{\text{Length of }(\mathcal{I})}{2\pi}$$}
\end{document}
