\documentclass[12pt]{article}
\usepackage{amsmath}
\usepackage{times}
\usepackage{anyfontsize}
\usepackage{amsfonts}
\usepackage[paperheight=7in,paperwidth=7in,margin=0.5in]{geometry}
\usepackage[x11names]{xcolor}
\usepackage{tikz}
\usepackage{tcolorbox}
\usepackage{graphicx}
\graphicspath{{images/}}
%\usepackage[pages=some]{background} \backgroundsetup{ scale=1, color=black, opacity=0.4, angle=0, contents={ \includegraphics[width=\paperwidth,height=\paperheight]{(1).jpg} }% }

%to set gradiant background for whole document
%\usepackage{background}
%\usepackage{blindtext}
%\backgroundsetup{ scale=1, angle=0, opacity=1, contents={\begin{tikzpicture}[remember picture,overlay] \path [inner color = DarkOliveGreen1,outer color = SpringGreen1] (current page.south west)rectangle (current page.north east);  \end{tikzpicture}} }

\usepackage{graphicx}
\graphicspath{{images/}}

\begin{document}
	
\newtcolorbox{mybox}{colback=red!5!white,colframe=red}

%for gradiant background this code below


%\begin{tikzpicture}[remember picture,overlay]
%\path [inner color=Yellow,outer color=Apricot] (current page.north east)rectangle (current page.south west);
%\end{tikzpicture}

%for adding picture in background this code below

\begin{tikzpicture}[remember picture,overlay]
\coordinate [below=12cm] (midpoint) at (current page.north);
\node at (current page.north west){\begin{tikzpicture}[remember picture,overlay]
\node[anchor=north west,inner sep=0pt] at (0,0) {\includegraphics[width=\paperwidth]{1.png}}; 
\end{tikzpicture}};
\end{tikzpicture}

%\begin{tikzpicture}
%\tikz[remember picture,overlay] \node[opacity=0.3,inner sep=0pt] at (current page.center){\includegraphics[width=\paperwidth,height=\paperheight]{Q.jpg}}; 
%\end{tikzpicture}



%\color{blue}
	\begin{center}
		\thispagestyle{empty}
		%\pagecolor{BrickRed}

%\vspace*{2cm}

\Huge{\textbf{\underline{Sunday Special \#17}}}
		
		\vspace*{1cm}

		{\Huge\textbf{DM me Your Answer!}}
		\vspace*{0.5cm}
	\end{center}
		
		{\Huge Let $0\leq\alpha\leq \pi $. $V_n (\alpha) $ denote the number of sign changes in the sequence $\cos\alpha,$ $\cos2\alpha,$ $\cos3\alpha,$ $\ldots,$ $\cos n\alpha $. Then prove that $$\lim\limits_{n\to\infty}\dfrac{V_n (\alpha)}{n}=\dfrac{\alpha}{\pi}$$}
		
		\vspace{0.5cm}
	\begin{center}	
		

{\Huge
		
		$$\boldsymbol{\sum \limits_{i=0}^{Creative} Math_i = Solving}$$}
		
	%	\vspace{1cm}
		
%		\begin{mybox}\Huge{\begin{center}\textbf{\textcolor{red}{Solution $\to$}} \end{center}}\end{mybox}
	\end{center}

\end{document}
