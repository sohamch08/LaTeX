\documentclass[12pt]{article}
\usepackage{amsmath,csquotes}
\usepackage{times}
\usepackage{anyfontsize}
\usepackage{amsfonts}
\usepackage[paperheight=7.2in,paperwidth=7.2in,margin=0.2in]{geometry}


\usepackage[x11names]{xcolor}
\usepackage{tikz}
\usepackage{tcolorbox}
\usepackage{graphicx}
\graphicspath{{images/}}
\newcommand{\mR}{\mathbb{R}}
\newcommand{\mN}{\mathbb{N}}
\newcommand{\mQ}{\mathbb{Q}}
\pagestyle{empty}

\usepackage{eso-pic}
\newcommand\BackgroundPic{%
	\put(0,0){%
		\parbox[b][\paperheight]{\paperwidth}{%
			\vfill
			\centering
			\includegraphics[width=\paperwidth,height=\paperheight,%
			keepaspectratio]{1.jpg}%
			\vfill
}}}

\newtcolorbox{mybox}{colback=black!5!white,colframe=black}
\begin{document}
	\AddToShipoutPicture*{\BackgroundPic}
	
	\begin{center}
		
		\huge{\textbf{Send me Your Answer!}}
		
		\vspace*{1cm}
		
		%Problem sent to me by $@fragzzqt$
		
		%\vspace*{1cm}
		
	\end{center}
	
	
	\Huge{Suppose $g(x)$ is a third degree polynomial with coefficient of $x^3$ as 1. Its all 3 roots are distinct (There was an error in the question post). Let $h(x)=x^2+x+a$.( where $a$ is some integer) If $g(h(x))$ has no real roots what is the lower bound of $g(a) ?$}
	
	\begin{center}
		\vspace{1cm}
		
		\Huge{$$\boldsymbol{\sum \limits_{i=0}^{Creative} Math_i = Solving}$$}
		
		\vspace{1cm}
		
		\begin{mybox}\Huge{\begin{center}\textbf{\textcolor{black}{Solution $\to$}} \end{center}}\end{mybox}
	\end{center}
	\pagebreak 
	
	As $g(x)$ has 3 distinct roots let its 3 roots are $p,q,r$. Hence $$g(x)=(x-p)(x-q)(x-r)$$As the coefficient of $x^3$ is 1 hence as $x\to \infty$, $g(x)\to \infty$. Similarly $h(x)\to \infty$ as $x\to\infty$. As it is given that $g(h(x))$ has no real roots we can say $g(h(x))>0$ $\forall x\in\mR$. Hence $h(x)-c$ has only imaginary roots where $c=\{p,q,r\}$. Hence the discriminant $1-4(a-c)<0\implies \min h(x)=a-\frac14 >c$ where $c=\{p,q,r\}$. Hence   $g(a)=(a-p)(a-q)(a-r)>\frac1{64}$. Therefore $\frac1{64}$ is the infimum of $g(a)$.
\end{document}