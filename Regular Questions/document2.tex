\documentclass[12pt]{article}
\usepackage{amsmath,csquotes}
\usepackage{times}
\usepackage{anyfontsize}
\usepackage{amsfonts}
\usepackage[paperheight=5in,paperwidth=5in,margin=0.1in]{geometry}


\usepackage[x11names]{xcolor}
\usepackage{tikz}
\usepackage{tcolorbox}
\usepackage{graphicx}
\graphicspath{{images/}}
\newcommand{\mR}{\mathbb{R}}
\newcommand{\mN}{\mathbb{N}}
\newcommand{\mQ}{\mathbb{Q}}
\pagestyle{empty}

\usepackage{eso-pic}
\newcommand\BackgroundPic{%
	\put(0,0){%
		\parbox[b][\paperheight]{\paperwidth}{%
			\vfill
			\centering
			\includegraphics[width=\paperwidth,height=\paperheight,%
			keepaspectratio]{1.jpg}%
			\vfill
}}}

\newtcolorbox{mybox}{colback=black!5!white,colframe=black}

\begin{document}
	\AddToShipoutPicture*{\BackgroundPic}
	%\color{white}
	
	
	
	Let $k$ be fixed. Consider tossing an unbiased coin until you get to see either $k+1$ Heads or $k+1$ Tails. It is evident, by the pigeonhole, that this experiment will last for at most $2k+1$ rounds (and also at least $k+1$) rounds.
	
	For any $k+1+\leq i\leq 2k+1$, let $P_i$ denotes the probability that the experiment ends at round $i$. Then, it is the case that either (a) round $i$ is the first time we see exactly $k+1$ Heads or (b) it is the first time we see exactly $k+1$ Tails. Both events are equiprobable with probability. Suppose we finally get heads. Then the remaining $k$ heads arrange in $i-1$ tosses in ${{i-1}\choose{k}}$ ways. Therefore $$P_i={{i-1}\choose{k}}2^{-i}\times 2={{i-1}\choose{k}}2^{-i+1}$$Now we know $\sum\limits_{i=k+1}^{2k+1}P_i=1$ and notice $$\sum\limits_{i=k+1}^{2k+1}P_i=\sum\limits_{i=k+1}^{2k+1}{{i-1}\choose{k}}2^{-(i-1)}=2^{-2k}\sum\limits_{i=0}^{k}{{k+i}\choose{i}}2^{k-i}=1$$Hence the given sum is equal to $2^{2k}=4^k$. [Proved
	\pagebreak 
	
\end{document}