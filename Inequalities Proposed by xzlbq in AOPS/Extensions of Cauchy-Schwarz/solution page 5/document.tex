\documentclass[12pt]{article}
\usepackage{amsmath}
\usepackage{anyfontsize}
\usepackage{titlesec}
\usepackage{amsfonts}
\usepackage[paperheight=7in,paperwidth=7in,margin=0.5in]{geometry}
\usepackage[x11names]{xcolor}
\usepackage{tikz}
\usepackage{tcolorbox}
\usepackage{graphicx}
\graphicspath{{images/}}

\titleformat*{\section}{\Huge\bfseries}
\setcounter{section}{1}

\begin{document}

\thispagestyle{empty}

\begin{tikzpicture}[remember picture,overlay]
\coordinate [below=12cm] (midpoint) at (current page.north);
\node at (current page.north west){\begin{tikzpicture}[remember picture,overlay]
\node[anchor=north west,inner sep=0pt] at (0,0) {\includegraphics[width=\paperwidth]{1.png}}; 
\end{tikzpicture}};
\end{tikzpicture}




\section{Solution 2(The Normal One) :-}

\huge{In  $n\alpha$ rotation the number of times full \linebreak[2mm] circle rotation occures $=\left\lfloor\frac{n\alpha}{2\pi}\right\rfloor$

In one full circle rotation sign change occures 2 times. Hence in $\left\lfloor\frac{n\alpha}{2\pi}\right\rfloor$ full rotation sign change occures $=2\left\lfloor\frac{n\alpha}{2\pi}\right\rfloor$

Now the rest angle is $$n\alpha-\bigg\lfloor\dfrac{n\alpha}{2\pi}\bigg\rfloor\times 2\pi$$

If we consider 0 as a change of sign in case of $\cos\left( \frac{\pi}{2}\right)$ and $\cos\left(\frac{3\pi}{2}\right)$ then:-}



\end{document}
