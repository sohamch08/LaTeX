\documentclass[11pt]{article}

\usepackage{verbatim,fullpage}
\usepackage{xcolor}
\usepackage[most]{tcolorbox}
\colorlet{verylightgray}{lightgray!70!}
\newtcbox{\enterkey}{on line, 
	boxsep=4pt, left=0pt,right=0pt,top=0pt,bottom=0pt,coltext=blue,
	colframe=white,colback=verylightgray,  
	highlight math style={enhanced},fontupper=\ttfamily
}

\usepackage{tikz}

\newcommand{\cd}[1]{\enterkey{#1}}

\begin{document}
	\section{Basic Commands}
\begin{itemize}
	\item \cd{git clone <url>} $\to$ Bring a repository that is hosted somewhere like Github into a folder on your local machine
	
	\item \cd{git add <file>} $\to$ Track your files and changes in Git
	
	\item \cd{git commit -m "<messege>" <file>} $\to$ Save your files in Git. You can commit like this $$\cd{git commit -m "messege" <file> -m "<description>"}$$ which also gives some description
	
	\item \cd{git push origin <branch-name>} $\to$ Upload Git commits to a remote repo, like Github
	
	\item \cd{git pull} $\to$ Download changes from remote repo to your local machine, the opposite of push
	\item \cd{git status} $\to$ Shows all the files that are created and commited and haven't been commited
\end{itemize}
\section{Initialize a Repo}
\begin{itemize}
	\item Create a folder let \cd{repo}
	\item Create a \cd{README.md} file 
	\item Run the command \cd{git init}
	\item Then add the file by \begin{align*}
		&\cd{git add README.md}\\
		&\cd{git commit -m "Created readme"}
	\end{align*}
	\item Now we can not just \cd{git push origin master} since git does not know where to push this since there is no connection. We have to create this connection
	\item Go to github and create an empty repository. Copy the given link
	\item Then run \cd{git remote add origin <copied-url>}
	\item Then run \cd{git push origin master}
	\item Later we will use only \cd{git push} but we will use something called upstream meaning this is where i want to push by default by \cd{git push -u origin master}
\end{itemize}
\end{document}
